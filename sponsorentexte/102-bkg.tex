\section*{Goldponsor und Aussteller}
\begin{flushright}
\includegraphics[width=0.7\textwidth]{102_BKG_Logo_RGB.png}
\end{flushright}
\noindent
Das {\bfseries Bundesamt für Kartographie und Geodäsie (BKG)} ist eine Behörde im Geschäftsbereich des Bundesministeriums des Innern und für Heimat (BMI). Es fungiert als zentraler Geodienstleister des Bundes. Zu den Aktivitäten des BKG gehören die Erfassung, Analyse, Kombination und Bereitstellung von Geodaten. Die Entwicklung und Verbreitung freier Software liegt ebenfalls im Aufgabenbereich des BKG. 

\noindent
Die Daten des BKG nutzen insbesondere Bundeseinrichtungen, die öffentliche Verwaltung, Wirtschaft und Wissenschaft. Doch von der Arbeit des BKG profitiert jeder Bürger in Deutschland. Ohne den Einsatz von Radioteleskopen, die das BKG weltweit betreibt, wäre eine Nutzung von Navigationsgeräten oder Kartenapps zur Navigation nicht möglich. Experten aus verschiedenen Bereichen wie beispielsweise Katastrophenvorsorge, Innere Sicherheit oder Umwelt verwenden die Daten des BKG für ihre Arbeit. 

\noindent
Das BKG setzt sich für eine offene Datenpolitik ein, wodurch die Verbreitung von Open Data gefördert wird. Dies schließt die Beratung anderer Bundesbehörden beim Umgang mit OSM-Daten ein.
Das BKG ist in Frankfurt am Main beheimatet und unterhält eine Außenstelle in Leipzig sowie geodätische Observatorien im In- und Ausland.

