\cleardoubleevenpage
\section*{Platinsponsor und Aussteller}
%\vspace*{-0.7\baselineskip}
%\begin{center}
  \includegraphics[width=0.7\textwidth]{003_dataport.png}
  %\end{center}
  \vspace{1.0\baselineskip}
  
\noindent
    {\bfseries Dataport - Anstalt des öffentlichen Rechts}
    \vspace{1.0\baselineskip}
    
\noindent
    {\bfseries Das Unternehmen}
    
\noindent    
Dataport ist der Partner für die Digitalisierung des öffentlichen Sektors. Für Schleswig-Holstein, Hamburg, Bremen und Sachsen-Anhalt. Und für die Steuerverwaltung in Niedersachsen und Meck\-lenburg-Vorpommern. Als IT-Dienstleister gestaltet Dataport den digitalen Wandel gemeinsam mit Ländern und Kommunen. Mit rund 5.000 Mitarbeiter:innen an acht Standorten. Das Unternehmen erzielte 2022 einen Umsatz von 1,18 Milliarden Euro.    
Dataport ist eine Anstalt des öffentlichen Rechts. Träger sind die Länder Schleswig-Holstein, Hamburg, Bremen, Sachsen-Anhalt, Niedersachsen und Meck\-lenburg-Vorpommern. Das Unternehmen arbeitet nicht gewinnorientiert, sondern strebt in enger Absprache mit den Trägern ein ausgeglichenes Betriebsergebnis an
\vspace{1.0\baselineskip}

\noindent
{\bfseries Der Auftrag}

\noindent
Dataport stellt dem öffentlichen Sektor alle benötigten IT-Services zur Verfügung. Dazu gehören der Betrieb von Infrastrukturen wie Rechenzentrum, Netze und Clients oder die zentrale Beschaffung von Informationstechnologie (IT). Außerdem die Entwicklung und der Betrieb von Software. Dataport unterstützt bei allen Aspekten der Digitalisierung. Durch umfassendes Consulting, Projektmanagement, Innovationsmanagement oder Geschäftsprozessmanagement.
\newpage
\noindent
{\bfseries Die Werte}
\vspace{1.0\baselineskip}

\noindent
{\em Digitale Sicherheit:}\\
Das Twin Data Center ist eines der sichersten Rechenzentren in Europa. Zertifiziert vom Bundesamt für Sicherheit in der Informationstechnik (BSI) und vom TÜViT. Der Betrieb von IT-Systemen folgt eindeutigen Prozessen und Regeln unter ständiger Kontrolle durch ein professionelles Sicherheitsmanagement. Das sorgt für Sicherheit.
\vspace{1.0\baselineskip}

\noindent
{\em Digitale Souveränität:}\\
Damit der Staat handlungsfähig und vertrauenswürdig ist, muss er stets die vollständige Hoheit über seine IT-Systeme und seine Daten behalten. Dafür sorgt Dataport mit hohen Sicherheitsstandards, klaren Datenschutzvorgaben und vorausschauender Unternehmenspolitik. Damit Bürger*innen und Unternehmen der Verwaltung ihre Daten anvertrauen können.
\vspace{1.0\baselineskip}

\noindent
{\em Kooperation:}\\
Seit der Gründung 2004 organisiert Dataport mit seinen Trägern erfolgreiche IT-Kooperationen. Der Dienstleister sorgt für Kooperationen auf verschiedenen Ebenen. Durch gemeinsam genutzte Infrastrukturen wie das Twin Data Center. Durch die gemeinsame Entwicklung von IT-Lösungen. Durch das Schaffen von Schnittstellen in IT-Systemen, die eine Zusammenarbeit ermöglichen. Länderübergreifend zwischen Bundesländern sowie ebenenübergreifend zwischen Bund, Ländern und Kommunen.
