\clearpage
\section*{Platinponsor und Aussteller}
%\vspace*{-0.7\baselineskip}
%\begin{center}
  \includegraphics[width=0.9\textwidth]{003_dataport.png}
  %\end{center}
  \vspace{1.0\baselineskip}
  
\noindent
    {\bfseries Dataport - Anstalt des öffentlichen Rechts}
    \vspace{1.0\baselineskip}
    
\noindent
Dataport ist der Partner für die Digitalisierung des öffentlichen Sektors.
Als Anstalt öffentlichen Rechts wird Dataport getragen von den Ländern Bremen,
Hamburg, Mecklenburg-Vorpommern, Niedersachsen, Sachsen-Anhalt und
Schleswig-Holstein sowie dem kommunalen IT-Verbund\linebreak Schleswig-Holstein.
Als IT-Dienst\-leister gestaltet Dataport den digitalen Wandel gemeinsam mit
den Ländern und Kommunen.

\noindent
Das Unternehmen arbeitet nicht gewinnorientiert, sondern strebt in enger
Absprache mit den Trägern ein ausgeglichenes Betriebsergebnis an.
Mit rund 5.500 Mitarbeiter:innen an acht Standorten erzielte das Unternehmen
2023 einen Umsatz von 1,36 Milliarden Euro. Dataport stellt dem öffentlichen
Sektor alle benötigten IT-Services zur Verfügung. Dazu gehören der Betrieb
von Infrastrukturen wie Rechenzentrum, Netze und Clients oder die zentrale
Beschaffung von Informationstechnologie (IT). Außerdem die Entwicklung und
der Betrieb von Software. Dataport unterstützt bei allen Aspekten der Digitalisierung.
Durch umfassendes Consulting, Projektmanagement, Innovationsmanagement
oder Geschäftsprozessmanagement.

\newpage
\noindent
Seit der Gründung 2004 organisiert Dataport mit seinen Trägern erfolgreiche
IT-Kooperationen. Der Dienstleister sorgt dafür auf verschiedenen Ebenen.
Durch gemeinsam genutzte Infrastrukturen wie das Twin Data Center – eines
der sichersten Rechenzentren in Europa. Durch die gemeinsame Entwicklung
von IT-Lösungen. Durch das Schaffen von Schnittstellen in IT-Systemen, die
eine Zusammenarbeit ermöglichen. Länderübergreifend zwischen Bundesländern
sowie Ebenen übergreifend zwischen Bund, Ländern und Kommunen.

