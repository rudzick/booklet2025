%\cleardoubleevenpage
\section*{Platinsponsor und Aussteller}
\begin{center}
  \centerline{\includegraphics[width=0.4\textwidth]{002_WhereGroup.jpg}}
\end{center}
\vspace*{-0.4cm}

\noindent
Die {\bfseries WhereGroup} ist eines der größten Softwareunternehmen in Deutschland, das für die Wirtschaft und die öffentliche Verwaltung GIS-Lösungen mit Open-Source-Software entwickelt. Als mittelständisches Unternehmen mit über 50 Mitarbeitenden an vier Standorten steht die {\bfseries WhereGroup} aus Überzeugung für Innovation und einen kooperativen Umgang mit ihren Kunden auf Augenhöhe.

\noindent
Das Unternehmen ist seit über 20 Jahren in der Welt der Open-Source-GIS-Anwendungen zuhause, steht in engem Austausch mit der Community und engagiert sich aus Überzeugung im FOSSGIS e.V., im QGIS-DE e.V. und bei der OSGeo.

\noindent
Das Angebots-Spektrum der {\bfseries WhereGroup} erstreckt sich von der Beratung über die Konzeption bis hin zur individuellen Entwicklung von dynamischen Kartenanwendungen sowie ihren Betrieb im Intra- und Internet. Damit ist eine kompetente Unterstützung in den Bereichen Geographische Informationssysteme (GIS), Web-GIS, Datenbanken, Standards, Interoperabilität und System-Integration aus einer Hand garantiert. Dazu gehört auch die nahtlose und professionelle Integration von Open-Source-Lösungen in proprietäre Software.

\noindent
Die {\bfseries WhereGroup} berät und begleitet bei der Planung, Migration und Einführung raumbezogener Informationssysteme, managt zahlreiche komplexe GIS-Projekte unterschiedlichster Art und implementiert Lösungen für internationale Konzerne, den Mittelstand und auf allen Ebenen der öffentlichen Verwaltung, darunter auch Landes- und Bundesbehörden.

\noindent
Auf dem Laufenden bleiben:
\begin{itemize}
\item Einblicke in unsere Arbeit im Blog: https://wheregroup.com/blog/    
\item WhereGroup-Newsletter abonnieren: https://wheregroup.com/unternehmen/newsletter/
\end{itemize}

\noindent
Außerdem bietet das Schulungsinstitut der WhereGroup - die FOSS Academy - praxisorientierte Schulungen zum Thema \glqq GIS mit Open-Source-Software\grqq\  an.

