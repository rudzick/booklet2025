\newpage
\renewcommand{\arraystretch}{1.4}
\section*{Vorträge am Mittwoch}\label{mittwoch}
\renewcommand{\conferenceDay}{\mittwoch}
\setPageBackground
\noindent\begin{tabular}{Z{0.85cm}Z{6.85cm}}
  & \multicolumn{1}{c}{\cellcolor{hellgelb} Anatomie}
  \tabularnewline
  10:30
  \talk{Die Welt als runde Sache auf der ebenen Karte}{Wolfgang Hinsch}
  \tabularnewline
  11:00
  \talk{Ehrenamt im FOSSGIS e.V.}{Hanna Krüger}
  \tabularnewline
  11:25
  \talk{Was ist Open Source?}{Marco Lechner}
  \tabularnewline
  11:40
  \talk{Was ist OpenStreetMap?}{Thomas Skowron}
  \tabularnewline
  \rowcolor{commongray}
  12:00 & \multicolumn{1}{c}{%
    \parbox[c]{24pt}{%
      \includegraphics[height=10pt]{restaurant}%
    }
    Mittagspause
  } \tabularnewline
\end{tabular}

\vspace{0.5\baselineskip}
\noindent\begin{tabular}{Z{0.85cm}Z{6.85cm}}
  & \multicolumn{1}{c}{\cellcolor{geoblau} Rundbau} \tabularnewline
  13:00
  \talk{Eröffnungs\-veranstaltung}{}
  \tabularnewline
  13:45
  \talk{\emph{Keynote} Pilgerstab in einer Hand, Brecheisen in der anderen}{Guillaume Rischard}
  \tabularnewline
  \rowcolor{commongray}
  14:30 & \multicolumn{1}{c}{%
    \parbox[c]{24pt}{%
      \includegraphics[height=10pt]{cafe}%
    }
    Kaffeepause} \tabularnewline
\end{tabular}
\newpage

\vspace{0.5\baselineskip}
\enlargethispage{1.0\baselineskip}
\renewcommand{\arraystretch}{1.3}
\noindent\begin{tabular}{lZ{2.0cm}Z{2.0cm}Z{2.0cm}}
  & \multicolumn{1}{c}{\cellcolor{geoblau} Rundbau}
  & \multicolumn{1}{c}{\cellcolor{hellgelb} Anatomie}
  & \multicolumn{1}{c}{\cellcolor{hellgruen} Weismannhaus}
  \tabularnewline
  15:00
  \talk{Von ArcGis nach QGIS}{Peter Heidelbach}
  \talk{2700 interaktive thematische Karten~-- ein Fall für Vector Tiles!}{Pirmin Kalberer}
  \talk{%Weniger ist mehr~-- Zur
  Auswahl darzustellender Elemente in der digitalen Kartographie}{Christoph \mbox{Hormann}}
  \tabularnewline
  15:30
  \talk{QGIS im Produktivbetrieb}{Thomas Baumann}
  \talk{Kartenherstellung zwischen Lizenzen, Daten, Programmcode und Darstellung}{Falk Zscheile}
  \talk{Umgang mit % vorhandenen und
  fehlenden Relevanzinformationen in OSM-Kartenstilen}{Michael \mbox{Reichert}}
  \tabularnewline
  16:00
  \talk{WebGIS kleiner Gemeinden mit QGIS-Server und Lizmap}{Günter Wagner}
  \talk{FAIRe Daten und FAIRe Software in der Biodiversitätsforschung}{Bernhard Seeger}
  \talk{Reliefdarstellung mit Höhenlinien}{Mathias Gröbe}
  \tabularnewline
\end{tabular}
\newpage
  
\begin{center}
  \renewcommand{\arraystretch}{1.3}
  \noindent\begin{tabular}{lZ{2.0cm}Z{2.0cm}Z{2.0cm}}
  & \multicolumn{1}{c}{\cellcolor{geoblau} Rundbau}
  & \multicolumn{1}{c}{\cellcolor{hellgelb} Anatomie}
  & \multicolumn{1}{c}{\cellcolor{hellgruen} Weismannhaus}
    \tabularnewline
    \rowcolor{commongray}
    16:30 & \multicolumn{3}{c}{%
    \parbox[c]{24pt}{%
      \includegraphics[height=10pt]{cafe}%
    }
    Kaffeepause} \tabularnewline
    17:00
    \talk{OSM-Daten in QGIS nutzen}{Astrid Emde}
    \talk{Ein einheitlicher Frontend"=Ansatz, um mehrere Routing"=Lösungen im Web-GIS zu nutzen}{Robert Klemm}
    \talk{GeoPortal Koblenz~-- digital, vielschichtig, maßgebend}{Andreas Weckbecker, Christine Dolezich}
    \tabularnewline
    17:30
    & \emph{Bitte Aushang beachten}
    \longTalk{2}{\emph{Demosession} Einführung zu GDAL/OGR: Geodaten mit der Kommandozeile verarbeiten}{Jakob Miksch}
    \talk{Der ÖREB-Kataster~-- eine Ode an offene Standards und Software}{Stefan Ziegler}
    \tabularnewline
    18:00
    \talk{Lightning Talks}{}
    &
    \talk{GeoPortal.rlp unchained}{Armin Retterath}
    \tabularnewline
    \rowcolor{commongray}
    19:00 &
    \multicolumn{3}{c}{%
      \parbox[c]{24pt}{%
        \includegraphics[height=10pt]{restaurant}%
      }
      Schwätzli uffem Campus (siehe Seite \pageref{schwaetzli})
    }
    \tabularnewline
  \end{tabular}
\end{center}
\renewcommand{\arraystretch}{1.0}
