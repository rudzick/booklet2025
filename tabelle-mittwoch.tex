\cleardoubleevenpage
%\newpage
%\enlargethispage{0.0\baselineskip}
\renewcommand{\arraystretch}{1.4}
\section*{Vorträge am Mittwoch}\label{mittwoch}
\renewcommand{\conferenceDay}{\mittwoch}
\setPageBackground
\noindent\begin{tabular}{Z{0.7cm}Z{6.85cm}}
  & \multicolumn{1}{c}{\HSeins}
  \tabularnewline
  10:00
  \talk{Eröffnung}{FOSSGIS e.V.}
  \tabularnewline
  10:10
  \talk{Keynote und Begrüßung}{Josef Hovenjuergen}
  \tabularnewline
 10:30
  \talk{25. Jahre FOSSGIS e.V. - eine Zeitreise durch das Vereinsleben}{Katja Haferkorn, Maik S., Christopher Lorenz}
  \tabularnewline
\end{tabular}

\noindent\begin{tabular}{Z{0.7cm}Z{3.0cm}Z{3.0cm}}
  & \multicolumn{1}{c}{\HSeins}
  & \multicolumn{1}{c}{\HSzwei}
  \tabularnewline
11:45
  \talk{Eine Reise durch die Geoportale Deutschlands}{Matthias Mohr}
  \talk{spatial.IO - Cloud-basierte Open-Source-Lösung zur Verwaltung räumlicher Daten}{Rebekka Lange}
  \tabularnewline
  12:20
  \talk{GIS-Schulprojekte in Zusammenarbeit mit kommunalen Gebietskörperschaften}{
Dietmar Holzner}
  \talk{eoAPI - eine skalierbare Geodateninfrastruktur}{
Felix Delattre}
  \tabularnewline
 \rowcolor{commongray}
 12:45 & \multicolumn{2}{c}{%
   \parbox[c]{24pt}{%
     \includegraphics[height=10pt]{restaurant}%
   }
   Mittagspause
} \tabularnewline
\end{tabular}
%
%\vspace{0.5\baselineskip}

\noindent\begin{tabular}{Z{0.7cm}Z{3.0cm}Z{3.0cm}}
  & \multicolumn{1}{c}{\HSdrei}
  & \multicolumn{1}{c}{\HSvier}
  \tabularnewline
  11:45
  \talk{GeoPandas - als Tool zur Basiskarten\-aktualisierung}{Markus Albrecht,\linebreak Markus Gruber}
  \talk{Keine Angst vor der GeoInfoDok 7 - 3A-Datenverarbeitung mit PostNAS}{Oliver Schmidt}
  \tabularnewline
  12:20
  \talk{aviary - ein generisches Python-Framework zur KI-Inferenz für Fernerkundungsdate}{Marius Maryniak}
  \talk{GDI per Knopfdruck: Automatisierung mit DevOps und Infrastruktur als Code}{Jakob Miksch}
  \tabularnewline
  \rowcolor{commongray}
  12:45 & \multicolumn{2}{c}{%
    \parbox[c]{24pt}{%
      \includegraphics[height=10pt]{restaurant}%
    }
    Mittagspause
  } \tabularnewline
\end{tabular}

\vspace{1.5cm}
\sponsorBoxA{401_terrestris.png}{0.45\textwidth}{5}{%
\textbf{Bronzesponsor und Aussteller}\\
\noindent\small {\bfseries terrestris GmbH \& Co. KG} terrestris ist Dienstleister für maßgeschneiderte Geoinformations-Lösungen mit Freier \& Open Source Software und deckt das gesamte Spektrum von Beratung, Konzeptionierung, Entwicklung bis hin zu Wartung \& Support ab. Wir entwickeln Lösungen, die den tatsächlichen Anforderungen unserer Kunden entsprechen.
\normalsize
}%


\noindent\begin{tabular}{Z{0.7cm}Z{3.0cm}Z{3.0cm}}
  & \multicolumn{1}{c}{\HSeins}
  & \multicolumn{1}{c}{\HSzwei}
\tabularnewline
  14:15
  \talk{basemap.de als Open Data - Neue Stile und Anwendungs\-beispiele}{Arnulf B. Bichler\linebreak (aka Christl)}
  \talk{Automatischer Import und Veröffent\-lichung von Betriebsmittelgeometrien mittels PyQGIS}{Philipp Opitz}
 \tabularnewline
  14:50
  \talk{Open Data in D: Perfekte Idee, halbherzige Umsetzung? Ein Erfahrungsbericht.}{Mike Elstermann}
\talk{QGIS-Werkzeuge und Python}{Isabelle Korsch}
  \tabularnewline
  15:25
 \talk{Open Data des BKG (II)}{Joachim Eisenberg}
  \talk{Schnupperkurs: Das Potential von QGIS mit der Python-Konsole freischalten}{Gordon Schlolaut}
    \tabularnewline
  \rowcolor{commongray}
  16:00 & \multicolumn{2}{c}{%
    \parbox[c]{24pt}{%
      \includegraphics[height=10pt]{cafe}%
    }
    Kaffeepause \emph{(30 min)}}
   \tabularnewline
\end{tabular}

\vspace{0.5cm}
\sponsorBoxA{406_Geofabrik_logo4.png}{0.16\textwidth}{3}{%
\textbf{Bronzesponsor}\\
\noindent\small Die {\bfseries Geofabrik GmbH} bietet die Datenauf\-bereitung von OpenStreetMap-Daten an, betreibt OpenStreetMap-basierte Serverdienste und hilft Ihnen bei der Installation eigener Karten-, Geocoding- oder Routingserver.
\normalsize
}%


\newpage

\renewcommand{\arraystretch}{1.4}
\noindent\begin{tabular}{Z{0.7cm}Z{3.0cm}Z{3.0cm}}
  & \multicolumn{1}{c}{\HSdrei}
  & \multicolumn{1}{c}{\HSvier}
  \tabularnewline
  14:15
  \talk{Superset - Business Intelligence meets Cartography}{Jan Suleiman}
  \talk{Lightning-Talks}{}
  \tabularnewline
  14:50
  \talk{Visualisierung von historischen Schiffsrouten mit unscharfer Datengrundlage}{Stefan Fuest, Andreas Gollenstede, Jennifer Tadge, Maximilian Herbers, Rieke Marie Kaiser}
  \talk{Wie MapLibre und Vektorkarten die Welt übernehmen}{Bart Louwers, Just van den Broecke}
    \tabularnewline
  15:25
  \talk{Automatisierte Verarbeitung von Daten der Meeresbodenkartografie mit QGIS}{Helge Staedtler}
  \talk{Vektor Tiles für Karten mit Echtzeitdaten}{Pirmin Kalberer}
  \tabularnewline
%  \rowcolor{commongray}
%  16:00 & \multicolumn{2}{c}{%
%   \parbox[c]{24pt}{%
%      \includegraphics[height=10pt]{cafe}%
%   }
%    Kaffeepause
% } \tabularnewline
\end{tabular}

%\subsubsection*{Anwendertreffen und Demo-Sessions am Mittwoch}%\label{demomittwoch}
\label{demomittwoch}
\renewcommand{\arraystretch}{1.4}
   \enlargethispage{10\baselineskip}
\justifying\setPageBackground

\noindent\begin{tabular}{Z{0.7cm}Z{2.0cm}Z{2.0cm}Z{2.0cm}}
  & \multicolumn{1}{c}{\BoFeins}
  & \multicolumn{1}{c}{\BoFzwei}
  & \multicolumn{1}{c}{\BoFdrei}
  \tabularnewline
14:15
  \talk{Studierende stellen ihre Arbeit vor \emph{(60min)}}{}
  \talk{BoF GeoNode-DE \emph{(60min)}}{Matthes Rieke, Henning Bredel}
  \talk{Lizmap Webclient \emph{(60min)}}{Günter Wagner}
  \tabularnewline
%  \rowcolor{commongray}
% 16:00 & \multicolumn{3}{c}{%
 %   \parbox[c]{24pt}{%
 %     \includegraphics[height=10pt]{cafe}%
%    }
%    Kaffeepause
%  } \tabularnewline
  \end{tabular}

\noindent\begin{tabular}{Z{0.7cm}Z{3.0cm}Z{3.0cm}}
  & \multicolumn{1}{c}{\HSeins}
  & \multicolumn{1}{c}{\HSzwei}
\tabularnewline
  16:30
     \talk{Small seeds - FOSS Communitys stärken!}{Paul Robben}
  \talk{Von Proprietär zu QGIS}{David Arndt}
  \tabularnewline
  17:05
  \talk{Governance von Open-Source-Software im öffentlichen Sektor: Make, Buy or Contribute?}{Christian Weidner}
  \talk{Von proprietär zu Open-Source -Umstellung der kommunalen GDI bei der Stadt Reutlingen}{Simon Kondic}
    \tabularnewline
  17:40
  \longTalk{2}{Panelgespräch: 25 Jahre FOSSGIS e.V. - was haben wir geschafft und wo wollen wir hin}{Torsten Friebe,\linebreak David Arndt, \linebreak Torsten Wiebke}
  \talk{Migration eines Auskunftssystems zu einer Open-Source Lösung mit QGIS}{Victor Ali Lagoa}
  \tabularnewline
  18:15
  \talk{}{}
%\talk{Projekt GEOrg bei den SWM: Mit Freier Software zur konzernweiten Geodatenplattform}{Nina Röckelein, Benedikt Seidl}
\talk{Projekt GEOrg bei den SWM: Mit Freier Software zur konzernweiten Geodatenplattform}{N. Röckelein, B. Seidl}
   \tabularnewline
\end{tabular}

\small
\noindent\begin{tabular}{Z{0.7cm}Z{6.0cm}}
  & \multicolumn{1}{c}{\BoFzwei}
  \tabularnewline
16:30
  \talk{Open Geodata und Open Source GIS Software in den Kultur- und Geisteswissenschaften \emph{(60min)}}{Anastasia Bauch, Carmen M. Enss, Klaus Stein}
\tabularnewline
  \end{tabular}
\normalsize

\newpage

\vspace{0.5\baselineskip}
\enlargethispage{4.0\baselineskip}
\renewcommand{\arraystretch}{1.4}
\renewcommand{\baselinestretch}{1.1}

\vspace{0.5\baselineskip}
%\enlargethispage{1.0\baselineskip}

\noindent\begin{tabular}{Z{0.7cm}Z{3.0cm}Z{3.0cm}}
  & \multicolumn{1}{c}{\HSdrei}
  & \multicolumn{1}{c}{\HSvier}
  \tabularnewline
  16:30
  \talk{Die Leistungsfähigkeit großer open source Sprachmodelle für Geoparsing-Aufgaben}{Juiwen Chang (Ariel)}
  \talk{Open ALKIS? – Oder was passiert, wenn der deutsche Föderalismus auf EU-Recht trifft}{Stefan Zaunseder}
  \tabularnewline
  17:05
  \talk{Entwicklung eines LLM-basierten Assistenten für die Suche nach Geodaten}{Matthes Rieke, Simeon Wetzel}
  \talk{Was wäre, wenn wir Algorithmen demokratisieren? Kollaborative Infrastrukturen für UDZ}{Rico H. Herzog}
  \tabularnewline
  17:40
  \talk{Künstliche Intelligenz als Unterstützung in geografische Applikationen}{Andrea Borghi, Marion Baumgartner}
  \talk{Metadaten für eine verantwortungsvolle und kritische Geodatenpraxis}{Ester Scheck}
  \tabularnewline
 18:15
  \talk{Open Data und KI im Einsatz: Geodaten für alle, nicht nur für Profis?}{Lisa Stubert, Klemens Maget}
  \talk{MOSIDI - Homogenisierung von offenen Daten für die kommunale Planung}{Sebastian Meier}
 %\tabularnewline
  \end{tabular}
  
\label{endevortraegemittwoch}
\renewcommand{\arraystretch}{1.4}
%\justifying\setPageBackground  
\newpage
