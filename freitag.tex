
% time: Friday 09:00
% URL: https://pretalx.com/fossgis2023/talk/fossgis2025-57964-foss-gis-foss-in-der-wind-und-solarbranche/

%
\newTimeslot{09:00}
\noindent\abstractHSeins{%
  Lars Roskoden, Björn Kilian%
}{%
  FOSS-GIS+FOSS in der Wind- und Solarbranche%
}{%
}{%
  Mit QGIS und PostgreSQL / PostGIS / PostNAS am Anfang eines Wind- oder Solarparkprojekts können
  auch die Prozesse in der Projektierung "`nach der Karte"' schneller, genauer, übersichtlicher, also
  effizienter gestaltet werden. Denn auch nach der Nutzung vom FOSS-GIS-Paket kann es mit
  FreeOpenSourceSoftware noch lange weitergehen. Dieser Praxisbericht zeigt, wie eine professionelle
  Zukunftsbranche für die Energieversorung eines Industrielands von FOSS profitieren kann.%
}%


%%%%%%%%%%%%%%%%%%%%%%%%%%%%%%%%%%%%%%%%%%%

% time: Friday 09:00
% URL: https://pretalx.com/fossgis2023/talk/fossgis2025-58095-qwc2-als-webgis-in-der-kommunalen-verwaltung-und-einblick-in-die-3d-funktionalitt/

%

\noindent\abstractHSzwei{%
  Daniel Cebulla, Sandro Mani%
}{%
  QWC2 als WebGIS in der kommunalen\linebreak Verwaltung und Einblick in die 3D Funktionalität%
}{%
}{%
  Dieser Vortrag befasst sich mit der Web-GIS-Anwendung QGIS Web Client 2 (QWC2). Seit Herbst 2024
  wird die bisherige 2D-Anwendung um eine 3D Ansicht erweitert. In diesem Vortrag wird zum einen der
  aktuelle Stand sowie ein Ausblick der Entwicklung der 3D Ansicht vorgestellt. Außerdem wird der
  Vortrag die Verwendung des QWC2 exemplarisch an ausgewählten Portalen präsentieren und dabei
  aufzeigen, welches Potential das System zur Anwendung in kommunalen Verwaltungen bietet.%
}%


%%%%%%%%%%%%%%%%%%%%%%%%%%%%%%%%%%%%%%%%%%%

% time: Friday 09:00
% URL: https://pretalx.com/fossgis2023/talk/fossgis2025-58272-nicht-von-neuen-algorithmen-berflutet-osm-daten-fr-waterwaymap-verarbeiten/

%

\noindent\abstractHSdrei{%
  Amanda McCann%
}{%
  Nicht von neuen Algorithmen überflutet: OSM-Daten für WaterwayMap verarbeiten%
}{%
}{%
  Heutzutage gibt es viele komplizierte Tools und Datenbanken. WaterwayMap.org zeigt, dass alle
  Flüsse und Bäche miteinander verbunden sind und verschiedene QA-Datensätze für den gesamten
  Planeten erstellt werden.
  In diesem Vortrag werden einige der grundlegenden Algorithmen erläutert, die zur Erstellung dieser
  Daten verwendet werden, sowie die sehr komplizierte Speichermethode "`einfach in den Speicher
  einlesen"'.
  Außerdem werden einige der neuen WWM-Funktionen behandelt.%
}%

%%%%%%%%%%%%%%%%%%%%%%%%%%%%%%%%%%%%%%%%%%%

% time: Friday 09:00
% URL: https://pretalx.com/fossgis2023/talk/fossgis2025-58036-osgeo-deegree-anwendertreffen/

%

\noindent\abstractAnwBoFeins{%
  Torsten Friebe%
}{%
  OSGeo deegree~-- Anwendertreffen%
}{%
}{%
  Zum Anwendertreffen sind Anwender:innen und Entwickler:innen herzlich eingeladen, die
  Netzwerkdienste wie WMS, WFS oder OGC API~-- Features mit dem OSGeo-Projekt
  deegree  bereits umsetzen oder dieses für die Zukunft planen.%
}%


%%%%%%%%%%%%%%%%%%%%%%%%%%%%%%%%%%%%%%%%%%%

% time: Friday 09:00
% URL: https://pretalx.com/fossgis2023/talk/fossgis2025-65233-bof-g3w-suite/

%

\noindent\abstractAnwBoFdrei{%
  Antonello Andrea, Walter Lorenzetti%
}{%
  BOF G3W-Suite%
}{%
}{%
  Diese BOF-Session soll Interessierte und Anwender in einem Raum zusammenbringen, um über die
  G3W-Suite zu diskutieren, Fragen an die Entwickler zu stellen oder einfach nur einen Blick darauf
  zu werfen.%
}%

%%%%%%%%%%%%%%%%%%%%%%%%%%%%%%%%%%%%%%%%%%%

% time: Friday 09:35
% URL: https://pretalx.com/fossgis2023/talk/fossgis2025-56862-untersttzung-des-regionalen-naturschutzes-in-luxemburg-mit-open-source-gis-software/

%
\newTimeslot{09:35}
\noindent\abstractHSeins{%
  Tobias Mosthaf%
}{%
  Unterstützung des regionalen Natur\-schutzes in Luxemburg mit Open Source GIS-Software%
}{%
}{%
  Der regionale Zweckverband für Umwelt- und Naturschutz SIAS in Luxemburg (https://www.sias.lu/)
  plant Naturschutzmaßnahmen für seine 21 Mitgliedsgemeinden und begleitet deren Umsetzung.
  Die 16 Mitarbeiter des SIAS führen dabei z.B. Kartierungen des Großen Feuerfalters (Tagfalter)
  durch oder planen das Anlegen von Streuobstwiesen. Wie sie dabei von Open Source GIS-Software
  unterstützt werden, soll in diesem Vortrag erläutert werden.%
}%


%%%%%%%%%%%%%%%%%%%%%%%%%%%%%%%%%%%%%%%%%%%

% time: Friday 09:35
% URL: https://pretalx.com/fossgis2023/talk/fossgis2025-57859-state-of-shogun-ein-flexibles-web-gis-framework/

%
\newLightningTimeslot{09:35}
\noindent\abstractHSzwei{%
  Daniel Koch%
}{%
  State of SHOGun: Ein flexibles Web-GIS Framework%
}{%
}{%
  SHOGun ist ein Anwendungsframework zum Aufbau von einfachen bis komplexen Geodateninfrastrukturen.
  Dabei kann SHOGun direkt und ohne spezifische Anpassungen verwendet oder aber auch nach Bedarf
  stark angepasst werden, um die Anforderungen spezifischer Projekte zu erfüllen.%
}%

\pagebreak
%%%%%%%%%%%%%%%%%%%%%%%%%%%%%%%%%%%%%%%%%%%

% time: Friday 09:40
% URL: https://pretalx.com/fossgis2023/talk/fossgis2025-58117-mapbender-neuigkeiten-aus-dem-projekt/

%
\newLightningTimeslot{09:40}
\noindent\abstractHSzwei{%
  Astrid Emde%
}{%
  Mapbender~-- Neuigkeiten aus dem Projekt%
}{%
}{%
  Im Mapbender-Projekt ist im letzten Jahr wieder viel passiert.
  Wir stellen Neuerungen vor
  - neues Element zum Routing
  - Auswahl von WMS Layer-Stilen im Ebenenbaum
  - Verbesserte Usability
  - REST API
  - Erweiterte Datenquellen
  - QGIS Plugin QGIS2Mapbender zur Administration von Mapbender aus QGIS heraus
  - Seriendruck
  Außerdem wollen wir einen Blick auf bestehende Mapbender-Lösungen richten und deren besondere
  Anforderungen oder Lösung vorstellen.%
}%


%%%%%%%%%%%%%%%%%%%%%%%%%%%%%%%%%%%%%%%%%%%

% time: Friday 09:45
% URL: https://pretalx.com/fossgis2023/talk/fossgis2025-58169-qgis-web-client-2-qwc2-neues-aus-dem-projekt/

%
\newLightningTimeslot{09:45}
\noindent\abstractHSzwei{%
  Sandro Mani%
}{%
  QGIS Web Client 2 (QWC2)~-- Neues aus dem Projekt%
}{%
}{%
  Dieser Vortrag stellt den QWC2 vor und zeigt, wie einfach es ist, eigene QGIS-Projekte im Web zu
  veröffentlichen. Es wird ein Überblick über die QWC2-Architektur gegeben. Dabei ist es auch eine
  Gelegenheit, die letzten neuen Funktionen, die im letzten Jahr entwickelt wurden, und die Ideen
  für zukünftige Verbesserungen zu entdecken.%
}%

\newpage
\enlargethispage{2.0\baselineskip}
%%%%%%%%%%%%%%%%%%%%%%%%%%%%%%%%%%%%%%%%%%%

% time: Friday 09:50
% URL: https://pretalx.com/fossgis2023/talk/fossgis2025-58072-openlayers-neues-und-ntzliches/

%
\newLightningTimeslot{09:50}
\noindent\abstractHSzwei{%
  Andreas Hocevar, Marc Jansen%
}{%
  OpenLayers~-- Neues und Nützliches%
}{%
}{%
  OpenLayers, das Urgestein der Open Source Web Mapping\linebreak Libraries, hat weiterhin steigende
  Nutzerzahlen und eine aktive Entwickler-Community. In diesem Vortrag zeigen wir neben neuen auch
  altbekannte nützliche Funktionen, aber auch spannende Anwendungsmöglichkeiten, die sich aus der
  Kombination mit anderen Libraries ergeben.%
}%

%%%%%%%%%%%%%%%%%%%%%%%%%%%%%%%%%%%%%%%%%%%

% time: Friday 09:35
% URL: https://pretalx.com/fossgis2023/talk/fossgis2025-57391-osm2world-updates-fr-den-3d-pionier/

%
\newSmallTimeslot{09:35}
\noindent\abstractHSdrei{%
  Tobias Knerr%
}{%
  OSM2World: Updates für den 3D-Pionier%
}{%
}{%
  OSM2World gehört zu den Pionieren der 3D-Darstellung von Open\-Street\-Map-Daten. Neue Features wie
  die Erzeugung von 3D Tiles und ein modernes Web-Frontend stellen sicher, dass der freie Renderer
  auch in Zukunft Maßstäbe setzt.%
}%


%%%%%%%%%%%%%%%%%%%%%%%%%%%%%%%%%%%%%%%%%%%

% time: Friday 09:35
% URL: https://pretalx.com/fossgis2023/talk/fossgis2025-58147-kartografie-verbesserungen-tips-und-tricks-in-qgis/

%
\newSmallTimeslot{09:35}
\noindent\abstractHSvier{%
  Andreas Neumann%
}{%
  Kartografie-Verbesserungen, Tips und Tricks in QGIS%
}{%
}{%
  In dieser Demo-Session sollen viele der jüngeren Verbesserungen im Bereich der QGIS Symbologie und
  der automatischen Beschriftung anhand von Kartenbeispielen aus der Praxis gezeigt werden. Des
  weiteren werden einige ältere, vermutlich wenig bekannte Möglichkeiten demonstriert, die schon
  länger in QGIS bestehen: selektives Maskieren von Beschriftungen und Symbolen, übersteuerbare
  Symbolebenen, komplexe Linientypen und Tips und Tricks rund um Expressions und daten-definierte
  Eigenschaften.%
}%



%%%%%%%%%%%%%%%%%%%%%%%%%%%%%%%%%%%%%%%%%%%

% time: Friday 10:10
% URL: https://pretalx.com/fossgis2023/talk/fossgis2025-57057-mit-qgis-zum-digitalen-prozess-1-gebudeeigenschaften/

%
\newTimeslot{10:10}
\noindent\abstractHSeins{%
  Marius Schäfer%
}{%
  Mit QGIS zum digitalen Prozess 1~-- Gebäudeeigenschaften%
}{%
}{%
  In der Ära der Digitalisierung rückt oft die Technologie, insbesondere digitale Dateiformate wie
  Word-Dokumente, Excel-Tabellen oder PDFs, in den Vordergrund. Doch bei genauerem Hinsehen wird
  deutlich, dass die eigentliche Revolution nicht nur in der bloßen Digitalisierung von Dateien
  liegt, sondern vielmehr im Durchdenken von Prozessen und der Implementierung von innovativen
  Methoden. Dabei kann QGIS und das Geodatenmanagement einen entscheidenden Beitrag leisten.%
}%


%%%%%%%%%%%%%%%%%%%%%%%%%%%%%%%%%%%%%%%%%%%

% time: Friday 10:10
% URL: https://pretalx.com/fossgis2023/talk/fossgis2025-58085-geostyler-arcgis-untersttzung-und-weitere-neue-features/

%
\newLightningTimeslot{10:10}
\noindent\abstractHSzwei{%
  Jan Suleiman, Benjamin Gerber%
}{%
  GeoStyler~-- ArcGIS Unterstützung und weitere neue Features%
}{%
}{%
  Dieser Vortrag zeigt die aktuellsten Neuerungen des OSGeo Community Projekts GeoStyler. Hier wird
  besonders auf die Unterstützung von ArcGIS Stilen eingegangen, und wie diese in offene Formate wie
  QGIS Styles, SLD, etc. konvertiert werden können. Zusätzlich zeigen wir die Weiterentwicklungen am
  GeoStyler-CLI, und was sonst noch u.a. im letzten GeoStyler Code Sprint umgesetzt wurde.%
}%

\pagebreak
%%%%%%%%%%%%%%%%%%%%%%%%%%%%%%%%%%%%%%%%%%%

% time: Friday 10:15
% URL: https://pretalx.com/fossgis2023/talk/fossgis2025-57880-state-of-geoserver-cloud/

%
\newLightningTimeslot{10:15}
\noindent\abstractHSzwei{%
  Nils Bühner, Daniel Koch%
}{%
  State of GeoServer Cloud%
}{%
}{%
  GeoServer ist ein vielseitiger Kartenserver, der in Microservice-Architekturen an seine Grenzen
  stößt. GeoServer Cloud als eigenes Projekt "`on top"' transformiert den GeoServer in skalierbare
  Einzelkomponenten für Container-Umgebungen wie Kubernetes.%
}%


%%%%%%%%%%%%%%%%%%%%%%%%%%%%%%%%%%%%%%%%%%%

% time: Friday 10:20
% URL: https://pretalx.com/fossgis2023/talk/fossgis2025-58214-neues-von-actinia-mit-knative-weiter-wolkenwrts/

%
\newLightningTimeslot{10:20}
\noindent\abstractHSzwei{%
  Carmen Tawalika, Anika Weinmann, Markus Neteler%
}{%
  Neues von actinia~-- mit Knative weiter wolkenwärts%
}{%
}{%
  "`Hallo, mein Name ist actinia. Ich bin eine REST API für GRASS GIS, die die Verwaltung und
  Visualisierung von Projects, Mapsets und Geodaten sowie die Ausführung von GRASS GIS Modulen
  ermöglicht und in einer Cloud-Umgebung installiert werden kann. Den aktuellen Trends folgend wird
  gezeigt, wie ich im ESA Projekt KNeo angepasst werde, um im Open Source Projekt Knative, einer
  Erweiterung für Kubernetes, installiert werden zu können. Um mehr Details zu erfahren, kommen Sie
  vorbei!"'%
}%

\pagebreak
%%%%%%%%%%%%%%%%%%%%%%%%%%%%%%%%%%%%%%%%%%%

% time: Friday 10:25
% URL: https://pretalx.com/fossgis2023/talk/fossgis2025-58020-stand-des-grass-gis-projekts-neuigkeiten-und-perspektiven/

%
\newLightningTimeslot{10:25}
\noindent\abstractHSzwei{%
  Markus Neteler%
}{%
  Stand des GRASS GIS Projekts: Neuigkeiten und Perspektiven%
}{%
}{%
  Wir geben einen Überblick über die neuesten Entwicklungen im GRASS GIS Projekt mit Fokus auf die
  Version 8.4.0: Der Begriff "`Location"' ist nun einfach "`Projekt"', die Support-Vektor-Machine
  Methode zur Bildklassifikation ist in GRASS-core verfügbar und optionale JSON Ausgabe neu in
  verschiedenen Tools. Verbesserungen in der grafischen Benutzeroberfläche und neue Werkzeuge für
  Jupyter Notebooks machen GRASS GIS zugänglicher und skalierbarer für Anwendungen von der
  Wissenschaft bis zur Industrie.%
}%

%%%%%%%%%%%%%%%%%%%%%%%%%%%%%%%%%%%%%%%%%%%

% time: Friday 10:10
% URL: https://pretalx.com/fossgis2023/talk/fossgis2025-58056-text-und-data-mining-in-der-openstreetmap-datenbank-aus-rechtlicher-sicht/

%
\newSmallTimeslot{10:10}
\noindent\abstractHSdrei{%
  Falk Zscheile%
}{%
  Text und Data Mining in der OpenStreetMap-Datenbank aus rechtlicher Sicht%
}{%
}{%
  Der Vortrag erläutert den Inhalt der relativ neuen rechtlichen Regelungen zum Text und Data
  Mining. Er geht der Frage nach, wer von dieser Regelung profitiert und welche Konsequenzen sich
  daraus für die OpenStreetMap-Daten als solche ergeben.%
}%



%%%%%%%%%%%%%%%%%%%%%%%%%%%%%%%%%%%%%%%%%%%

% time: Friday 11:10
% URL: https://pretalx.com/fossgis2023/talk/fossgis2025-58013-fehlende-brgersteige/

%
\newTimeslot{11:10}
\noindent\abstractHSeins{%
  Roland Olbricht%
}{%
  Fehlende Bürgersteige%
}{%
}{%
  Die Fähigkeit zum Fußgängerrouting gilt als ein nahezu Alleinstellungsmerkmal von OpenStreetMap.
  Doch tatsächlich haben über 50\% aller Hauptstraßen in Deutschland keine Information zu
  Bürgersteigen irgendeiner Art.
  Zeit zu Mappen! Ich habe in meiner Heimatstadt Wuppertal angefangen und setze die Arbeit in den
  Nachbarstädten fort. Welche Werkzeuge und Tagging-Stile ich versucht und welche Werkzeuge und
  Tagging-Style ich nachher genutzt habe, ist Gegenstand dieses Vortrages.%
}%


%%%%%%%%%%%%%%%%%%%%%%%%%%%%%%%%%%%%%%%%%%%

% time: Friday 11:10
% URL: https://pretalx.com/fossgis2023/talk/fossgis2025-57208-ableitung-von-rasterkarten-aus-vector-tiles-fr-basemap-de/

%

\noindent\abstractHSzwei{%
  Sebastian Ratjens%
}{%
  Ableitung von Rasterkarten aus Vector Tiles für basemap.de%
}{%
}{%
  Für die Produktion der Rasterkarten von basemap.de wird die Open-Source-Software "`VT Raster
  Converter"' entwickelt, mit der Vector-Tiles-Karten in Rasterbilder konvertiert werden können. Der
  Vortrag gibt einen Überblick über die Software und deren Nutzung in Verbindung mit dem MapProxy.
  Es werden die Erfahrungen und Herausforderungen in der Produktion der basemap.de Rasterkarten
  mittels Vektor-Raster-Konvertierung erläutert.%
}%

\pagebreak
%%%%%%%%%%%%%%%%%%%%%%%%%%%%%%%%%%%%%%%%%%%

% time: Friday 11:10
% URL: https://pretalx.com/fossgis2023/talk/fossgis2025-58128-qgis-und-blender-fr-beeindruckende-3d-darstellungen/

%

\noindent\abstractHSdrei{%
  Gordon Schlolaut%
}{%
  QGIS und Blender für beeindruckende 3D-Darstellungen%
}{%
}{%
  Blender ist eine freie Software für professionelle Visual Effects und Animationen. In Kombination
  mit QGIS können diese auch dafür verwendet werden, um Geo-Daten in 3D effektvoll darzustellen und
  zu animieren. Obwohl Blender sehr komplex ist, können sehr schnell und mit wenigen Schritten erste
  hervorragende Ergebnisse erzielt werden. Es wird ein Einstieg in den Workflow gegeben und die
  einzelnen relevanten Komponenten von Blender an mehreren Beispielen erklärt.%
}%


%%%%%%%%%%%%%%%%%%%%%%%%%%%%%%%%%%%%%%%%%%%

% time: Friday 11:10
% URL: https://pretalx.com/fossgis2023/talk/fossgis2025-58066-der-elefant-kann-s-auch-allein-graph-erstellung-aus-osm-in-der-postgis-datenbank-/

%

\noindent\abstractHSvier{%
  Matthias Daues%
}{%
  Der Elefant kann's auch allein: Graph-Erstellung aus OSM in der PostGIS-Datenbank.%
}{%
}{%
  Vom osm-dump zum voll vernetzten Graphen: Mit osmium, osm2\-pgsql und einigen simplen
  Datenbank-Prozeduren gelingt die Umwandlung von rein geographischen Informationen in logische
  Datenstrukturen.
  Im GitHub findest Du alles, was Du zum selber machen brauchst.%
}%

\pagebreak
%%%%%%%%%%%%%%%%%%%%%%%%%%%%%%%%%%%%%%%%%%%

% time: Friday 11:10
% URL: https://pretalx.com/fossgis2023/talk/fossgis2025-57397-indoor-osm/

%

\noindent\abstractAnwBoFeins{%
  Tobias Knerr, Volker Krause%
}{%
  Indoor OSM%
}{%
}{%
  Ein Treffen für alle, die als Mapper oder Entwickler mit Indoor-Karten in OpenStreetMap zu tun
  haben%
}%


%%%%%%%%%%%%%%%%%%%%%%%%%%%%%%%%%%%%%%%%%%%

% time: Friday 11:10
% URL: https://pretalx.com/fossgis2023/talk/fossgis2025-57852-qgis-in-der-ffentlichen-verwaltung/

%

\noindent\abstractAnwBoFzwei{%
  David Arndt%
}{%
  QGIS in der öffentlichen Verwaltung%
}{%
}{%
  QGIS erfreut sich einer immer größeren Beliebtheit in der Öffentlichen Verwaltung. Die Vernetzung
  der Kommunen ist hier ein wichtiger Bestandteil für die erfolgreiche Einführung von QGIS. Die
  Session soll dazu dienen Erfahrungen auszutauschen. Geeignet für Neueinsteiger und Profis.%
}%


%%%%%%%%%%%%%%%%%%%%%%%%%%%%%%%%%%%%%%%%%%%

% time: Friday 11:10
% URL: https://pretalx.com/fossgis2023/talk/fossgis2025-65688-ask-me-anything-qgis-/

%

\noindent\abstractAnwBoFdrei{%
  Marco Bernasocchi, Matthias Kuhn%
}{%
  Ask me anything QGIS!%
}{%
}{%
  QGIS-Chairman Marco Bernasocchi und Kernentwickler Matthias Kuhn stehen während einer Stunde für
  alle QGIS-relevanten Fragen zur Verfügung.%
}%


%%%%%%%%%%%%%%%%%%%%%%%%%%%%%%%%%%%%%%%%%%%

% time: Friday 11:45
% URL: https://pretalx.com/fossgis2023/talk/fossgis2025-58094-radinfra-de-gute-daten-und-kampagnen-zur-radinfrastruktur-fr-ganz-deutschland/

%
\newTimeslot{11:45}
\enlargethispage{2.0\baselineskip}
\noindent\abstractHSeins{%
  Tobias Jordans%
}{%
  radinfra.de~-- gute Daten und Kampagnen zur Radinfrastruktur für ganz Deutschland%
}{%
}{%
  OpenStreetMap bietet die einzigen flächendeckend verfügbaren und standardisierten
  Radinfrastrukturdaten in Deutschland. Um diese Daten für die Öffentlichkeit sowie für die planende
  Verwaltung zugänglich zu machen und systematisch zu verbessern, wurde radinfra.de entwickelt.%
}%


%%%%%%%%%%%%%%%%%%%%%%%%%%%%%%%%%%%%%%%%%%%

% time: Friday 11:45
% URL: https://pretalx.com/fossgis2023/talk/fossgis2025-58114-pmtiles-das-cloud-native-format-fr-kacheln/

%

\noindent\abstractHSzwei{%
  Pirmin Kalberer%
}{%
  PMTiles~-- das cloud-native Format für Kacheln%
}{%
}{%
  PMTiles ist ein neues Format zur effizienten Speicherung
  und einfachen Publikation von Vektor- und Rasterkacheln.%
}%


%%%%%%%%%%%%%%%%%%%%%%%%%%%%%%%%%%%%%%%%%%%

% time: Friday 11:45
% URL: https://pretalx.com/fossgis2023/talk/fossgis2025-58287-airborne-laserscanning-in-deutschland-verfgbarkeiten-herausforderungen-potenziale/

%

\noindent\abstractHSdrei{%
  Jens Wiesehahn%
}{%
  Airborne-Laserscanning in Deutschland: Verfügbarkeiten, Herausforderungen,\linebreak Potenziale%
}{%
}{%
  Airborne-Laserscanning (ALS)-Daten werden in Deutschland großflächig durch die Bundesländer
  erhoben. Trotz einer Entwicklung hin zu mehr Open Access bestehen weiterhin Hürden aufgrund
  uneinheitlicher Datenqualität, Verfügbarkeit und Zugangsregelungen. Der Vortrag gibt einen
  Überblick über die Verfügbarkeit von ALS-Daten in Deutschland, zeigt bestehende Probleme bei der
  Datennutzung auf und diskutiert Strategien für eine verbesserte Bereitstellung.%
}%


%%%%%%%%%%%%%%%%%%%%%%%%%%%%%%%%%%%%%%%%%%%

% time: Friday 11:45
% URL: https://pretalx.com/fossgis2023/talk/fossgis2025-58140-opensource-sicher-entwickeln-und-betreiben-prozesse-anforderung-und-tools-im-fokus/

%

\noindent\abstractHSvier{%
  Florian Micklich%
}{%
  OpenSource sicher entwickeln und\linebreak betreiben~-- Prozesse, Anforderung und Tools im Fokus%
}{%
}{%
  Open Source ist mehr als nur Software~-- sicher wird sie jedoch erst durch gezielte Entwicklungs-
  und Betriebsprozesse. Der Vortrag zeigt, wie Open Source-Projekte sicher gestaltet und betrieben
  werden können. Mit "`Shifting Left"' und Automatisierung werden Sicherheitslücken frühzeitig
  erkannt, während Standards wie BSI-Grundschutz und der Cyber Resilience Act (CRA) wichtige
  Rahmenbedingungen setzen. Praxisnahe Einblicke in SAST, DAST und Supply-Chain-Security runden das
  Thema ab.%
}%


%%%%%%%%%%%%%%%%%%%%%%%%%%%%%%%%%%%%%%%%%%%

% time: Friday 12:20
% URL: https://pretalx.com/fossgis2023/talk/fossgis2025-58143-ptna-qualittssicherung-fr-pnv-linien-in-openstreetmap/

%
\newSmallTimeslot{12:20}
\noindent\abstractHSeins{%
  Toni Erdmann%
}{%
  PTNA: Qualitätssicherung für ÖPNV-Linien in OpenStreetMap%
}{%
}{%
  ÖPNV Informationen werden in OSM mittels Route-Master-/Route-Relationen nach dem sogenannten
  PTv2-Schema modelliert.
  PTNA~-- Public Transport Network Analysis erlaubt eine Soll-Ist-Analyse  (OSM vs Realität) mit
  Hilfe einer CSV-Liste der existierenden Linie von Verkehrsverbünden.  Zusätzlich werden
  umfangreichen Qualitätsanalysen angewandt.
  Die Fortschritte seit der FOSSGIS 2020 werden vorgestellt.%
}%


%%%%%%%%%%%%%%%%%%%%%%%%%%%%%%%%%%%%%%%%%%%

% time: Friday 12:20
% URL: https://pretalx.com/fossgis2023/talk/fossgis2025-58090-routingplus-goes-masterportal-unser-weg-zu-einer-modernen-benutzeroberflche/

%
\newLightningTimeslot{12:20}
\noindent\abstractHSzwei{%
  Benjamin Würzler%
}{%
  RoutingPlus goes Masterportal: Unser Weg zu einer modernen Benutzeroberfläche%
}{%
}{%
  Steigt mit ein in unseren Lightning Talk und hört von unseren Erkenntnissen aus dem agilen
  Entwicklungsprozess einer neuen graphischen Benutzeroberfläche für den Routingdienst RoutingPlus
  des Bundesamtes für Kartographie und Geodäsie (BKG). Dabei erfahrt ihr die Gründe, warum wir uns
  für die Open Source Lösung Masterportal entschieden haben, und welche Vorteile sich dadurch nicht
  nur für uns, sondern auch die Open Source Community ergeben.%
}%



%%%%%%%%%%%%%%%%%%%%%%%%%%%%%%%%%%%%%%%%%%%

% time: Friday 12:25
% URL: https://pretalx.com/fossgis2023/talk/fossgis2025-58243-fiboa-standardisierte-feldgrenzen/

%
\newLightningTimeslot{12:25}
\noindent\abstractHSzwei{%
  Matthias Mohr%
}{%
  fiboa~-- Standardisierte Feldgrenzen%
}{%
}{%
  Das Projekt Field Boundaries for Agriculture (fiboa) zielt darauf ab, Feldgrenzen / Feldblöcke in
  einem einheitlichen Format auf globaler Ebene offen zugänglich zu machen. Dies wird ermöglicht
  durch eine offene Spezifikation für Feldgrenzen. Dazu sind quelloffene Programme und frei
  evrfügbare Daten verfügbar, mit denen man schnell und einfach loslegen kann. Diese Präsentation
  gibt eine kurze EInführung in das Projekt, mit der Hoffnung auf einen regen Austausch im
  Anschluss.%
}%

\newpage
\enlargethispage{2.0\baselineskip}
%%%%%%%%%%%%%%%%%%%%%%%%%%%%%%%%%%%%%%%%%%%

% time: Friday 12:30
% URL: https://pretalx.com/fossgis2023/talk/fossgis2025-58284-absicherung-von-diensten-mit-keycloak/

%
\newLightningTimeslot{12:30}
\noindent\abstractHSzwei{%
  Astrid Emde%
}{%
  Absicherung von Diensten mit Keycloak%
}{%
}{%
  Nicht jeder OWS Dienst soll offen verfügbar sein. Nicht jeder Serversoftware über die Dienste
  konfiguriert werden können, stellt eine Möglichkeit der Absicherung zur Verfügung. Wie kann eine
  einfache und einheitliche Lösung für den geschützten Zugriff aussehen?
  Hier kommt Keycloak ins Spiel.
  In nur 5 Minuten wird das Projekt vorgestellt und versucht, Lust auf den Einsatz von Keycloak zu
  machen.%
}%


%%%%%%%%%%%%%%%%%%%%%%%%%%%%%%%%%%%%%%%%%%%

% time: Friday 12:35
% URL: https://pretalx.com/fossgis2023/talk/fossgis2025-57633-datenbankschema-mit-mermaid-visualisieren/

%
\newLightningTimeslot{12:35}
\noindent\abstractHSzwei{%
  Jakob Miksch%
}{%
  Datenbankschema mit Mermaid visualisieren%
}{%
}{%
  Vorstellung von mermerd und Mermaid zu visuellen Darstellung von  (Geo)-Datenbankstrukturen.
  Zusätzlich wird gezeigt wie man diesen Prozess automatisiert.%
}%

%%%%%%%%%%%%%%%%%%%%%%%%%%%%%%%%%%%%%%%%%%%

% time: Friday 12:20
% URL: https://pretalx.com/fossgis2023/talk/fossgis2025-56997-rmische-grabdenkmler-im-3d-webviewer/

%
\newSmallTimeslot{12:20}
\noindent\abstractHSdrei{%
  Johannes Frank, Homayoon Afsharpoor%
}{%
  Römische Grabdenkmäler im 3D-Webviewer%
}{%
}{%
  Wie kann aus hochgenauen Geodaten wie 3D-Punktwolken und 3D-Objekten ein Museumsviewer entwickelt
  werden, der auch von fachfremden Personen genutzt werden kann? Hierfür wurden die
  Open-Source-Webframeworks Potree und 3D-HOP erweitert, um einen immersiven und intuitiven
  Museumsrundgang zu ermöglichen. Die Ergebnisse des digitalen Museums werden in einem Kurzvortrag
  präsentiert.%
}%


%%%%%%%%%%%%%%%%%%%%%%%%%%%%%%%%%%%%%%%%%%%

% time: Friday 14:15
% URL: https://pretalx.com/fossgis2023/talk/fossgis2025-57896-koordinatenreferenzsysteme-fr-d-a-ch/

%
\newTimeslot{14:15}
\noindent\abstractHSeins{%
  Javier Jimenez Shaw%
}{%
  Koordinatenreferenzsysteme für D-A-CH%
}{%
}{%
  Die Eigenschaften und Unterschiede der alten und neuen Koordinatenreferenzsysteme aus der
  D-A-CH-Region werden mithilfe von spatialreference.org erklärt.%
}%


%%%%%%%%%%%%%%%%%%%%%%%%%%%%%%%%%%%%%%%%%%%

% time: Friday 14:45
% URL: https://pretalx.com/fossgis2023/talk/fossgis2025-58091-lebewohl-web-merkator/

%
\newSmallTimeslot{14:45}
\noindent\abstractHSeins{%
  Pirmin Kalberer%
}{%
  Lebewohl Web Merkator%
}{%
}{%
  Die meisten heutigen Webkarten verwenden die
  [Web-Merkator-Projektion](https://de.wikipedia.org/wiki/Merkator-Projektion).
  Das Hauptproblem der Web-Merkator-Projektion ist die starke Verzerrung weit vom Äquator entfernter
  Flächen.%
}%


%%%%%%%%%%%%%%%%%%%%%%%%%%%%%%%%%%%%%%%%%%%

% time: Friday 15:15
% URL: https://pretalx.com/fossgis2023/talk/fossgis2025-58136-ein-blick-in-die-koordinierungsstelle-des-fossgis-e-v-/

%
\newSmallTimeslot{15:15}
\noindent\abstractHSeins{%
  Katja Haferkorn, Jochen Topf%
}{%
  Ein Blick in die Koordinierungsstelle des FOSSGIS e.V.%
}{%
}{%
  Seit einigen Jahren gibt es im FOSSGIS e.V. die Koordinierungsstelle. Dort werden viele Dinge
  erledigt, die den Verein und die FOSSGIS-Konferenz betreffen. In diesem Vortrag unterhalten sich
  Katja und Jochen über ihre Arbeit in der Koordinierungsstelle des FOSSGIS e.V..%
}%


%%%%%%%%%%%%%%%%%%%%%%%%%%%%%%%%%%%%%%%%%%%

% time: Friday 15:40
% URL: https://pretalx.com/fossgis2023/talk/fossgis2025-59154-abschlussveranstaltung/

%
\newTimeslot{15:40}
\noindent\abstractHSeins{%
  %
}{%
  Abschlussveranstaltung%
}{%
}{%
  Drei spannende Konferenztage gehen zu Ende. Ein gemeinsamer Abschluss soll erfolgen mit Rückblick
  auf die Konferenz und das Erlebte. Natürlich auch mit einem Ausblick auf kommende Veranstaltungen
  und die Konfernz im Jahr 2026.%
}%


%%%%%%%%%%%%%%%%%%%%%%%%%%%%%%%%%%%%%%%%%%%

% time: Friday 16:10
% URL: https://pretalx.com/fossgis2023/talk/fossgis2025-59155-sektempfang-im-foyer/

%
\newSmallTimeslot{16:10}
\noindent\abstractHSeins{%
  %
}{%
  Sektempfang im Foyer%
}{%
}{%
  Der FOSSGIS e.V. lädt alle Mitglieder des FOSSGIS-Vereins, Freunde und Interessierte herzlich zum
  Sektempfang zum Ausklang der FOSSGIS 2025 am FOSSGIS-Vereins-Stand ein.%
}%


%%%%%%%%%%%%%%%%%%%%%%%%%%%%%%%%%%%%%%%%%%%
