
% time: Wednesday 10:00
% URL: https://pretalx.com/fossgis2023/talk/fossgis2025-59156-erffnung/

%
\newTimeslot{10:00}
\noindent\abstractHSeins{%
  %
}{%
  Eröffnung%
}{%
}{%
  Feierliche Eröffnung der Konferenz durch Vertreter des FOSSGIS~e.V. mit Hinweisen zum
  Ablauf und der Organisation.%
}%


%%%%%%%%%%%%%%%%%%%%%%%%%%%%%%%%%%%%%%%%%%%

% time: Wednesday 10:10
% URL: https://pretalx.com/fossgis2023/talk/fossgis2025-65632-keynote-und-begrung/

%
\newSmallTimeslot{10:10}
\noindent\abstractHSeins{%
  Josef Hovenjuergen%
}{%
  Keynote und Begrüßung%
}{%
}{%
  Begrüßungsworte und Keynotebeitrag des parlamentarischen Staatssekretärs im Ministerium für Heimat, Kommunales, Bau und Digitalisierung
  des Landes Nordrhein-Westfalen, Josef Hovenjuergen.%
}%


%%%%%%%%%%%%%%%%%%%%%%%%%%%%%%%%%%%%%%%%%%%

% time: Wednesday 10:30
% URL: https://pretalx.com/fossgis2023/talk/fossgis2025-57803-25-jahre-fossgis-e-v-eine-zeitreise-durch-das-vereinsleben/

%
\newSmallTimeslot{10:30}
\noindent\abstractHSeins{%
  Katja Haferkorn, Maik S., Christopher Lorenz%
}{%
  25. Jahre FOSSGIS e.V.~-- eine Zeitreise durch das Vereinsleben%
}{%
}{%
  Die AG-25 hat sich überlegt einen Vortragsblock für den Eröffnungsblock zu organisieren, in dem
  einige Mitglieder Erinnerungen, Erfahrungen oder Highlights aus der Vereinsarbeit teilen.
  Angelehnt ans Lightningtalk-Format wird ein Feuerwerk aus 25 Jahren FOSSGIS~e.V. gezündet,
  untermalt mit Geschichten, Bildern und Fotos.%
}%


%%%%%%%%%%%%%%%%%%%%%%%%%%%%%%%%%%%%%%%%%%%

% time: Wednesday 11:45
% URL: https://pretalx.com/fossgis2023/talk/fossgis2025-58249-eine-reise-durch-die-geoportale-deutschlands/

%
\newTimeslot{11:45}
\noindent\abstractHSeins{%
  Matthias Mohr%
}{%
  Eine Reise durch die Geoportale Deutschlands%
}{%
}{%
  Im letzten Jahr musste ich Datensätze für Feldgrenzen in Deutschland finden. Was erst einmal
  einfach klingt endete mit einer Odyssee durch die verschiedensten Geoportale der Länder. Ich nehme
  euch auf dieser Reise mit und beschreibe, was mir bei der Reise aufgefallen ist und wie man diese
  Erfahrung in der Praxis verbessern könnte.%
}%


%%%%%%%%%%%%%%%%%%%%%%%%%%%%%%%%%%%%%%%%%%%

% time: Wednesday 11:45
% URL: https://pretalx.com/fossgis2023/talk/fossgis2025-58113-spatial-io-cloud-basierte-open-source-lsung-zur-verwaltung-rumlicher-daten/

%

\noindent\abstractHSzwei{%
  Rebekka Lange%
}{%
  spatial.IO~-- Cloud-basierte Open-Source-Lösung zur Verwaltung räumlicher Daten%
}{%
}{%
  Mit zunehmenden Datenmengen im Bereich der Umweltsystemforschung steigt die Notwendigkeit für
  geeignete Dateninfrastrukturen zur automatisierten Verwaltung und Repräsentation der Daten. In
  diesem Vortrag stellen wir die Open-Source-Lösung
  spatial.IO des Helmholtz-Zentrums für
  Umweltforschung (UFZ) vor, welche unabhängig von bestehenden IT-Infrastrukturen überall betrieben
  werden kann.%
}%

\pagebreak
%%%%%%%%%%%%%%%%%%%%%%%%%%%%%%%%%%%%%%%%%%%

% time: Wednesday 11:45
% URL: https://pretalx.com/fossgis2023/talk/fossgis2025-58166-geopandas-als-tool-zur-basiskartenaktualisierung/

%

\noindent\abstractHSdrei{%
  Markus Gruber, Markus Albrecht%
}{%
  GeoPandas~-- als Tool zur Basiskartenaktualisierung%
}{%
}{%
  Die Grundlagenkarte der Stadtwerke München ist eine Sammlung von Daten aus verschiedenen Quellen
  wie amtlichen Daten und eigenen Erhebungen. Sie umfasst ca. 14.000 km² und wird mit PostGIS
  verwaltet. 2024 wurden die ETL-Prozesse auf eine Open Source Lösung basierend auf Python und
  GeoPandas migriert, um die Flexibilität und Performance zu erhöhen. Die Transformationen werden
  auf einem Kubernetes-Cluster ausgeführt, wodurch der Import von mehreren Tagen auf 6 Stunden
  reduziert werden konnte.%
}%


%%%%%%%%%%%%%%%%%%%%%%%%%%%%%%%%%%%%%%%%%%%

% time: Wednesday 11:45
% URL: https://pretalx.com/fossgis2023/talk/fossgis2025-57049-keine-angst-vor-der-geoinfodok-7-3a-datenverarbeitung-mit-postnas/

%

\noindent\abstractHSvier{%
  Oliver Schmidt%
}{%
  Keine Angst vor der GeoInfoDok 7~-- 3A-Datenverarbeitung mit PostNAS%
}{%
}{%
  Für viele Andwendungen müssen 3A-Daten im Format der GeoInfoDok 7.1.2 verarbeitet werden. Vor dem
  Hintergrund der Open\-Data-Bereitstellung des LVermGeo Rheinland-Pfalz steigt nun auch das Interesse
  an originären 3A-Daten und an AdV-konformen WFS und WMS.
  Import- und Verarbeitungsschritte, mit denen 3A-Daten in eine PostGIS-Datenbank überführt werden,
  sind Teil dieses Vortrages. Ebenso werden eigene Skripte und hieraus erzeugte Geowebdienste
  vorgestellt.%
}%


%%%%%%%%%%%%%%%%%%%%%%%%%%%%%%%%%%%%%%%%%%%

% time: Wednesday 12:20
% URL: https://pretalx.com/fossgis2023/talk/fossgis2025-58280-gis-schulprojekte-in-zusammenarbeit-mit-kommunalen-gebietskrperschaften/

%
\newTimeslot{12:20}
\noindent\abstractHSeins{%
  Dietmar Holzner%
}{%
  GIS-Schulprojekte in Zusammenarbeit mit kommunalen Gebietskörperschaften%
}{%
}{%
  Zusammenarbeit zwischen einem technischen Gymnasium und den umliegenden kommunalen Körperschaften
  zur georeferenzierten Erfassung von Infrastrukturen. Im Rahmen eines einwöchigen
  Vermessungspraktikums vermessen Schüler/innen die Objekte und erfassen sie in einer
  georeferenzierten Datenbank. Grafische Darstellung und Datenhandling erfolgt mittels QGIS durch
  die Verbindung mit der Datenbank.%
}%


%%%%%%%%%%%%%%%%%%%%%%%%%%%%%%%%%%%%%%%%%%%

% time: Wednesday 12:20
% URL: https://pretalx.com/fossgis2023/talk/fossgis2025-57279-eoapi-eine-skalierbare-geodateninfrastruktur/

%

\noindent\abstractHSzwei{%
  Felix Delattre%
}{%
  eoAPI~-- eine skalierbare Geodaten\-infrastruktur%
}{%
}{%
  In diesem Vortrag geht es um eoAPI, ein Open-Source-Baukasten zur schnellen Bereitstellung einer
  skalierbaren Geodaten\-infra\-struktur auf Grundlage der STAC-Spezifikation. eoAPI ermöglicht die
  einfache Auffindbarkeit und breite Nutzung von Erdbeobachtungsdaten. Es werden ihre
  leistungsstarken Funktionen sowie die Komponenten TiTiler, Tipg, pgSTAC und stac-fastapi
  erläutert.%
}%

\pagebreak
%%%%%%%%%%%%%%%%%%%%%%%%%%%%%%%%%%%%%%%%%%%

% time: Wednesday 12:20
% URL: https://pretalx.com/fossgis2023/talk/fossgis2025-58285-aviary-ein-generisches-python-framework-zur-ki-inferenz-fr-fernerkundungsdaten/

%

\noindent\abstractHSdrei{%
  Marius Maryniak%
}{%
  aviary~-- ein generisches Python-Framework zur KI-Inferenz für Fernerkundungsdaten%
}{%
}{%
  aviary ist ein generisches Python-Framework, das die Inferenz von KI-Modellen für
  Fernerkundungsdaten vereinfacht.
  Es bietet verschiedene Pipelines mit austauschbaren, erweiterbaren Komponenten. Neben der Nutzung
  als Python-Package können vorgefertigte Pipelines über die Kommandozeile verwendet werden.
  Künftig sollen vortrainierte Modelle für diverse Anwendungsfälle sowie
  speziell auf Fernerkundungsdaten trainierte Foundation-Modelle bereitgestellt werden.%
}%


%%%%%%%%%%%%%%%%%%%%%%%%%%%%%%%%%%%%%%%%%%%

% time: Wednesday 12:20
% URL: https://pretalx.com/fossgis2023/talk/fossgis2025-57638-gdi-per-knopfdruck-automatisierung-mit-devops-und-infrastruktur-als-code/

%

\noindent\abstractHSvier{%
  Jakob Miksch%
}{%
  GDI per Knopfdruck: Automatisierung mit DevOps und Infrastruktur als Code%
}{%
}{%
  Der Vortrag zeigt auf wie eine Geodateninfrastruktur(GDI) automatisiert über Code eingerichtet
  werden kann und geht dabei auf die von uns genutzten Software-Bausteine und deren Alternativen
  ein.%
}%


%%%%%%%%%%%%%%%%%%%%%%%%%%%%%%%%%%%%%%%%%%%

% time: Wednesday 14:15
% URL: https://pretalx.com/fossgis2023/talk/fossgis2025-58266-basemap-de-als-open-data-neue-stile-und-anwendungsbeispiele/

%
\newTimeslot{14:15}
\noindent\abstractHSeins{%
  Arnulf B. Bichler (aka Christl)%
}{%
  basemap.de als Open Data~-- Neue Stile und Anwendungsbeispiele%
}{%
}{%
  Die Grundlage der basemap.de sind amtliche Daten der Länder, die seit Mitte 2024 als Open Data
  bereitgestellt werden. In dem Vortrag werden neue Möglichkeiten gezeigt, die sich daraus ergeben
  und neue Stile vorgestellt. In Anwendungsbeispielen werden die Potenziale für Behörden und eigene
  Entwicklungen gezeigt.%
}%


%%%%%%%%%%%%%%%%%%%%%%%%%%%%%%%%%%%%%%%%%%%

% time: Wednesday 14:15
% URL: https://pretalx.com/fossgis2023/talk/fossgis2025-57703-automatischer-import-und-verffentlichung-von-betriebsmittelgeometrien-mittels-pyqgis/

%

\noindent\abstractHSzwei{%
  Philipp Opitz%
}{%
  Automatischer Import und Veröffent\-lichung von Betriebsmittelgeometrien mittels PyQGIS%
}{%
}{%
  Ziel von Versorgungsunternehmen ist ein minimaler Zeitraum zwischen Beendingung einer Baumaßnahme
  und Integration der Geodaten in die Leitungsauskunft. Eine Möglichkeit bietet die automatische
  Integration der Geometrien als Vorabauskunft. In der Umsetzung bei SachsenEnergie wurde für die
  Datenveröffentlichung ein ETL-Prozess auf QGIS/PyQGIS-Basis entwickelt, welcher differenziell und
  performant nächtlich die DXF-Daten in eine Datenbank überführt. Die Publikation der Daten erfolgt
  mittels WMS.%
}%


%%%%%%%%%%%%%%%%%%%%%%%%%%%%%%%%%%%%%%%%%%%

% time: Wednesday 14:15
% URL: https://pretalx.com/fossgis2023/talk/fossgis2025-58092-superset-business-intelligence-meets-cartography/

%

\noindent\abstractHSdrei{%
  Jan Suleiman%
}{%
  Superset~-- Business Intelligence meets Cartography%
}{%
}{%
  Superset ist eine der meistgenutzten Open-Source Tools im Bereich der Business Intelligence (BI).
  Wir haben Superset um die Erstellung thematischer Karten erweitert, wodurch auch raum-zeitliche
  Aspekte besser analysiert werden können. In diesem Vortrag zeigen wir, wie sowohl Kartodiagramme,
  Choroplethenkarten, sowie Proportional Symbol Maps in Superset erstellt und für raum-zeitliche
  Analysen verwendet werden können.%
}%

\newLightningTimeslot{14:15}
%%%%%%%%%%%%%%%%%%%%%%%%%%%%%%%%%%%%%%%%%%%

% time: Wednesday 14:15
% URL: https://pretalx.com/fossgis2023/talk/fossgis2025-57772-foss-basierte-schnittstelle-zum-management-von-heritage-bim-modellen/

%

\noindent\abstractHSvier{%
  Güren Tan Dinga, Laura Fernandez Resta%
}{%
  FOSS-basierte Schnittstelle zum Management von Heritage BIM Modellen%
}{%
}{%
  Die Integration von Heritage BIM in die Erhaltung historischer Gebäude ist insbesondere bei der
  zukünftigen Bewirtschaftung dieser von hoher Relevanz. Am Beispiel des UNESCO-Welterbes
  Speicherstadt Hamburg wurde eine benutzerorientierte Schnittstelle entwickelt, welche es
  Nutzer:innen ohne BIM-Kenntnisse ermöglicht, Zugang zu strukturierten Gebäudeinformationen zu
  gewähren. Dafür wurden offene Standards wie IFC und FOSS genutzt, wodurch die Interoperabilität
  und Kollaboration gefördert wurden.%
}%


%%%%%%%%%%%%%%%%%%%%%%%%%%%%%%%%%%%%%%%%%%%

% time: Wednesday 14:20
% URL: https://pretalx.com/fossgis2023/talk/fossgis2025-57804-amtliche-orthofotos-zentral-verfgbar-eine-open-source-lsung-fr-deutschland/

%

\newLightningTimeslot{14:20}
\noindent\abstractHSvier{%
  Moritz Lucas%
}{%
  Amtliche Orthofotos zentral verfügbar: Eine Open-Source-Lösung für Deutschland%
}{%
}{%
  In Deutschland werden amtliche Orthophotos mittlerweile in allen Bundesländern frei zugänglich
  bereitgestellt, geregelt durch die INSPIRE-Richtlinie. Doch die föderale Struktur führt zu 16
  unterschiedlichen Diensten, die den Zugriff auf diese hochwertigen Bilddaten erschweren. Die
  vorgestellte Software bietet eine zentrale Lösung, die den deutschlandweiten Download vereinfacht.
  Ziel ist es, mehr Aufmerksamkeit für Orthophotos zu schaffen und deren Nutzung sowie
  Weiterentwicklung zu fördern.%
}%


%%%%%%%%%%%%%%%%%%%%%%%%%%%%%%%%%%%%%%%%%%%

% time: Wednesday 14:25
% URL: https://pretalx.com/fossgis2023/talk/fossgis2025-59167-ergebnisprsentation-des-code-sprints/

%

\newLightningTimeslot{14:25}
\noindent\abstractHSvier{%
  %
}{%
  Ergebnispräsentation des Code-Sprints%
}{%
}{%
  In diesem doppelten Lightning-Talk-Block werden im Schnelldurchlauf die Ergebnisse des bereits
  seit Anfang der Woche stattfindenden Code-Sprints präsentiert.%
}%

%\vspace{0.5cm}
\sponsorBoxA{402_mundialis.png}{0.37\textwidth}{5}{%
\textbf{Bronzesponsor und Aussteller}\\
\noindent\small {\bfseries mundialis GmbH \& Co. KG} mundialis ist spezialisiert auf die Auswertung und Verarbeitung von Fernerkundungs- und Geodaten mit dem Schwerpunkt Cloud-basierte Geoprozessierung. Wir setzen Freie \& Open Source Geoinformationssysteme (GRASS GIS, actinia, QGIS, u.a.) ein, mit denen wir maßgeschneiderte Lösungen für den Kunden entwickeln.
\normalsize
}%


\newTimeslot{14:15}
\enlargethispage{1.8\baselineskip}
%%%%%%%%%%%%%%%%%%%%%%%%%%%%%%%%%%%%%%%%%%%

% time: Wednesday 14:15
% URL: https://pretalx.com/fossgis2023/talk/fossgis2025-63518-studierende-stellen-ihre-arbeit-vor/

%

\noindent\abstractAnwBoFeins{%
  %
}{%
  Studierende stellen ihre Arbeit vor%
}{%
}{%
  Studierende stellen Ihre Arbeit vor (Masterarbeit, Bachelorarbeit, aktuell in Arbeit,
  Seminararbeit, Praktikumsaufgaben, Abschlussarbeiten(Ausbildung) .%
}%


%%%%%%%%%%%%%%%%%%%%%%%%%%%%%%%%%%%%%%%%%%%

% time: Wednesday 14:15
% URL: https://pretalx.com/fossgis2023/talk/fossgis2025-57854-bof-geonode-de/

%

\noindent\abstractAnwBoFzwei{%
  Henning Bredel, Matthes Rieke%
}{%
  BoF GeoNode-DE%
}{%
}{%
  Die Gruppe GeoNode-DE möchte gemeinsam Fragen zur Weiterentwicklung diskutieren und das vorhandene
  Netzwerk zu stärken.%
}%


%%%%%%%%%%%%%%%%%%%%%%%%%%%%%%%%%%%%%%%%%%%

% time: Wednesday 14:15
% URL: https://pretalx.com/fossgis2023/talk/fossgis2025-58049-lizmap-webclient/

%

\noindent\abstractAnwBoFdrei{%
  Günter Wagner%
}{%
  Lizmap Webclient%
}{%
}{%
  Diese Fragestunde, kombiniert mit Demo-Beispielen, ermöglicht Interessierten einen Einblick in den
  Webclient Lizmap in Kombination mit dem QGIS-Server. Neben den Fragen der Teilnehmer werden
  spezielle Funktionen und Neuerungen vorgestellt. Dabei ist die Expert:innenfragestunde sowohl für
  neu interessierte als auch für Anwender vom Lizmap Webclient interessant.%
}%
\sponsorBoxA{403_grit.png}{0.16\textwidth}{2}{%
\textbf{Bronzesponsor}\\
\noindent\small Die {\bfseries grit GmbH} ist ein Software- und Beratungsunternehmen im Bereich der Geo-IT mit Standorten in Werne und Olpe. Die Kernkompetenzen sind Geoinformationssysteme und Geodateninfrastrukturen, wobei vorzugsweise Open Source-Technologien in containerisierten Umgebungen zum Einsatz kommen.
\normalsize
}%


%%%%%%%%%%%%%%%%%%%%%%%%%%%%%%%%%%%%%%%%%%%

% time: Wednesday 14:50
% URL: https://pretalx.com/fossgis2023/talk/fossgis2025-58131-open-data-in-d-perfekte-idee-halbherzige-umsetzung-ein-erfahrungsbericht-/

%
\newTimeslot{14:50}
\noindent\abstractHSeins{%
  Mike Elstermann%
}{%
  Open Data in D: Perfekte Idee, halbherzige Umsetzung? Ein Erfahrungsbericht.%
}{%
}{%
  Open Data ist gut, aber nur, wenn ...
  Lange haben wir sie gefordert und darum gekämpft und zum Glück gibt es sie jetzt, diese deutschen
  offenen Geobasisdaten, in allen Bundesländern, aber mit unterschiedlicher Ausprägung. Mein
  persönlicher Erfahrungsbericht soll den aktuellen Status und das Verbesserungspotenzial zeigen und
  zur Diskussion zwischen Nutzern und Anbietern anregen .%
}%


%%%%%%%%%%%%%%%%%%%%%%%%%%%%%%%%%%%%%%%%%%%

% time: Wednesday 14:50
% URL: https://pretalx.com/fossgis2023/talk/fossgis2025-58278-qgis-werkzeuge-und-python/

%

\noindent\abstractHSzwei{%
  Isabelle Korsch%
}{%
  QGIS-Werkzeuge und Python%
}{%
}{%
  Mit Python kann der Funktionsumfang von QGIS erweitert werden, aber es muss nicht immer ein
  QGIS-Plugin sein: Häufig lohnt es sich auch mit Python neue Werkzeuge in QGIS zu erstellen. Wie
  funktioniert das?%
}%

\pagebreak
%%%%%%%%%%%%%%%%%%%%%%%%%%%%%%%%%%%%%%%%%%%

% time: Wednesday 14:50
% URL: https://pretalx.com/fossgis2023/talk/fossgis2025-57948-visualisierung-von-historischen-schiffsrouten-mit-unscharfer-datengrundlage/

%

\noindent\abstractHSdrei{%
  Stefan Fuest, Andreas Gollenstede, Jennifer Tadge, Maximilian Herbers, Rieke Marie Kaiser%
}{%
  Visualisierung von historischen Schiffs\-routen mit unscharfer Datengrundlage%
}{%
}{%
  Im Forschungsverbund DiViAS (www.divias.de) werden Quellen wie Logbücher und Journale aus dem
  18./19. Jh. ausgewertet und analysiert. Im Fokus dieses Vortrags stehen dabei verschiedene
  Möglichkeiten der Visualisierung von Schiffsrouten mit unscharfer Datengrundlage, welche mit QGIS
  und Mapbox erstellt werden. Grundlage für die kartographische Darstellung ist eine KI-gestützte
  Extraktion von unscharfen Orts- und Zeitangaben aus diesen Quellen und deren Modellierung in einer
  PostgreSQL-Datenbank.%
}%


%%%%%%%%%%%%%%%%%%%%%%%%%%%%%%%%%%%%%%%%%%%

% time: Wednesday 14:50
% URL: https://pretalx.com/fossgis2023/talk/fossgis2025-58293-wie-maplibre-und-vektorkarten-die-welt-bernehmen/

%

\noindent\abstractHSvier{%
  Bart Louwers, Just van den Broecke%
}{%
  Wie MapLibre und Vektorkarten die Welt übernehmen%
}{%
}{%
  Vektorbasierte Karten sind die Zukunft! Oder vielleicht sogar schon die Gegenwart? In diesem
  Vortrag werden beide Perspektiven beleuchtet! Bart spricht aus der Perspektive der Entwicklung von
  MapLibre, und gibt einen Einblick in den neuesten Stand. Just erzählt von seinen Erfahrungen als
  Benutzer der MapLibre-Stack um vektorbasierte Karten zu gestalten.%
}%


%%%%%%%%%%%%%%%%%%%%%%%%%%%%%%%%%%%%%%%%%%%

% time: Wednesday 15:25
% URL: https://pretalx.com/fossgis2023/talk/fossgis2025-58227-open-data-des-bkg-ii-/

%
\newTimeslot{15:25}
\noindent\abstractHSeins{%
  Joachim Eisenberg%
}{%
  Open Data des BKG (II)%
}{%
}{%
  Das Bundesamt für Kartographie und Geodäsie (BKG) hat ein breites Open Data-Angebot, das auf der
  letzten FOSSGIS-Konferenz in Hamburg vorgestellt wurde. Was hat sich mit der Umsetzung der
  HVD-Verordnung an diesem Angebot getan?%
}%


%%%%%%%%%%%%%%%%%%%%%%%%%%%%%%%%%%%%%%%%%%%

% time: Wednesday 15:25
% URL: https://pretalx.com/fossgis2023/talk/fossgis2025-58127-schnupperkurs-das-potential-von-qgis-mit-der-python-konsole-freischalten/

%

\noindent\abstractHSzwei{%
  Gordon Schlolaut%
}{%
  Schnupperkurs: Das Potential von QGIS mit der Python-Konsole freischalten%
}{%
}{%
  Programmierung mit Python ermöglicht es, sich eigene Werkzeuge in QGIS zu erstellen und so den
  Funktionsumfang von QGIS individuell zu erweitern und den eigenen Bedürfnissen anzupassen. Aber
  wie funktioniert das alles eigentlich und womit fängt man an? Dieser Vortrag gibt einen kurzen und
  verständlichen Überblick auch für alle ohne Programmier-Vorkenntnisse. Wir werden live ein kleines
  Werkzeug schreiben, an dem wir die grundlegenden Konzepte vorstellen und zeigen: es ist gar nicht
  so schwierig.%
}%
\sponsorBoxA{405_GBD.png}{0.35\textwidth}{2}{%
\textbf{Bronzesponsor und Aussteller}\\
\noindent\small Die {\bfseries Geoinformatikbüro Dassau GmbH} aus Düsseldorf bietet seit 2006 Beratung, Konzeption, Schulung, Wartung, Support und Programmierung zum Thema GIS und GDI auf Open Source Basis. Ein Fokus liegt auf der Software QGIS, QGIS Server, QField, QGIS Web Client, GBD WebSuite, PostgreSQL/PostGIS und GRASS GIS.
\normalsize
}%

\pagebreak
%%%%%%%%%%%%%%%%%%%%%%%%%%%%%%%%%%%%%%%%%%%

% time: Wednesday 15:25
% URL: https://pretalx.com/fossgis2023/talk/fossgis2025-58269-automatisierte-verarbeitung-von-daten-der-meeresbodenkartografie-mit-qgis/

%

\noindent\abstractHSdrei{%
  Helge Staedtler%
}{%
  Automatisierte Verarbeitung von Daten der Meeresbodenkartografie mit QGIS%
}{%
}{%
  Es wird ein Einblick gegeben wie Rasterdaten mit Hilfe genannter GIS Open Source Werkzeuge \&
  Dienste automatisiert verarbeitet werden können. Im unternehmerischen Kontext beschäftige ich mich
  gemeinsam im Team mit hyperspektralen und RGB Unterwasserdaten, und deren ML-unterstützter
  Verarbeitung, um den Meeresboden zu analysieren \& zu kartografieren. Für die Verarbeitung werden
  Werkzeuge wie z.B. GDAL, QGIS, OSM und \LaTeX\  eingesetzt.%
}%


%%%%%%%%%%%%%%%%%%%%%%%%%%%%%%%%%%%%%%%%%%%

% time: Wednesday 15:25
% URL: https://pretalx.com/fossgis2023/talk/fossgis2025-58089-vektor-tiles-fr-karten-mit-echtzeitdaten/

%

\noindent\abstractHSvier{%
  Pirmin Kalberer%
}{%
  Vektor Tiles für Karten mit Echtzeitdaten%
}{%
}{%
  Vektorkacheln sind eine effiziente Art, um Karten mit grossen Mengen an Echtzeitdaten
  bereitzustellen.%
}%


%%%%%%%%%%%%%%%%%%%%%%%%%%%%%%%%%%%%%%%%%%%

% time: Wednesday 16:30
% URL: https://pretalx.com/fossgis2023/talk/fossgis2025-57856-small-seeds-foss-communities-strken-/

%
\newTimeslot{16:30}
\noindent\abstractHSeins{%
  Paul Robben%
}{%
  Small seeds~-- FOSS Communities stärken!%
}{%
}{%
  Mehr finanzielle Mittel für Freie und Open-Source-Software~-- die unendliche Geschichte. In einer
  von Start-ups, VC Funding und datensammelnden Apps geprägten Tech-Welt ist der Kampf um
  nachhaltige Finanzierungen für ethische Technologieprojekte besonders hart. Nach einigen großen
  Erfolgen für die Förderung von FOSS in den letzten Jahren geht es in diesem Vortrag nun darum, die
  kleinen, weniger sichtbaren zivilgesellschaftlichen Projekte nicht zu vergessen.%
}%


%%%%%%%%%%%%%%%%%%%%%%%%%%%%%%%%%%%%%%%%%%%

% time: Wednesday 16:30
% URL: https://pretalx.com/fossgis2023/talk/fossgis2025-57851-von-proprietr-zu-qgis/

%

\noindent\abstractHSzwei{%
  David Arndt%
}{%
  Von Proprietär zu QGIS%
}{%
}{%
  QGIS erfreut sich einer großen Beliebtheit. Der Regionalverband Ruhr setzt QGIS schon seit einiger
  Zeit als Standard GIS-System ein. In diesem Vortrag sollen die Erfahrungen geteilt werden wie ein
  Umstieg von ArcGIS erfolgen kann. Dabei werden Aspekte wie Kosten, Features, Einbindung in eine
  vorhandene GDI betrachtet.%
}%

\pagebreak
%%%%%%%%%%%%%%%%%%%%%%%%%%%%%%%%%%%%%%%%%%%

% time: Wednesday 16:30
% URL: https://pretalx.com/fossgis2023/talk/fossgis2025-58126-die-leistungsfhigkeit-groer-open-source-sprachmodelle-fr-geoparsing-aufgaben/

%

\noindent\abstractHSdrei{%
  Juiwen Chang (Ariel)%
}{%
  Die Leistungsfähigkeit großer open source Sprachmodelle für Geoparsing-Aufgaben%
}{%
}{%
  Wir präsentieren einen Geoparsing-Workflow, der Name Entity Recognition und Geokodierung
  kombiniert, um Ortsangaben inklusive Hausnummern aus Texten zu extrahieren und in einem WebGIS zu
  visualisieren. Wir haben moderne großer Sprachmodelle (LLM) wie Meta Llama3.1-70b-instruct und
  Mistral-large getestet und dabei herausgefunden, dass ein hybrider Open-Source-Ansatz bis
  zu 70 \% der Standorte korrekt erkennt~-- womit der Ansatz besser ist als Anthropic Claude und
  ChatGPT o1-preview.%
}%


%%%%%%%%%%%%%%%%%%%%%%%%%%%%%%%%%%%%%%%%%%%

% time: Wednesday 16:30
% URL: https://pretalx.com/fossgis2023/talk/fossgis2025-58192-open-alkis-oder-was-passiert-wenn-der-deutsche-fderalismus-auf-eu-recht-trifft/

%

\noindent\abstractHSvier{%
  Stefan Zaunseder%
}{%
  Open ALKIS?~-- Oder was passiert, wenn der deutsche Föderalismus auf EU-Recht trifft%
}{%
}{%
  Wir befinden uns im Jahr 2024. Die Durchführungsverordnung (EU) 2023/138 tritt in Kraft und sorgt
  dafür, dass wesentliche Teile des Liegenschaftskatasters als Open Data zur Verfügung stehen. Das
  gilt auch in Deutschland für ALKIS. Doch hier sind die Bundesländer zuständig, was zu einem
  Sammelsurium von Umsetzungen führt. Und ein von unbeugsamen Bajuwaren bevölkertes Land hört nicht
  auf, den Europäern Widerstand zu leisten. Das Leben ist nicht leicht für Open Data-Enthusiasten...%
}%

\pagebreak
%%%%%%%%%%%%%%%%%%%%%%%%%%%%%%%%%%%%%%%%%%%

% time: Wednesday 16:30
% URL: https://pretalx.com/fossgis2023/talk/fossgis2025-63538-studierende-stellen-ihre-arbeit-vor/

%

\noindent\abstractAnwBoFeins{%
  %
}{%
  Studierende stellen ihre Arbeit vor%
}{%
}{%
  Studierende stellen Ihre Arbeit vor (Masterarbeit, Bachelorarbeit, aktuell in Arbeit,
  Seminararbeit, Praktikumsaufgaben, Abschlussarbeiten(Ausbildung) .%
}%


%%%%%%%%%%%%%%%%%%%%%%%%%%%%%%%%%%%%%%%%%%%

% time: Wednesday 16:30
% URL: https://pretalx.com/fossgis2023/talk/fossgis2025-58071-open-geodata-und-open-source-gis-software-in-den-kultur-und-geisteswissenschaften/

%

\noindent\abstractAnwBoFzwei{%
  Klaus Stein, Carmen M. Enss, Anastasia Bauch%
}{%
  Open Geodata und Open Source GIS Software in den Kultur- und Geisteswissenschaften%
}{%
}{%
  Vernetzungstreffen für alle, die Open Source GIS Software in den Kultur- und Geisteswissenschaften
  einsetzen (wollen), an offenen Geodaten arbeiten, oder an diesen Themen Interesse haben.%
}%


%%%%%%%%%%%%%%%%%%%%%%%%%%%%%%%%%%%%%%%%%%%

% time: Wednesday 17:05
% URL: https://pretalx.com/fossgis2023/talk/fossgis2025-58262-governance-von-open-source-software-im-ffentlichen-sektor-make-buy-or-contribute-/

%
\newTimeslot{17:05}
\noindent\abstractHSeins{%
  Christian Weidner%
}{%
  Governance von Open-Source-Software im öffentlichen Sektor: Make, Buy or\linebreak Contribute?%
}{%
}{%
  Die Governance von OSS im öffentlichen Sektor variiert stark: Während einige Projekte offen und
  kollaborativ sind, bleiben andere hierarchisch strukturiert. Dieser Beitrag untersucht, warum
  Behörden sich für oder gegen die Öffnung ihrer Softwareprojekte entscheiden. Die Untersuchung gibt
  Einblick in ausgewählte OSS-Projekte der öffentlichen Verwaltung und liefert einen
  Erklärungsansatz, warum Faktoren wie Unsicherheit, Innovationsdruck und normative Erwartungen die
  Entscheidung beeinflussen.%
}%


%%%%%%%%%%%%%%%%%%%%%%%%%%%%%%%%%%%%%%%%%%%

% time: Wednesday 17:05
% URL: https://pretalx.com/fossgis2023/talk/fossgis2025-58190-von-proprietr-zu-open-source-umstellung-der-kommunalen-gdi-bei-der-stadt-reutlingen/

%

\noindent\abstractHSzwei{%
  Simon Kondic, Linus Lambrecht%
}{%
  Von proprietär zu Open-Source -Umstellung der kommunalen GDI bei der Stadt Reutlingen%
}{%
}{%
  Die Stadt Reutlingen, eine Großstadt in Baden-Württemberg, stellte ihre kommunale
  Geodateninfrastruktur (GDI) von proprietärer auf Open-Source Software um. Ziel ist eine größere
  digitale Souveränität, Lizenzkostenreduktion und Unabhängigkeit von kommerziellen Anbietern. Im
  Vortrag wird die Vorgehensweise der Systemmigration und der Aufbau der neuen Open-Source GDI
  vorgestellt.%
}%


%%%%%%%%%%%%%%%%%%%%%%%%%%%%%%%%%%%%%%%%%%%

% time: Wednesday 17:05
% URL: https://pretalx.com/fossgis2023/talk/fossgis2025-58109-entwicklung-eines-llm-basierten-assistenten-fr-die-suche-nach-geodaten/

%

\noindent\abstractHSdrei{%
  Simeon Wetzel, Matthes Rieke%
}{%
  Entwicklung eines LLM-basierten Assistenten für die Suche nach Geodaten%
}{%
}{%
  Unser Vortrag stellt ein innovatives Framework zur Verbesserung der Geodatensuche vor. Durch die
  Kombination von Large Language Models, dialogbasierter Nutzerinteraktion und semantischer Suche
  sollen die Limitierungen traditioneller, metadatenbasierter Suchsysteme überwunden werden. Das
  System ermöglicht eine präzisere Erfassung von Nutzeranforderungen und kann sowohl Metadaten als
  auch die eigentlichen Geodatenattribute durchsuchen, was die Qualität der Suchergebnisse deutlich
  verbessert.%
}%


%%%%%%%%%%%%%%%%%%%%%%%%%%%%%%%%%%%%%%%%%%%

% time: Wednesday 17:05
% URL: https://pretalx.com/fossgis2023/talk/fossgis2025-58130-was-wre-wenn-wir-algorithmen-demokratisieren-kollaborative-infrastrukturen-fr-udz/

%

\noindent\abstractHSvier{%
  Rico H Herzog%
}{%
  Was wäre, wenn wir Algorithmen demokratisieren? Kollaborative Infrastrukturen für UDZ%
}{%
}{%
  Das Paper zeigt, wie neue kollaborative Infrastrukturen für Urbane Digitale Zwillinge eine bessere
  Teilhabe und Zugang zur Simulationsentwicklung schaffen können. Am Beispiel von drei
  Open-Source-Tools aus dem "`Connected Urban Twins"-Projekt wird demonstriert, wie Szenarien für
  Stadtentwicklungsprozesse gemeinsam entwickelt, analysiert und zugänglich gemacht werden können.
  Ziel ist eine nachhaltigere, inklusivere Planung, die durch vielfältige Perspektiven blinde
  Flecken in Modellen reduziert.%
}%


%%%%%%%%%%%%%%%%%%%%%%%%%%%%%%%%%%%%%%%%%%%

% time: Wednesday 17:40
% URL: https://pretalx.com/fossgis2023/talk/fossgis2025-58038-25-jahre-fossgis-e-v-was-haben-wir-geschafft-und-wo-wollen-wir-hin/

%
\newTimeslot{17:40}
\noindent\abstractHSeins{%
  Torsten Friebe, David Arndt, Torsten Wiebke%
}{%
  25 Jahre FOSSGIS e.V.~-- was haben wir geschafft und wo wollen wir hin%
}{%
}{%
  Seit Juni 2021 beschäftigt sich die Arbeitsgruppe \glqq Öffentliche Ausschreibungen mit FOSS\grqq\  des
  FOSSGIS~e.V. mit dem Thema der
  Beschaffung und Vergabe von IT-Lösungen auf Basis von FOSS. In dieser Dialogrunde wollen wir mit
  Vertreter:innen des FOSSGIS-Community über die Erfahrungen aus 25 Jahren Einsatz von Open Source
  Software in der öffentlichen Verwaltung diskutieren.%
}%


%%%%%%%%%%%%%%%%%%%%%%%%%%%%%%%%%%%%%%%%%%%

% time: Wednesday 17:40
% URL: https://pretalx.com/fossgis2023/talk/fossgis2025-57890-migration-eines-auskunftssystems-zu-einer-open-source-lsung-mit-qgis/

%

\noindent\abstractHSzwei{%
  Victor Ali Lagoa%
}{%
  Migration eines Auskunftssystems zu einer Open-Source Lösung mit QGIS%
}{%
}{%
  Die Geodatenabteilung der Stadtwerke München nutzte GE Smallworld für ihr Netzinformationssystem,
  was aufgrund steigender Komplexität und Nutzerzahl zu hohen Kosten führte. Vor fünf Jahren begann
  die Migration zu NIS-QGIS, einer kostensparenden, auf QGIS-Desktop und PyQGIS-Plugins basierenden
  Open-Source-Lösung. Der Vortrag liefert einen Überblick über die neue Anwendung, Anwendungsfälle,
  Herausforderungen und Erkenntnisse.%
}%

\pagebreak
%%%%%%%%%%%%%%%%%%%%%%%%%%%%%%%%%%%%%%%%%%%

% time: Wednesday 17:40
% URL: https://pretalx.com/fossgis2023/talk/fossgis2025-58175-knstliche-intelligenz-als-untersttzung-in-geografische-applikationen/

%

\noindent\abstractHSdrei{%
  Marion Baumgartner, Andrea Borghi%
}{%
  Künstliche Intelligenz als Unterstützung in geografische Applikationen%
}{%
}{%
  In dieser Präsentation zeigen wir die Ergebnisse von zwei POCs, welche wir mit dem Large Language
  Models (LLM) Open Source Python Framework "`LangChain” durchgeführt haben. Mittels LLM (wie GPT)
  zeigen wir, wie eine geographische Anwendung durch künstliche Intelligenz benutzerfreundlicher und
  zugänglicher wird. Auch im Hintergrund arbeiten wir mit künstlicher Intelligenz, um die
  Anreicherung und die Indexierung von Metadaten zu verbessern und zu automatisieren.%
}%


%%%%%%%%%%%%%%%%%%%%%%%%%%%%%%%%%%%%%%%%%%%

% time: Wednesday 17:40
% URL: https://pretalx.com/fossgis2023/talk/fossgis2025-57509-metadaten-fr-eine-verantwortungsvolle-und-kritische-geodatenpraxis/

%

\noindent\abstractHSvier{%
  Ester Scheck%
}{%
  Metadaten für eine verantwortungsvolle und kritische Geodatenpraxis%
}{%
}{%
  Die Open Data Bewegung ermöglicht den Zugang zu zahlreichen offenen Datensätzen. Allerdings fehlen
  oft detailliertere Metadaten, die den Kontext der Daten, wie Erhebungsmethoden oder Entscheidungen
  zu Klassifikationen, erklären. Für eine verantwortungsvolle und kritische Geodatenpraxis ist es
  jedoch essentiell, diesen Entstehungskontext einzubeziehen. Ansatzpunkte dafür können Frameworks
  zur Dokumentation und Reflexion des Datenkontextes sowie Metadaten Standards sein.%
}%


%%%%%%%%%%%%%%%%%%%%%%%%%%%%%%%%%%%%%%%%%%%

% time: Wednesday 18:15
% URL: https://pretalx.com/fossgis2023/talk/fossgis2025-57825-projekt-georg-bei-den-swm-mit-freier-software-zur-konzernweiten-geodatenplattform/

%
\newTimeslot{18:15}
\noindent\abstractHSzwei{%
  Nina Röckelein, Benedikt Seidl%
}{%
  Projekt GEOrg bei den SWM: Mit Freier Software zur konzernweiten Geodatenplattform%
}{%
}{%
  Mit freier Software aus dem QGIS-Universum bauen wir bei den Stadtwerken München eine konzernweite
  Geodatenplattform. Wir erzählen von unseren Erfahrungen bei der Integration verschiedener
  Anwendungen und der Etablierung von Open-Source Lösungen im Unternehmen.%
}%


%%%%%%%%%%%%%%%%%%%%%%%%%%%%%%%%%%%%%%%%%%%

% time: Wednesday 18:15
% URL: https://pretalx.com/fossgis2023/talk/fossgis2025-57193-open-data-und-ki-im-einsatz-geodaten-fr-alle-nicht-nur-fr-profis-/

%

\noindent\abstractHSdrei{%
  Klemens Maget, Lisa Stubert%
}{%
  Open Data und KI im Einsatz: Geodaten für alle, nicht nur für Profis?%
}{%
}{%
  Im Rahmen der aktuellen Diskussion zur Nutzung von KI-Sprach\-modellen untersuchen wir in einem
  Proof of Concept, wie KI im Zusammenspiel mit offenen Geodaten eingesetzt werden kann. Kann KI
  dabei unterstützen, frei verfügbare Daten besser aufzufinden, verständlicher zu machen, ihre
  Relevanz für spezifische Projekte zu bewerten oder ganz neue Ideen zu entwickeln? Oder eröffnet
  sie gar die Möglichkeit, neue Nutzendengruppen außerhalb der traditionellen
  Geodaten-Expert:innenkreise anzusprechen?%
}%

\newpage
\enlargethispage{3.0\baselineskip}
%%%%%%%%%%%%%%%%%%%%%%%%%%%%%%%%%%%%%%%%%%%

% time: Wednesday 18:15
% URL: https://pretalx.com/fossgis2023/talk/fossgis2025-58059-mosidi-homogenisierung-von-offenen-daten-fr-die-kommunale-planung/

%
\enlargethispage{3.0\baselineskip}
\noindent\abstractHSvier{%
  Sebastian Meier, Leonard Higi%
}{%
  MOSIDI~-- Homogenisierung von offenen Daten  für die kommunale Planung%
}{%
}{%
  Im Rahmen des Projekts InNoWest wird ein homogenisiertes Datenschema für die kommunale Verwaltung
  entwickelt. Ziel ist es, heterogene räumliche Daten aus öffentlichen und Non-Profit-Quellen zu
  aggregieren und über ein niedrigschwelliges Benutzerinterface bereitzustellen. Durch die
  nutzerzentrierte Entwicklung, abgestimmt auf reale Bedürfnisse brandenburgischer Kommunen, können
  Daten flexibel kombiniert und neue Erkenntnisse für Planung und Daseinsvorsorge generiert werden.%
}%


%%%%%%%%%%%%%%%%%%%%%%%%%%%%%%%%%%%%%%%%%%%

\vspace{-0.2cm}
\sponsorBoxA{202_EFTAS_Slogan.png}{0.4\textwidth}{4}{%
\textbf{Silbersponsor und Aussteller}
\vspace{0.5\baselineskip}

\noindent {\bfseries EFTAS Fernerkundung Technologietransfer GmbH}\\
{\bfseries GeoIT rund um Luft- und Satellitenbilder!}  
Mit Geodaten können wir die Welt besser verstehen.
Deshalb entwickeln wir passgenaue GeoIT-Lösungen, die präzise sind und aktiv die Zukunft
gestalten. Dafür nutzen wir KI und Cloud-Compuing und erstellen individuelle,
technologieoffene Lösungen aus einer Hand, unabhängig von Software- und Datenprovidern.    
{\bfseries Unsere Anwendungswelten:}
\begin{itemize}
\item Land- \& Forstwirtschaft  
\item Umwelt, Natur \& Landschaft  
\item Stadt \& Verkehr  
\ietm Bergbau \& Georessourcen
\end{itemize}
{\bfseries Unser Fokus:}\\ 
Fernerkundung auf Basis von Luft- und Satellitenbildern, die großflächige Erfassung von In-
situ-Daten \& individuelle GeoIT-Anwendungen.   
{\bfseries Unser Ziel:}\\
Umweltrelevante Prozesse und Entscheidungen unterstützen \& beschleunigen - und zwar
weltweit.

\normalsize
}%

