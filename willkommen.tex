\newpage
\section*{Willkommen zur FOSSGIS-Konferenz 2023 in Berlin!} \label{welcome}
Die Abkürzung { FOSSGIS} steht für {\bfseries f}reie und {\bfseries O}pen"={\bfseries S}ource"={\bfseries S}oftware für {\bfseries G}eo{\bfseries i}nformations{\bfseries s}ysteme.
Die FOSSGIS-Konferenz 2023 wird vom gemeinnützigen FOSSGIS e.V, der
OpenStreetMap"=Community und der Humboldt-Universität zu Berlin
veranstaltet.
Ziel der jährlich stattfindenden Konferenz ist die Verbreitung von freier,
quelloffener Software für Geoinformationssysteme. In den nächsten vier Tagen
haben Sie die Gelegenheit, sich mit Entwicklern und anderen Anwendern
auszutauschen und \mbox{neueste} Informationen zu Anwendungen und
Arbeitsmöglichkeiten zu erhalten.

\section*{Anwender- und Expert:inentreffen, Birds of a Feather}
Die FOSSGIS-Konferenz ist eine Communityveranstaltung.
Die gesamte Konferenz über stehen zwei Räume für Anwender- und Expert:innentreffen oder spontan organisierte
Treffen Gleichgesinnter (Birds of a Feather), u.\,ä.
zur Verfügung. Eigene Sessions können Sie an der Pinnwand beim
Welcome Desk selbst eintragen.

\newpage
\include{sponsorentexte/001-camptocamp}
\newpage
\section*{Goldsponsor und Aussteller}
\begin{center}
  \includegraphics[width=0.8\textwidth]{002-wheregroup}
\end{center}
Die WhereGroup gehört zu den führenden Anbietern von Geoinformationssystemen
mit Open-Source-Software in Deutschland.

Wir bieten alle Dienstleistungen rund um die räumliche Datenverarbeitung,
Geoinformationssysteme und Datenbanken mit freier Software. Angefangen bei der
Beratung, Konzeption und Entwicklung bis hin zum Betrieb dynamischer
Kartenanwendungen im Intra- und Internet.

Dabei reicht das Spektrum unserer Lösungen von Geoportalen und kartenbasierter
Datenverwaltung bis hin zu hochverfügbaren Anwendungen für die freie Wirtschaft
und die öffentliche Verwaltung.

Unser Schulungsinstitut, die FOSS Academy, bietet außerdem praxisorientierte
Schulungen zum Thema „GIS mit Open-Source-Software“ an.  Das WhereGroup-Team
umfasst über 40 Kollegen unterschiedlicher Fachrichtungen verteilt auf die
Standorte Bonn (Hauptsitz), Freiburg und Berlin.  Grundlage unseres Schaffens
ist der Open-Source-Gedanke. Als Teil einer starken Community, mit der wir in
engem Austausch stehen, engagieren wir uns aus Überzeugung im FOSSGIS e. V. und
bei der OSGeo.

Mehr zur WhereGroup finden Sie unter www.wheregroup.com und www.foss-academy.com.

\newpage
\section*{Platinsponsor}
%\begin{center}
\begin{flushright}
  \includegraphics[width=0.6\textwidth]{003_hu_siegel-kombi_rgb.png}
 %\end{center}
\end{flushright}
\noindent
Die {\bfseries Humboldt-Universität zu Berlin}, gegründet 1810, ist die älteste Hochschule in Berlin
und eine der renommiertesten Universitäten weltweit. Das Lehr- und Forschungsangebot der HU umfasst heute alle grundlegenden
Wissenschaftsdisziplinen der Geistes-, Sozial- und Kulturwissenschaften, der Rechtswissenschaften, der Lebenswissenschaften, der Mathematik und Naturwissenschaften, der Medizin, der Agrarwissenschaften und der Nachhaltigkeits- und Antikeforschung. Aktuell studieren an der Humboldt-Universität fast 37\,000 junge Menschen aus über 100 Ländern in 171 Bachelor- und Masterstudiengängen betreut von über 400 Professor:innen. Zirka 34 Prozent der wissenschaftlichen Mitarbeiter:innen kommen aus anderen Ländern. Aufgrund zahlreicher Projekte der Spitzenforschung und renommierter internationaler Netzwerke ist die Humboldt-Universität eine der bedeutendsten Universitäten im deutschsprachigen Raum. Bei der Exzellenzstrategie 2019 wurde sie gemeinsam mit den Partner:innen der Berlin University Alliance als Exzellenzverbund ausgezeichnet. Zuvor gehörte sie seit 2012 zu einer der elf deutschen Exzellenzuniversitäten. Die HU verbindet Forschungsexzellenz mit innovativer Nachwuchsförderung. Im Fokus der Lehre stehen forschendes Lernen, Interdisziplinarität und Internationalisierung.

\newpage
\section*{Platinsponsor und Aussteller}

\vspace{-0.5cm}
\centerline{\includegraphics[width=0.6\textwidth]{004_TSB_quer.png}}
\small
Die Technologiestiftung Berlin ist eine unabhängige und gemeinnützige Stiftung. Wir arbeiten für ein lebenswertes, smartes Berlin – und eine lebendige, transparente Stadtgesellschaft, die alle am digitalen Wandel teilhaben lässt. Mit digitalen Tools und smarten Lösungen tragen wir aktiv dazu bei, dass Berlin offen, nachhaltig und effizient wird. Viele unserer Projekte sind Leuchttürme, die beispielhaft die Chancen der Digitalisierung zeigen und Berlin über die Stadtgrenzen hinaus profilieren.

Open Data und Open Source gehören zu unserer DNA. Gemeinsam mit Stadtgesellschaft, Verwaltung, Wissenschaft und Unternehmen nutzen wir das Potenzial offener Daten und Anwendungen, um Transparenz zu schaffen, Teilhabe zu ermöglichen und innovative Lösungen zu fördern. Und das in ganz unterschiedlichen Bereichen: Mit der Senatsverwaltung für Inneres, Digitalisierung und Sport bieten wir die Open Data Informationsstelle an. Sie unterstützt die Berliner Verwaltung bei der Bereitstellung offener Daten und entwickelt darauf basierend eigene Anwendungen wie die Visualisierung der Berliner Haushaltsdaten oder das Organigramm-Tool. Im Projekt kulturdaten.berlin schaffen wir die erste offene, digitale Infrastruktur für Kulturschaffende und -institutionen in Berlin, gefördert durch die Senatsverwaltung für Kultur und Europa. Und die Open-Map-Anwendung Gieß den Kiez vom CityLAB Berlin ermöglicht Bürger:innen die Pflege von über 800.000 Stadtbäumen.

Auf der FOSSGIS freuen wir uns auf den gemeinsamen Austausch und neue Impulse zu offenen Daten sowie Datenbank- und OpenStreetMap-Anwendungen!
\normalsize

\newpage
\section*{Goldponsor und Aussteller}
\begin{flushright}
\includegraphics[width=0.7\textwidth]{101_BKG_Logo_RGB.png}
\end{flushright}
\noindent
Das {\bfseries Bundesamt für Kartographie und Geodäsie (BKG)} ist eine Behörde im Geschäftsbereich des Bundesministeriums des Innern und für Heimat (BMI). Es fungiert als zentraler Dienstleister des Bundes und Kompetenzzentrum für Geoinformation und geodätische Referenzsysteme. Das BKG befasst sich mit der Beobachtung sowie der Datenhaltung bis hin zur Analyse, Kombination und Bereitstellung von Geodaten. Das BKG ermöglicht aufgrund der zentralen Geodatenbereitstellung eine optimale und wirtschaftliche Geodatennutzung im Bundesbereich.

\noindent
Das BKG setzt sich für eine offene Datenpolitik ein, wodurch die Verbreitung von Open Data gefördert wird. Dies schließt die Beratung anderer Bundesbehörden beim Umgang mit OSM-Daten ein. Die Nutzung, Entwicklung und Verbreitung der Nutzung freier Software liegen ebenfalls im Bereich der Aktivitäten.

\noindent
Von der Arbeit des BKG profitieren insbesondere Bundeseinrichtungen, die öffentliche Verwaltung, Wirtschaft, Wissenschaft~-- und fast jeder Bürger in Deutschland. Experten aus den verschiedensten Bereichen wie Verkehr, Katastrophenvorsorge, Innere Sicherheit, Energie und Umwelt verwenden Geodaten, Landkarten, Referenzsysteme und Informationsdienste des BKG für ihre Pläne und Untersuchungen. Das BKG unterhält ein Dienstleistungszentrum in Leipzig sowie geodätische Observatorien im In- und Ausland.

\newpage
\section*{Goldponsor und Aussteller}
\begin{center}
  \includegraphics[width=0.8\textwidth]{102_Logo_QFieldCloud-by-OpenGIS_transparent.png}
\end{center}
Wir planen und entwickeln personalisierte Open Source GIS Lösungen als Desktop-, Web- oder Mobilapplikationen für Ingenieurbüros, Organisationen und den öffentlichen Sektor – kosteneffizient, massgeschneidert und von A bis Z. Mit Open Source Technologie-Erfahrung, POSTGIS Expertise und QGIS und QField Entwicklerwissen finden wir elegante Lösungen auch für komplexe Aufgaben. Darauf geben wir Ihnen unser Schweizer GeoNinja-Ehrenwort - oder einen Support Vertrag mit SLA.

Aber nicht nur wir sind von uns überzeugt. Viel wichtiger, auch die Deutsche Bahn, das Bundesamt für Umwelt, mehrere Kantone und weitere Auftraggeber sind sich einig: OPENGIS.ch ist der ideale Partner, wenn es um Open Source GIS Projekte geht.

QFieldCloud ergänzt die mobile Applikation QField für die Synchronisierung der erfassten Daten und erleichtert die Zusammenarbeit von mehreren Personen oder Teams im Feld. Nutzer und Rollen werden klar definiert, Änderungen können nachverfolgt und erfasste Daten einfach über die Cloud synchronisiert werden.

