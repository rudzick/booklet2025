\newpage
\section*{Willkommen zur FOSSGIS-Konferenz 2020 in Freiburg!} \label{welcome}
Die Abkürzung { FOSSGIS} steht für {\bfseries f}reie und {\bfseries O}pen"={\bfseries S}ource"={\bfseries S}oftware für {\bfseries G}eo{\bfseries i}nformations{\bfseries s}ysteme.
Die FOSSGIS-Konferenz 2020 wird vom gemeinnützigen FOSSGIS e.V, der
OpenStreetMap"=Community und der Albert-Ludwigs-Universität Freiburg
veranstaltet.
Ziel der jährlich stattfindenden Konferenz ist die Verbreitung von freier,
quelloffener Software für Geoinformationssysteme. In den nächsten vier Tagen
haben Sie die Gelegenheit, sich mit Entwicklern und anderen Anwendern
auszutauschen und \mbox{neueste} Informationen zu Anwendungen und
Arbeitsmöglichkeiten zu erhalten.

\section*{Birds of a Feather und Anwendertreffen}
Die FOSSGIS-Konferenz ist eine Communityveranstaltung.
Die gesamte Konferenz über stehen zwei Räume für spontan organisierte
Treffen Gleichgesinnter (Birds of a Feather), Anwendertreffen u.\,ä.
zur Verfügung. Eigene Sessions können Sie an der Pinnwand beim
Welcome Desk selbst eintragen.

\section*{Kombiticket}
Namensschilder \emph{mit RVF-Kombiticket} gelten im angegebenen Gültigkeitszeitraum in Bussen, Straßenbahnen und Zügen im RVF-Gebiet in der zweiten Wagenklasse.

\newpage
\section*{Goldsponsor und Aussteller}
\begin{center}
	\includegraphics[width=0.97\textwidth]{001-geops}
\end{center}
Digitale Lösungen auf Basis von Geodaten und Echtzeitdaten in den
Bereichen Mobilität und Umwelt sind die Kernkompetenzen von geOps. Die
Bandbreite umfasst kundenspezifische Entwicklungen, Standard-Produkte und
-Dienste sowie Betrieb von Server-Infrastrukturen. Bei der Auswahl von
Technologien setzen wir einen klaren Fokus auf Open Source. Neben der
Nutzung offener Software unterstützen wir Open Source mit eigenem Code
und regelmässigen Sponsoring von Konferenzen, Hackathons und ähnlichen
Events.

geOps wurde 2002 in Freiburg gegründet und 2013 um eine
Niederlassung in der Schweiz erweitert. Aktuell beschäftigt geOps über 20
Mitarbeiter und Mitarbeiterinnen.  Wir wollen weiter wachsen und suchen
daher Verstärkung für unser Team.

\newpage
\section*{Goldsponsor und Aussteller}
\begin{center}
  \includegraphics[width=0.8\textwidth]{002-wheregroup}
\end{center}
Die WhereGroup gehört zu den führenden Anbietern von Geoinformationssystemen
mit Open-Source-Software in Deutschland.

Wir bieten alle Dienstleistungen rund um die räumliche Datenverarbeitung,
Geoinformationssysteme und Datenbanken mit freier Software. Angefangen bei der
Beratung, Konzeption und Entwicklung bis hin zum Betrieb dynamischer
Kartenanwendungen im Intra- und Internet.

Dabei reicht das Spektrum unserer Lösungen von Geoportalen und kartenbasierter
Datenverwaltung bis hin zu hochverfügbaren Anwendungen für die freie Wirtschaft
und die öffentliche Verwaltung.

Unser Schulungsinstitut, die FOSS Academy, bietet außerdem praxisorientierte
Schulungen zum Thema „GIS mit Open-Source-Software“ an.  Das WhereGroup-Team
umfasst über 40 Kollegen unterschiedlicher Fachrichtungen verteilt auf die
Standorte Bonn (Hauptsitz), Freiburg und Berlin.  Grundlage unseres Schaffens
ist der Open-Source-Gedanke. Als Teil einer starken Community, mit der wir in
engem Austausch stehen, engagieren wir uns aus Überzeugung im FOSSGIS e. V. und
bei der OSGeo.

Mehr zur WhereGroup finden Sie unter www.wheregroup.com und www.foss-academy.com.

\newpage
\section*{Goldsponsor und Aussteller}
\begin{center}
  \includegraphics[width=0.8\textwidth]{003-camptocamp}
\end{center}
Camptocamp gehört zu den führenden Dienstleistern im Bereich Open-Source-GIS
und ist in vielen unterschiedlichen Open-Source-Communitys stark engagiert.
Unsere Dienstleistungen stützen sich auf über 15 Jahre Erfahrung in der
Umsetzung von innovativen GIS-Lösungen für Behörden und Unternehmen und
erlauben einen hochwertigen und individuellen Service. Das Besondere an
Camptocamp sind die hochqualifizierten Mitarbeiter und ihr großes Engagement im
"`Ökosystem"' der eingesetzten Open-Source Software-Lösungen, indem sehr enge
Beziehungen zu den Herstellern der jeweiligen Produkte gepflegt werden.

Um die oft anspruchsvollen Projekte umzusetzen, erstellt Camptocamp
individuelle Lösungen, die auf den am besten geeigneten und fortschrittlichsten
Open Source-Technologien basieren. Camptcamp ist in München, Lausanne, Olten,
Paris und Chambéry vertreten und bietet neben Lösungen im GIS-Bereich auch eine
große Expertise im den Bereichen ERP (Enterprise-Resource-Planning) und
IT-Infrastruktur-Lösungen.

\newpage
\section*{Goldsponsor}
\begin{center}
  \includegraphics[width=0.8\textwidth]{004-linuxhotel}
\end{center}
Im Linuxhotel finden IT-Fachleute ein Lernumfeld für intensive Schulungen zu
Open"=Source"=Software. Durch unsere einmalige Kombination aus
Rückzugsmöglichkeiten und Gruppenarbeit setzen sich Schulungsteilnehmer viel
tiefer mit den behandelten Themen auseinander.  Auch im GIS-Bereich bieten wir
Fachanwendern, Systemadministratoren und Entwicklern ein breites
Schulungsangebot:

\begin{itemize}
  \setlength{\itemsep}{-3pt}
  \item QGIS und GRASS
  \item PostGIS
  \item WebGIS \& Mapserver
  \item OpenLayers
  \item Leaflet
  \item Mapfish
  \item Python
  \item JavaScript
  \item SQL-Grundlagen
  \item Linux-Grundlagen
  \item Linux-Administration
\end{itemize}

\newpage
\section*{Goldsponsor und Aussteller}
\newlength\sourcepoleHeight
\setlength\sourcepoleHeight{2.5\baselineskip}
\begin{wrapfigure}[4]{r}[0pt]{0.35\textwidth}
  \centering\vspace{-1\baselineskip}
  \includegraphics[width=0.35\textwidth]{005-sourcepole}
\end{wrapfigure}
%\begin{center}
%  \includegraphics[height=\sourcepoleHeight]{005-sourcepole}
%\end{center}
Sourcepole entwickelt kundenspezifische Lösungen im Bereich Geoinformatik. Wir
erweitern und integrieren Open"=Source"=Software nach Ihren Bedürfnissen.  Als
offizielle Commiter in mehreren OSGeo"=Projekten können wir auch direkt in den
Original"=Code eingreifen.

\begin{wrapfigure}[3]{r}[0pt]{0.35\textwidth}
  \centering\vspace{-1\baselineskip}
  \includegraphics[width=0.35\textwidth]{005-qgis-enterprise}
\end{wrapfigure}
%\begin{center}
%  \hfill
%  \includegraphics[width=0.4\textwidth]{005-qgis-enterprise}
%  \hfill
%  \includegraphics[width=0.4\textwidth]{005-qgis-cloud}
%  \hfill
%\end{center}

QGIS Enterprise ist ein komplettes Wartungs- und Support-Paket für eine GDI,
die vollständig auf Open"=Source"=Software aufbaut.  Kern des Angebotes ist die
GIS"=Suite basierend auf QGIS Desktop, QGIS Server und QGIS WebClient.

\begin{wrapfigure}[3]{r}[0pt]{0.35\textwidth}
  \centering\vspace{-1\baselineskip}
  \includegraphics[width=0.35\textwidth]{005-qgis-cloud}
\end{wrapfigure}
Mit QGIS Cloud können Karten auf einfache Weise im Internet publiziert werden.
Ein Plugin für QGIS Desktop ermöglicht das Hochladen von Daten und das
Publizieren fertiger Karten. Auf Knopfdruck werden sie auf qgiscloud.com für
Web-Browser, mobile Clients und als OGC WMS verfügbar.

%\begin{center}
%  \includegraphics[height=\sourcepoleHeight]{005-qwc-services}
%\end{center}

\begin{wrapfigure}[7]{r}[0pt]{0.35\textwidth}
  \centering%\vspace{-1\baselineskip}
  \includegraphics[width=0.30\textwidth]{005-qwc-services}
\end{wrapfigure}
Der QGIS WebClient 2 (QWC2) ist ein moderner Kartenclient, der auf die
Publikation von Karten mit QGIS Server spezialisiert ist. Dank dem Einsatz von
Micro-Services ist er sowohl für die Erstellung einfacher In"=House"=Clients, als
auch für umfangreiche Lösungen in Enterprise-Infrastrukturen geeignet.

\newpage
\section*{Goldsponsor und Aussteller}
\begin{center}
  \includegraphics[width=0.97\textwidth]{006-opengis}
\end{center}
{\dejavuSansFont OPENGIS.ch:} Wir sind {\dejavuSansFont Schweizer Software-Entwickler}~-- aber noch mehr sind
wir {\dejavuSansFont Geo-Ninjas!} Wir planen und entwickeln personalisierte {\dejavuSansFont Open-Source-GIS-Lösungen} als {\dejavuSansFont Desk\-top- oder Mobilapplikationen}. Elegante Lösungen finden
wir übrigens auch {\dejavuSansFont für vielschichtige Aufgaben}.

Wie wir das machen? Mit
jahrelanger {\dejavuSansFont Open"=Source"=Technologie"=Erfahrung, QGIS} und {\dejavuSansFont POSTGIS"=Expertise,
QField- und QField\-Cloud"=Entwicklerwissen} und unserem eigenen, hohen
{\dejavuSansFont Schweizer Anspruch.

Persönlicher Service} in {\dejavuSansFont sechs Sprachen} und eine
ausgezeichnete {\dejavuSansFont Projektunterstützung von A-Z} gehören da natürlich dazu. Wir
übernehmen gerne {\dejavuSansFont Verantwortung} für das, was wir tun. Darauf geben wir
Ihnen unser {\dejavuSansFont Ninja"=Ehrenwort}~-- oder einen {\dejavuSansFont Support"=Vertrag mit SLA}.

