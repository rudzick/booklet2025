\newpage
\section*{Willkommen zur FOSSGIS-Konferenz 2024 in Hamburg!}\label{welcome}
Die Abkürzung { FOSSGIS} steht für {\bfseries f}reie und {\bfseries O}pen"={\bfseries S}ource"={\bfseries S}oftware für {\bfseries G}eo{\bfseries i}nformations{\bfseries s}ysteme.
Die FOSSGIS-Konferenz 2024 wird vom gemeinnützigen FOSSGIS e.V., der
OpenStreetMap"=Community in Kooperation mit dem Institut für Verkehrsplanung und Logistik der Technischen Universität Hamburg und dem Labor für Geoinformatik und Geovisualisierung der HafenCityUniversität veranstaltet. 
Ziel der jährlich stattfindenden Konferenz ist die Verbreitung von freier,
quelloffener Software für Geoinformationssysteme. In den nächsten vier Tagen
haben Sie die Gelegenheit, sich mit Entwickler:innen und anderen Anwender:innen
auszutauschen und \mbox{neueste} Informationen zu Anwendungen und
Arbeitsmöglichkeiten zu erhalten.

\section*{Anwender- und Expert:innentreffen, Birds of a Feather}
Die FOSSGIS-Konferenz ist eine Communityveranstaltung.
Die gesamte Konferenz über stehen drei Räume für Anwender- und Expert:innentreffen oder spontan organisierte Treffen Gleichgesinnter (Birds of a Feather), u.\,ä.
zur Verfügung. Eigene Sessions können Sie an der Pinnwand im Foyer selbst eintragen.

\section*{Abendveranstaltung am Mittwoch}\label{schwaetzli}
Am ersten Abend der FOSSGIS-Konferenz findet die Abendveranstaltung im Privathotel Lindtner Hamburg (Heimfelder Straße 123, 21075 Hamburg) von {\bfseries 19:15 bis 22:00 Uhr} statt. Eine Anmeldung ist erforderlich.
\bigskip

\begin{minipage}[c]{0.9\textwidth}
  \includegraphics[width=1.0\textwidth]{ioki_shuttle_social_event.png}
\end{minipage}


\section*{Exkursionen {\normalfont\em (Anmeldung erforderlich)}}
\noindent
{\large \bfseries Einblick in die Kartensammlung der Staats- und Universitätsbibliothek Hamburg}\\
Es wird eine Auswahl von Karten und Atlanten ausgelegt, die man sich ansehen kann. {\bfseries Donnerstag, 21.03.2024, 9-11 Uhr}.
\bigskip

\noindent
{\large \bfseries Abendliche Hafenrundfahrt}\\
Geplante Tour: TUHH – Wilhelmsburg – Köhlbrandbrücke – Finkenwerder – Landungsbrücken am {\bfseries Freitag, 22.03.2024 um 17:45 Uhr}, Treffpunkt Haltestelle Kasernenstraße (TU Harburg) stadtauswärts.
\bigskip

\noindent
{\large \bfseries Hamburger Unterwelten}\\
Besuch des Tiefbunkers Steintorwall am {\bfseries Freitag, 22.03.2024 um 18 Uhr }. Treffpunkt wird bekanntgegeben.
\bigskip

\noindent
{\large \bfseries Stadtrundgang}\\
Geplante Tour: Michaeliskirche – Elbtunnel – Elbphilharmonie – Speicherstadt – Karolinenviertel – Rathaus – Binnenalster am {\bfseries Samstag, 23.03.2024, 9:15 Uhr }, Treffpunkt: S-Bahn Haltestelle Stadthausbrücke, Bahnsteig, Aufgang zur Michaeliskirche.
\bigskip

\noindent
{\large \bfseries MILLERNTOUR! Stadionführung bei St.~Pauli}\\
Die Stadionführung beim FC St. Pauli gibt Einblicke in die Geschichte und Kultur eines ungewöhnlichen Vereins. {\bfseries Samstag, 23.03.2024, 14 Uhr}.

\section*{Rahmenprogramm am Donnerstag}
\subsection*{Gruppenfoto}
Auch in diesem Jahr wollen wir uns das Gruppenfoto nicht entgehen lassen und laden Sie am {\bfseries Donnerstag} in der {\bfseries Nachmittagspause} zum Gruppenfoto ein zwischen Haus H und Haus I an den halbrunden Stufen.

\subsection*{Mitgliederversammlung des FOSSGIS e.V.}
Am {\bfseries Donnerstag} sind alle Mitglieder und Gäste ab {\bfseries 19.00 Uhr} herzlich zur Mitgliederversammlung des FOSSGIS e.V. eingeladen zum Diskutieren, Kennenlernen, Abstimmen und zu Neuwahlen. Ab 18:30 Uhr
gibt es Getränke und Pizza für alle. Der Verein freut sich über zahlreiches Erscheinen.

\section*{Rahmenprogramm am Freitag}
\subsection*{Jeopardy-Quiz}
Am {\bfseries Freitag} gibt es vor dem Abschluss des regulären Konferenzprogramms um {\bfseries 14.45 Uhr} das nunmehr legendäre und sehr humorvolle Quiz in Form des FOSSGIS-Jeopardy mit Hannes, Güren und Mathias.

\subsection*{Sektempfang am FOSSGIS-Stand}
Alle Mitglieder des FOSSGIS-Vereins, Freunde und Interessierte sind am {\bfseries Freitag} ab {\bfseries 16.30 Uhr} herzlich zum Sektempfang zum Ausklang der FOSSGIS 2024 am FOSSGIS-Vereins-Stand eingeladen.

\subsection*{OSM-Event am Freitagabend}
Für alle, die am {\bfseries Freitagabend} noch in der Stadt sind und/oder am OSM-Event teilnehmen möchten gibt es ab {\bfseries 19:00 Uhr} ein gemeinsames Treffen in de Fischhalle-Harburg, Kanalplatz 16, 21079, Hamburg. Jeder zahlt seine Rechnung selbst.

%\cleartoevenpage
\small
\label{platinsposoren}
%\cleardoublepage
\include{sponsorentexte/001-wheregroup}
%\newpage
\cleardoubleevenpage
\section*{Platinsponsor und Aussteller}
%\vspace*{-0.7\baselineskip}
%\begin{center}
  \includegraphics[width=0.7\textwidth]{003_dataport.png}
  %\end{center}
  \vspace{1.0\baselineskip}
  
\noindent
    {\bfseries Dataport - Anstalt des öffentlichen Rechts}
    \vspace{1.0\baselineskip}
    
\noindent
    {\bfseries Das Unternehmen}
    
\noindent    
Dataport ist der Partner für die Digitalisierung des öffentlichen Sektors. Für Schleswig-Holstein, Hamburg, Bremen und Sachsen-Anhalt. Und für die Steuerverwaltung in Niedersachsen und Meck\-lenburg-Vorpommern. Als IT-Dienstleister gestaltet Dataport den digitalen Wandel gemeinsam mit Ländern und Kommunen. Mit rund 5.000 Mitarbeiter:innen an acht Standorten. Das Unternehmen erzielte 2022 einen Umsatz von 1,18 Milliarden Euro.    
Dataport ist eine Anstalt des öffentlichen Rechts. Träger sind die Länder Schleswig-Holstein, Hamburg, Bremen, Sachsen-Anhalt, Niedersachsen und Meck\-lenburg-Vorpommern. Das Unternehmen arbeitet nicht gewinnorientiert, sondern strebt in enger Absprache mit den Trägern ein ausgeglichenes Betriebsergebnis an
\vspace{1.0\baselineskip}

\noindent
{\bfseries Der Auftrag}

\noindent
Dataport stellt dem öffentlichen Sektor alle benötigten IT-Services zur Verfügung. Dazu gehören der Betrieb von Infrastrukturen wie Rechenzentrum, Netze und Clients oder die zentrale Beschaffung von Informationstechnologie (IT). Außerdem die Entwicklung und der Betrieb von Software. Dataport unterstützt bei allen Aspekten der Digitalisierung. Durch umfassendes Consulting, Projektmanagement, Innovationsmanagement oder Geschäftsprozessmanagement.
\newpage
\noindent
{\bfseries Die Werte}
\vspace{1.0\baselineskip}

\noindent
{\em Digitale Sicherheit:}\\
Das Twin Data Center ist eines der sichersten Rechenzentren in Europa. Zertifiziert vom Bundesamt für Sicherheit in der Informationstechnik (BSI) und vom TÜViT. Der Betrieb von IT-Systemen folgt eindeutigen Prozessen und Regeln unter ständiger Kontrolle durch ein professionelles Sicherheitsmanagement. Das sorgt für Sicherheit.
\vspace{1.0\baselineskip}

\noindent
{\em Digitale Souveränität:}\\
Damit der Staat handlungsfähig und vertrauenswürdig ist, muss er stets die vollständige Hoheit über seine IT-Systeme und seine Daten behalten. Dafür sorgt Dataport mit hohen Sicherheitsstandards, klaren Datenschutzvorgaben und vorausschauender Unternehmenspolitik. Damit Bürger*innen und Unternehmen der Verwaltung ihre Daten anvertrauen können.
\vspace{1.0\baselineskip}

\noindent
{\em Kooperation:}\\
Seit der Gründung 2004 organisiert Dataport mit seinen Trägern erfolgreiche IT-Kooperationen. Der Dienstleister sorgt für Kooperationen auf verschiedenen Ebenen. Durch gemeinsam genutzte Infrastrukturen wie das Twin Data Center. Durch die gemeinsame Entwicklung von IT-Lösungen. Durch das Schaffen von Schnittstellen in IT-Systemen, die eine Zusammenarbeit ermöglichen. Länderübergreifend zwischen Bundesländern sowie ebenenübergreifend zwischen Bund, Ländern und Kommunen.

%\newpage
\include{sponsorentexte/002-camptocamp}
%\newpage
\section*{Goldponsor und Aussteller}
\begin{flushright}
\includegraphics[width=0.7\textwidth]{101_BKG_Logo_RGB.png}
\end{flushright}
\noindent
Das {\bfseries Bundesamt für Kartographie und Geodäsie (BKG)} ist eine Behörde im Geschäftsbereich des Bundesministeriums des Innern und für Heimat (BMI). Es fungiert als zentraler Dienstleister des Bundes und Kompetenzzentrum für Geoinformation und geodätische Referenzsysteme. Das BKG befasst sich mit der Beobachtung sowie der Datenhaltung bis hin zur Analyse, Kombination und Bereitstellung von Geodaten. Das BKG ermöglicht aufgrund der zentralen Geodatenbereitstellung eine optimale und wirtschaftliche Geodatennutzung im Bundesbereich.

\noindent
Das BKG setzt sich für eine offene Datenpolitik ein, wodurch die Verbreitung von Open Data gefördert wird. Dies schließt die Beratung anderer Bundesbehörden beim Umgang mit OSM-Daten ein. Die Nutzung, Entwicklung und Verbreitung der Nutzung freier Software liegen ebenfalls im Bereich der Aktivitäten.

\noindent
Von der Arbeit des BKG profitieren insbesondere Bundeseinrichtungen, die öffentliche Verwaltung, Wirtschaft, Wissenschaft~-- und fast jeder Bürger in Deutschland. Experten aus den verschiedensten Bereichen wie Verkehr, Katastrophenvorsorge, Innere Sicherheit, Energie und Umwelt verwenden Geodaten, Landkarten, Referenzsysteme und Informationsdienste des BKG für ihre Pläne und Untersuchungen. Das BKG unterhält ein Dienstleistungszentrum in Leipzig sowie geodätische Observatorien im In- und Ausland.

%\newpage
\section*{Goldponsor und Aussteller}
\begin{flushright}
\includegraphics[width=1.0\textwidth]{102_LGV.png }
\end{flushright}
\noindent
Der {\bfseries Landesbetrieb Geoinformation und Vermessung (LGV)} ist der zukunftsgestaltende und innovative Dienstleister, wenn es um die Erhebung, Pflege und Bereitstellung von (Geo-)Daten geht. Zu seinem Angebotsportfolio gehören IT-basierte urbane (Geo-)Anwendungen genauso, wie 3D-Darstellungen, Datenanalysen, alle vermessungsrelevanten Aufgaben sowie Immobilienbewertungen.

\noindent
Aufgrund seiner langjährigen Erfahrung in der Geodäsie und Geoinformation ist der LGV nicht nur in Hamburg, sondern auch bundesweit Impulsgeber und technologischer Vorreiter für die digitale Vernetzung und Online-Darstellung von (Geo-)Daten.

\noindent
Als Teil der Hamburger Behörde für Stadtentwicklung und Wohnen agiert der LGV seit 2003 eigenständig. Circa 370 Beschäftigte kümmern sich in Hamburg-Wilhelmsburg um die städtischen Anforderungen seitens der Bürgerinnen und Bürger, Verwaltungen und Wirtschaft. Der LGV unterstützt den Senat bei der Umsetzung seiner Strategie „Digitale Stadt“.


%\newpage
\section*{Goldponsor}
\begin{flushright}
\includegraphics[width=1.0\textwidth]{103_opengisch.png}
\end{flushright}
\noindent
{\bfseries OPENGIS.ch GmbH} Wir planen und entwickeln personalisierte Open Source GIS Lösungen als Desktop-, Web- oder Mobilapplikationen für Ingenieurbüros, Organisationen und den öffentlichen Sektor – kosteneffizient, massgeschneidert und von A bis Z. Mit Open Source Technologie-Erfahrung, POSTGIS Expertise und QGIS und QField Entwicklerwissen finden wir elegante Lösungen auch für komplexe Aufgaben. Darauf geben wir Ihnen unser Schweizer GeoNinja-Ehrenwort - oder einen Support Vertrag mit SLA.

\noindent
Aber nicht nur wir sind von uns überzeugt. Viel wichtiger, auch die Deutsche Bahn, das Bundesamt für Umwelt, mehrere Kantone und weitere Auftraggeber sind sich einig: OPENGIS.ch ist der ideale Partner, wenn es um Open Source GIS Projekte geht.

\noindent
QFieldCloud ergänzt die mobile Applikation QField für die Synchronisierung der erfassten Daten und erleichtert die Zusammenarbeit von mehreren Personen oder Teams im Feld. Nutzer und Rollen werden klar definiert, Änderungen können nachverfolgt und erfasste Daten einfach über die Cloud synchronisiert werden.


%\newpage
\section*{Goldponsor}
\begin{flushright}
\includegraphics[width=1.0\textwidth]{104_Mobidrom_Logo.png}
\end{flushright}
\noindent
 {\bfseries NRW.Mobidrom GmbH – Partner für Mobilitätsdaten in NRW}\\
Das Mobidrom ist ein IT-Unternehmen des Landes NRW und Ansprechpartner für digitalisierte und vernetzte Mobilität. Oberstes Ziel ist, das Bereitstellen und Nutzen von Mobilitätsdaten so einfach wie möglich zu gestalten. Dazu werden diese künftig über die Mobidrom Datenplattform NRW-weit gebündelt, aufbereitet und zur Verfügung gestellt.

\noindent
Darüber hinaus verantwortet das Mobidrom die Weiterentwicklung und den Betrieb von Verkehr.NRW. Das kartenbasierte Verkehrsportal macht die Verkehrslage und -angebote in Echtzeit zugänglich und ermöglicht Reiseplanung mit unterschiedlichen Verkehrsmitteln. Seit Anfang 2025 erweitert SEVAS die Mobidrom-Produktfamilie mit Daten für eine effiziente Lkw-Navigation.

\noindent
Das Mobidrom steht allen Datengebern und -nutzern im Bereich Mobilitätsdaten beratend zur Seite und beteiligt sich aktiv an Entwicklung und Betrieb von Open-Source-Anwendungen. Dazu gehört das MOTIS-Projekt, das den Kern für die intermodalen Mobidrom Routing Services darstellt.

\normalsize

