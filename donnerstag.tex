
% time: Thursday 09:00
% URL: https://pretalx.com/fossgis2023/talk/fossgis2025-58123-der-digitale-zwilling-so-wertvoll-wie-eine-karte-im-mastab-1-1/

%
\newTimeslot{09:00}
\noindent\abstractHSeins{%
  Matthias Daues%
}{%
  Der Digitale Zwilling~-- so wertvoll wie eine Karte im Maßstab 1:1%
}{%
}{%
  Digitale Zwillinge sind heute in aller Munde: Ein Allheilmittel für Probleme der räumlichen
  Modellierung. Aber stimmt das? Übersehen wir da nicht im Rausch des Machbaren die Frage nach dem
  Sinnvollen?
  Ein "`Rant"' gegen die "`Landkarte im Maßstab 1:1"'.%
}%


%%%%%%%%%%%%%%%%%%%%%%%%%%%%%%%%%%%%%%%%%%%

% time: Thursday 09:00
% URL: https://pretalx.com/fossgis2023/talk/fossgis2025-58195-routing-aber-mehr-explorativ-statt-automatisch/

%

\noindent\abstractHSzwei{%
  Katharina Rasch%
}{%
  Routing, aber mehr explorativ statt automatisch%
}{%
}{%
  Bei langen Zugreisen quer durch Europa ist auch immer ein bisschen der Weg das Ziel. Beim Planen
  solcher Reisen ist für mich der von Routingtools vorgeschlagene Reiseplan oft nur ein
  Ausgangspunkt. Ich möchte easy den Plan anpassen können, Verbindungen hin und her schieben,
  schauen, wie sich das auswirkt; vielleicht hier eine Übernachtung einlegen, oder dort;  vielleicht
  einen Umweg über eine Stadt, in der ich noch nie war. Genau dafür baue ich eine UI und von der
  möchte ich euch gern erzählen.%
}%


%%%%%%%%%%%%%%%%%%%%%%%%%%%%%%%%%%%%%%%%%%%

% time: Thursday 09:00
% URL: https://pretalx.com/fossgis2023/talk/fossgis2025-58261-anreicherung-von-straendaten-mittels-deep-learning-methoden-und-mapillary-bildern/

%

\noindent\abstractHSdrei{%
  Benjamin Herfort, Sukanya Randhawa%
}{%
  Anreicherung von Straßendaten mittels Deep-Learning-Methoden und Mapillary Bildern%
}{%
}{%
  Wir haben einen globalen Datensatz zu Straßeneigenschaften ("befestigt"' oder "`unbefestigt")
  mithilfe von KI-Methoden erstellt, basierend auf 105 Millionen Bildern von Mapillary . Unser
  Ansatz kombiniert SWIN-Transformer-Vorhersagen des Straßenbelags mit einer CLIP-Filterung
  minderwertiger Bilder. Der Datensatz deckt 36 \% des weltweiten Straßennetzes ab, wobei die
  Abdeckung in Städten höher ist. Er kann in der Stadtplanung, im Katastrophenmanagement, in der
  Logistik genutzt werden.%
}%


%%%%%%%%%%%%%%%%%%%%%%%%%%%%%%%%%%%%%%%%%%%

% time: Thursday 09:00
% URL: https://pretalx.com/fossgis2023/talk/fossgis2025-57589-g3w-suite-das-os-framework-fr-die-verffentlichung-und-verwaltung-von-qgis-projekte/

%

\noindent\abstractHSvier{%
  Walter Lorenzetti, Antonello Andrea%
}{%
  G3W-SUITE: das OS-Framework für die Veröffentlichung und Verwaltung von QGIS-Projekte%
}{%
}{%
  G3W-SUITE ist eine modulare Client-Server-Anwendung (basierend auf QGIS-Server) für die Verwaltung
  und Veröffentlichung von interaktiven kartographischen QGIS-Projekten verschiedener Art auf eine
  völlig unabhängige, einfache und schnelle Weise.
  Die Anwendung, die auf GitHub unter der Mozilla Public Licence 2.0 veröffentlicht wurde, ist mit
  den QGIS LTR Versionen kompatibel und basiert auf einer starken Integration mit der QGIS API.%
}%


%%%%%%%%%%%%%%%%%%%%%%%%%%%%%%%%%%%%%%%%%%%

% time: Thursday 09:00
% URL: https://pretalx.com/fossgis2023/talk/fossgis2025-57915-kartographie-mit-qgis/

%

\noindent\abstractAnwBoFeins{%
  Mathias Gröbe, Andreas Neumann%
}{%
  Kartographie mit QGIS%
}{%
}{%
  QGIS bringt inzwischen viele für die Kartenherstellung relevante Funktionen mit~-- aber die ein
  oder andere Funktion wäre noch wünschenswert. Die Session soll die Möglichkeit bieten, sich über
  Lösungen auszutauschen, Ideen zu entwickeln und zu überlegen, wie man das ein oder andere noch in
  QGIS implementieren könnte.%
}%


%%%%%%%%%%%%%%%%%%%%%%%%%%%%%%%%%%%%%%%%%%%

% time: Thursday 09:00
% URL: https://pretalx.com/fossgis2023/talk/fossgis2025-58019-openstreetmap-fr-und-mit-ffentlicher-verwaltung-und-behrden/

%

\noindent\abstractAnwBoFdrei{%
  Lars Lingner, Christopher Lorenz, Tobias Jordans%
}{%
  OpenStreetMap für und mit  öffentlicher Verwaltung und Behörden%
}{%
}{%
  Wie können Daten für OpenStreetMap zur Verfügung gestellt werden und wie stellt OpenStreetMap
  Daten zur Verfügung? Welche best practices gibt es und welche Sackgassen? Diese Fragestunde ist
  ein Angebot an Fachanwender:innen und andere Mitarbeiter:innen in öffentlicher Verwaltung und
  Behörden.%
}%


%%%%%%%%%%%%%%%%%%%%%%%%%%%%%%%%%%%%%%%%%%%

% time: Thursday 09:35
% URL: https://pretalx.com/fossgis2023/talk/fossgis2025-58084-der-wuppertaler-weg-vom-geoportal-zum-digitalen-zwilling/

%
\newTimeslot{09:35}
\noindent\abstractHSeins{%
  Stefan Sander%
}{%
  Der Wuppertaler Weg vom Geoportal zum Digitalen Zwilling%
}{%
}{%
  Die Stadt Wuppertal realisiert in einem bis Ende 2026 laufenden Förderprojekt einen Urbanen
  Digitalen Zwilling (UDZ), den "`DigiTal Zwilling"'. In diesem Vortrag werden sie softwaretechnischen
  Herausforderungen und Lösungsansätze bei der Entwicklung der Anwendungskomponenten zum DigiTal
  Zwilling vorgestellt. Insbesondere entsteht eine generische Anwendung, die den Erwartungen an
  einen UDZ gerecht werden soll, aber auch die klassischen Anwendungsfälle eines kommunalen
  Geoportals bedienen soll.%
}%


%%%%%%%%%%%%%%%%%%%%%%%%%%%%%%%%%%%%%%%%%%%

% time: Thursday 09:35
% URL: https://pretalx.com/fossgis2023/talk/fossgis2025-58082-flexibles-open-source-routing-mit-valhalla/

%

\noindent\abstractHSzwei{%
  Christian Beiwinkel%
}{%
  Flexibles Open Source Routing mit Valhalla%
}{%
}{%
  Valhalla ist eine vielseitige Routing Engine für das OSM Ökosystem, die vor allem durch ihre
  Flexibilität und RAM-Effizienz glänzt. Dieser Vortrag setzt sich zusammen aus einer allgemeinen
  Einführung in Valhalla sowie Updates zu neuen und verbesserten Features.%
}%


%%%%%%%%%%%%%%%%%%%%%%%%%%%%%%%%%%%%%%%%%%%

% time: Thursday 09:35
% URL: https://pretalx.com/fossgis2023/talk/fossgis2025-58250-versiegelungsanalyse-zur-bioklimatischen-bewertung-von-stadtgebieten/

%

\noindent\abstractHSdrei{%
  Markus Metz, Markus Eichhorn, Victoria-Leandra Brunn, Anika Weinmann%
}{%
  Versiegelungsanalyse zur bioklimatischen Bewertung von Stadtgebieten%
}{%
}{%
  Für eine bioklimatische Bewertung des Stadtgebietes Düsseldorf wurde der Grad der Versiegelung
  erfasst, wobei verschiedene Arten von vollversiegelten, teilversiegelten und unversiegelten
  Flächen unterschieden wurden. Zusätzlich wurde hohe Vegetation über diesen Flächen erfasst. Für
  diese Klassifikation wurde ein neuronales Netz (U-Net) trainiert und auf die Jahre 2021, 2022 und
  2023 angewendet.%
}%


%%%%%%%%%%%%%%%%%%%%%%%%%%%%%%%%%%%%%%%%%%%

% time: Thursday 09:35
% URL: https://pretalx.com/fossgis2023/talk/fossgis2025-58149-historische-karten-mit-qgis-georeferenzieren/

%

\noindent\abstractHSvier{%
  Niklas Alt, Johannes Wagner%
}{%
  Historische Karten mit QGIS georeferenzieren%
}{%
}{%
  Dank umfangreicher Digitalisierungskampagnen steht historisch Interessierten ein enormer Schatz an
  kartografischen Produkten aus mehr als zwei Jahrhunderten zur Verfügung. Mit dem
  QGIS-Georeferencer können diese einfach und niedrigschwellig für die Nutzung im beliebten
  Geoinformationssystem aufbereitet werden. Wir demonstrieren die wichtigsten Vorgehensweisen,
  weisen auf Fallstricke hin und geben wertvolle Produktivitätstipps für die Referenzierung.%
}%


%%%%%%%%%%%%%%%%%%%%%%%%%%%%%%%%%%%%%%%%%%%

% time: Thursday 10:00
% URL: https://pretalx.com/fossgis2023/talk/fossgis2025-58083-masterportal-liegenschaftsauskunft-onlika-4-0-mit-keycloak-und-bundid/

%
\newTimeslot{10:00}
\noindent\abstractHSeins{%
  Laura Meierkort%
}{%
  Masterportal~-- Liegenschaftsauskunft ONLIKA 4.0 mit Keycloak und BundID%
}{%
}{%
  Im Rahmen dieses Vortrags wird die "`Online-Auskunft für die Daten des Liegenschaftskatasters"
  Thüringen~-- kurz ONLIKA 4.0~-- vorgestellt.  Das automatisierte Abrufverfahren wurde vom Thüringer
  Landesamt für Bodenmanagement und Geoinformation (TLBG) in Zusammenarbeit mit grit entwickelt. Das
  Rollen- und Rechtekonzept wird durch die Authentifizierung via BundID und Keycloak abgebildet.
  Basis ist die Open Source-Lösung Masterportal.%
}%


%%%%%%%%%%%%%%%%%%%%%%%%%%%%%%%%%%%%%%%%%%%

% time: Thursday 10:00
% URL: https://pretalx.com/fossgis2023/talk/fossgis2025-58079-transitous-freies-public-transport-routing/

%

\noindent\abstractHSzwei{%
  Volker Krause%
}{%
  Transitous~-- Freies Public Transport Routing%
}{%
}{%
  Während es im OSM Umfeld eine Reihe Routing-Dienste für den Individualverkehr gibt, fehlte bisher
  ein freier und unabhängig betriebener Routing-Dienst für den öffentlichen Personenverkehr, der
  nicht auf einen engen räumlichen Bereich begrenzt ist. Transitous schließt diese Lücke.%
}%


%%%%%%%%%%%%%%%%%%%%%%%%%%%%%%%%%%%%%%%%%%%

% time: Thursday 10:00
% URL: https://pretalx.com/fossgis2023/talk/fossgis2025-58081-ermittlung-von-versiegelten-flchen-als-komplexe-aufgabe-projekt-seal-/

%

\noindent\abstractHSdrei{%
  Peter Lorkowski%
}{%
  Ermittlung von versiegelten Flächen als komplexe Aufgabe (Projekt SEAL)%
}{%
}{%
  Die Kartierung von Flächenversiegelungen wird für immer mehr Bereiche relevant. Bei der Auswahl
  vorzugsweise freier Daten und Software besteht eine Vielfalt von Möglichkeiten. Im Rahmen des hier
  beschriebenen Projektes wird eine Prozesskette so organisiert, dass die einzelnen Datenquellen,
  Prozessschritte (manuelle Datenerfassung, GIS-Analysen, KI) und (Zwischen-)Ergebnisse flexibel
  kombinierbar sind. Das hierfür entworfene Rahmenwerk sowie erste Zwischenergebnisse werden
  vorgestellt.%
}%


%%%%%%%%%%%%%%%%%%%%%%%%%%%%%%%%%%%%%%%%%%%

% time: Thursday 11:10
% URL: https://pretalx.com/fossgis2023/talk/fossgis2025-57855-fossgis-bei-opencode-de/

%
\newTimeslot{11:10}
\noindent\abstractHSeins{%
  David Arndt, Torsten Friebe%
}{%
  FOSSGIS bei OpenCode.de%
}{%
}{%
  IT-Sicherheit spielt eine immer größere Rolle. Dabei ist die Zertifizierung ein wichtiger
  Schlüssel um Software in der öffentlichen Verwaltung einzusetzen. Anfang 2024 ist das Zentrum für
  Digitale Souveränität GmbH gestartet, um Open Source in der öffentlichen Verwaltung
  voranzutreiben.
  Der Nutzen für die FOSSGIS Community soll in diesem Vortrag beleuchtet werden.%
}%


%%%%%%%%%%%%%%%%%%%%%%%%%%%%%%%%%%%%%%%%%%%

% time: Thursday 11:10
% URL: https://pretalx.com/fossgis2023/talk/fossgis2025-57842-barrierefreies-routing-mit-motis/

%

\noindent\abstractHSzwei{%
  Felix Gündling%
}{%
  Barrierefreies Routing mit MOTIS%
}{%
}{%
  MOTIS v2 routet nicht nur die erste und letzte Meile auf OpenStreetMap Daten, sondern auch
  Umstiegswege. Um dies effizient auch für große Gebiete zu ermöglichen und Stockwerke korrekt zu
  behandeln, wurde ein skalierbarer, profilbasierter Router (OSR) entwickelt. Hierbei wird in der
  Verbindungssuche die zeitliche Verfügbarkeit von Aufzügen und Einstiegshilfen am Bahnhof
  berücksichtigt. Wege und andere Informationen werden nach Level unterteilt dargestellt.%
}%


%%%%%%%%%%%%%%%%%%%%%%%%%%%%%%%%%%%%%%%%%%%

% time: Thursday 11:10
% URL: https://pretalx.com/fossgis2023/talk/fossgis2025-58065-verarbeitung-offener-satellitendaten-mit-freier-software-fr-die-visuelle-anwendung/

%

\noindent\abstractHSdrei{%
  Christoph Hormann%
}{%
  Verarbeitung offener Satellitendaten mit freier Software für die visuelle Anwendung%
}{%
}{%
  Was steckt dahinter, wenn man in einem interaktiven Kartendienst im Internet die
  Satellitenbild-Ebene betrachtet?  Dieser Vortrag gibt einen Einblick auf Grundlage von 20 Jahren
  praktischer Erfahrung in der Produktion derartiger Darstellungen.%
}%


%%%%%%%%%%%%%%%%%%%%%%%%%%%%%%%%%%%%%%%%%%%

% time: Thursday 11:10
% URL: https://pretalx.com/fossgis2023/talk/fossgis2025-57449-10-jahre-opensensemap-neue-features-fr-die-zukunft-der-umweltdaten/

%

\noindent\abstractHSvier{%
  Frederick Bruch%
}{%
  10 Jahre openSenseMap: Neue Features für die Zukunft der Umweltdaten%
}{%
}{%
  Die *openSenseMap* ist eine interaktive Karte für Umweltdaten, die von Schulen, Städten und
  Bürge\textbackslash1(\textbackslash2) weltweit genutzt wird. Seit 2014 wurden über 14.000 Sensoren registriert und mehr als
  10 Milliarden Messwerte in Echtzeit bereitgestellt. Die neue Version verbessert die
  Datenvisualisierung und bietet viele weitere neue Features. In unserem Lightning-Talk stellen wir
  diese Neuerungen vor und feiern 10 Jahre openSenseMap, die einen wichtigen Beitrag zu Forschung
  und Bildung leistet.%
}%


%%%%%%%%%%%%%%%%%%%%%%%%%%%%%%%%%%%%%%%%%%%

% time: Thursday 11:10
% URL: https://pretalx.com/fossgis2023/talk/fossgis2025-58048-anwendertreffen-lizmap-webclient/

%

\noindent\abstractAnwBoFeins{%
  Günter Wagner%
}{%
  Anwendertreffen Lizmap-Webclient%
}{%
}{%
  Die deutschsprachige Anwendergruppe für den WebClient Lizmap möchte das Treffen zum
  Erfahrungsaustausch nutzen.
  Teilnehmer können ihre eigenen, mit QGIS-Server und Lizmap realisierten WebGIS-Projekte
  vorstellen. Ferner kann über aktuelle Fragen/Probleme und zukünftige, gewünschte Erweiterungen in
  Lizmap diskutiert werden.
  Das Anwendertreffen richtet sich sowohl an neu Interessierte, als auch an Anwender, die bereits
  mit Lizmap arbeiten.%
}%


%%%%%%%%%%%%%%%%%%%%%%%%%%%%%%%%%%%%%%%%%%%

% time: Thursday 11:10
% URL: https://pretalx.com/fossgis2023/talk/fossgis2025-58141-gbd-websuite-anwendertreffen/

%

\noindent\abstractAnwBoFzwei{%
  Otto Dassau%
}{%
  GBD WebSuite Anwendertreffen%
}{%
}{%
  Die GBD WebSuite ist eine flexible und leistungsstarke Open Source WebGIS Plattform zur
  Geodatenverarbeitung (https://gbd-websuite.de). Sie wird seit 2017 entwickelt und deutschlandweit
  von Kommunen und privaten Unternehmen genutzt. Wir möchten die FOSSGIS für den Austausch zwischen
  Anwendern nutzen und Interessierten das Open Source Projekt vorstellen.%
}%


%%%%%%%%%%%%%%%%%%%%%%%%%%%%%%%%%%%%%%%%%%%

% time: Thursday 11:10
% URL: https://pretalx.com/fossgis2023/talk/fossgis2025-58292-herausforderungen-bei-der-beschriftungs-steuerung-im-qgis/

%

\noindent\abstractAnwBoFdrei{%
  Claas Leiner%
}{%
  Herausforderungen bei der Beschriftungs-Steuerung im QGIS%
}{%
}{%
  Die Möglichkeiten zur  Beschriftungssteuerung im QGIS sind ebenso beeindruckend wie
  herausfordernd. In der Fragestunde können Probleme der Platzierung, der Größensteuerung  sowie
  Einpassung in Polygonen uns vieles andere Thema sein. Wobei ich natürlich nicht alles weis und
  mich freue, wenn Leute mit den unterschiedlichsten Erfahrungsständen zur Fragestunde kommen, um so
  einen spannenden Erfahrungsaustausch zu ermöglichen%
}%


%%%%%%%%%%%%%%%%%%%%%%%%%%%%%%%%%%%%%%%%%%%

% time: Thursday 11:15
% URL: https://pretalx.com/fossgis2023/talk/fossgis2025-58191-verkehrliche-manahmen-aus-historischen-osm-daten-identifizieren/

%
\newTimeslot{11:15}
\noindent\abstractHSvier{%
  Simon Metzler%
}{%
  Verkehrliche Maßnahmen aus historischen OSM-Daten identifizieren%
}{%
}{%
  Eine zentrale Datenbank zu verkehrlichen Maßnahmen fehlt in Deutschland. Dieses Projekt entwickelt
  einen Prozess, der mithilfe von OSM-Daten und ohsome versucht verkehrliche Maßnahmen räumlich und
  zeitlich zu identifizieren. Anhand von Prototypen und Szenarien für Benchmarking soll die Methode
  getestet werden. Die Herausforderungen bestehen insbesondere in der Differenzierung echter
  Maßnahmen von anderen Änderungen und der Komplexität bei Geometrieanpassungen.%
}%


%%%%%%%%%%%%%%%%%%%%%%%%%%%%%%%%%%%%%%%%%%%

% time: Thursday 11:20
% URL: https://pretalx.com/fossgis2023/talk/fossgis2025-57632-openstreetmap-featureinfo/

%
\newTimeslot{11:20}
\noindent\abstractHSvier{%
  Jakob Miksch%
}{%
  OpenStreetMap FeatureInfo%
}{%
}{%
  Der Vortrag stellt eine Implementierung vor, die es erlaubt Informationen zu OSM Objekten um einen
  Ort herum über eine einfache Schnittstelle abzufragen. Die Idee dahinter ist, das Rad nicht neu zu
  erfinden, sondern bewährte Software-Komponenten zu kombinieren und auf etablierte Standards zu
  setzen.%
}%


%%%%%%%%%%%%%%%%%%%%%%%%%%%%%%%%%%%%%%%%%%%

% time: Thursday 11:25
% URL: https://pretalx.com/fossgis2023/talk/fossgis2025-58204-pnktlich-zur-uni-bus-tracking-mit-der-efa-json-api/

%
\newTimeslot{11:25}
\noindent\abstractHSvier{%
  Christoph Friedrich%
}{%
  Pünktlich zur Uni? Bus-Tracking mit der EFA JSON API%
}{%
}{%
  Im "`Bahnland Bayern"' sind Echtzeit-Mobilitätsdaten über eine Schnittstelle namens "`EFA JSON API"
  verfügbar. Indem man diese mitschneidet, kann eine Pünktlichkeitsstatistik z.B. der Würzburger
  Stadtbusse erstellt werden, die aufgrund der nachwievor fehlenden Straßenbahnanbindung der Uni
  gerade zum morgendlichen Vorlesungsbeginn nicht prickelnd aussieht.%
}%


%%%%%%%%%%%%%%%%%%%%%%%%%%%%%%%%%%%%%%%%%%%

% time: Thursday 11:45
% URL: https://pretalx.com/fossgis2023/talk/fossgis2025-58037-xplanung-mit-open-source-software/

%
\newTimeslot{11:45}
\noindent\abstractHSeins{%
  Torsten Friebe%
}{%
  XPlanung mit Open Source Software%
}{%
}{%
  Im April 2022 wurde der Quellcode der Software xPlanBox der Firma lat/lon im Rahmen eines
  Pilotprojekts auf der OpenCoDE-Plattform des BMI veröffentlicht. Seitdem wird die Software
  kontinuierlich weiterentwickelt und kommt im Rahmen des Onlinezugangsgesetz (OZG) und des
  "`Einer-für-Alle"-Prinzips (EfA) zum Einsatz. Der Vortrag stellt kurz die wichtigsten Neuerungen
  der Version 8.0 und den erweiterten Möglichenkeiten für den Betrieb auf der Kubernetes-Plattform
  vor.%
}%


%%%%%%%%%%%%%%%%%%%%%%%%%%%%%%%%%%%%%%%%%%%

% time: Thursday 11:45
% URL: https://pretalx.com/fossgis2023/talk/fossgis2025-58295-online-karten-fr-die-verkehrswende-mit-opendata-und-foss/

%

\noindent\abstractHSzwei{%
  Johanna Klitzschmüller%
}{%
  Online-Karten für die Verkehrswende mit OpenData und FOSS%
}{%
}{%
  Mit OpenData und freier/open-source Software lässt sich eine Online-Kartenanwendung bauen, mit der
  sich die Straßenraumaufteilung der Stadt Kiel darstellen und untersuchen lässt. Damit kann diese
  Online-Karte ein weiteres Werkzeug im Kampf für eine Verkehrswende sein.%
}%


%%%%%%%%%%%%%%%%%%%%%%%%%%%%%%%%%%%%%%%%%%%

% time: Thursday 11:45
% URL: https://pretalx.com/fossgis2023/talk/fossgis2025-58080-mit-enmap-box-und-osm-durch-namibia-ein-hyperspektraler-praxisbericht/

%

\noindent\abstractHSdrei{%
  Benjamin Jakimow%
}{%
  Mit EnMAP-Box und OSM durch Namibia: ein hyperspektraler Praxisbericht%
}{%
}{%
  Verbuschung von Savannen bedroht deren Artenvielfalt und Nutzung als Weideland. Unser Vortrag
  zeigt, wie wir Verbuschung in Namibia bei Feldarbeiten mithilfe von QGIS, der EnMAP-Box und
  anderer freier und open-source Software dokumentiert haben, um sie anschließend mithilfe von
  EnMAP- und anderen Satellitendaten zu beschreiben.%
}%


%%%%%%%%%%%%%%%%%%%%%%%%%%%%%%%%%%%%%%%%%%%

% time: Thursday 11:45
% URL: https://pretalx.com/fossgis2023/talk/fossgis2025-57656-digitaler-zwilling-niedersachsen-auf-basis-der-unreal-engine-und-lgln-open-data/

%

\noindent\abstractHSvier{%
  Julian Müller, Vincent-Aleister Raveling%
}{%
  Digitaler Zwilling Niedersachsen auf Basis der Unreal Engine und LGLN Open Data%
}{%
}{%
  Dieses Projekt aus dem Bereich "`digitaler Zwilling"' erkundet die Verbindbarkeit von offenen
  Geodaten des LGLNs mit der Unreal Engine 5 um moderne Echtzeitanalysen zu entwickeln. Als
  Szenegrundlage dient das Geländemodell mit digitalen Orthophotos zur Texturierung und LOD 1 und 2
  Gebäudemodellen. Auch 3D-Meshes und Punktwolken können geladen werden. Alle Daten werden aus der
  Cloud bezogen, was die Dateigröße der Anwendung an sich gering hält.%
}%


%%%%%%%%%%%%%%%%%%%%%%%%%%%%%%%%%%%%%%%%%%%

% time: Thursday 12:20
% URL: https://pretalx.com/fossgis2023/talk/fossgis2025-57514-modellgetriebene-xplanung-von-uml-zur-ogc-api-for-features/

%
\newTimeslot{12:20}
\noindent\abstractHSeins{%
  Tobias Kraft%
}{%
  Modellgetriebene XPlanung: von UML zur OGC API for Features%
}{%
}{%
  Basierend auf einem modellgetriebenen Ansatz können komplexe XPlanungsdaten mit dem Python-Projekt
  XPlan-Tools verarbeitet und manipuliert werden. So können sie u.a. in eine PostGIS-Datenbank
  überführt und mit ldproxy ohne Weiteres per OGC API for Features veröffentlicht werden. Darüber
  hinaus sind eine Versionsmigration (XPlanung v5 auf v6) sowie die Transformation der
  XPlanungsdaten nach INSPIRE PLU implementiert.%
}%


%%%%%%%%%%%%%%%%%%%%%%%%%%%%%%%%%%%%%%%%%%%

% time: Thursday 12:20
% URL: https://pretalx.com/fossgis2023/talk/fossgis2025-58228-erreichbarkeitsanalyse-in-lndlichen-rumen-potenziale-des-radverkehrs/

%

\noindent\abstractHSzwei{%
  Simon Metzler, Caroline Huth%
}{%
  Erreichbarkeitsanalyse in ländlichen Räumen: Potenziale des Radverkehrs%
}{%
}{%
  Der Beitrag untersucht die Erreichbarkeit von Versorgungszielen in ländlichen Räumen per Rad. Mit
  einer offenen Methode werden Isochronen für Startpunkte berechnet um deren Erreichbarkeit zu
  Attraktionspunkten zu bewerten. Die Verfahrenspipeline, entwickelt in Python, ermöglicht eine
  reproduzierbare Analyse und Vergleichbarkeit zwischen Routingprofilen. Die Methode wurde
  exemplarisch in Kall getestet und soll nun um zusätzliche Routingprofile, Attraktionspunkte und
  Gebiete erweitert werden.%
}%


%%%%%%%%%%%%%%%%%%%%%%%%%%%%%%%%%%%%%%%%%%%

% time: Thursday 12:20
% URL: https://pretalx.com/fossgis2023/talk/fossgis2025-58254-klimaanpassungsrelevante-vegetationsindikatoren-mit-hilfe-von-sentinel-2-zeitreihen/

%

\noindent\abstractHSdrei{%
  Benjamin Stöckigt%
}{%
  Klimaanpassungsrelevante Vegetationsindikatoren mit Hilfe von Sentinel-2 Zeitreihen%
}{%
}{%
  Im Vortrag wird die Methodik zur Herleitung von Vegetationsindikatoren wie Grünvolumen und
  Beschirmungsgrad auf Grundlage von Satellitendaten vorgestellt. Außerdem werden weiterführende
  Analysen für klimaanpassungsrelevante Fragestellungen präsentiert. Die Indikatoren werden jährlich
  aktuell bereitgestellt und ermöglichen aufbauende Analysen auch außerhalb des urbanen Kontexts.
  Methodik und Daten stehen unter CC BY-NC und GPL-V.3 zur Verfügung.%
}%


%%%%%%%%%%%%%%%%%%%%%%%%%%%%%%%%%%%%%%%%%%%

% time: Thursday 14:15
% URL: https://pretalx.com/fossgis2023/talk/fossgis2025-57887-kinder-karten-open-source/

%
\newTimeslot{14:15}
\noindent\abstractHSeins{%
  Dr. Roland Olbricht%
}{%
  Kinder, Karten, Open Source%
}{%
}{%
  Wie begeistern wir die nächste Generation für Open Source und OpenStreetMap?
  Der Vortrag soll eine Bestandsaufnahme bieten, was es an Bildungs- und Begeisterungsmaterial und
  -konzepten für verschiedene Altersgruppen unter 18 zum Raum Erfahren schon gibt. Es ist Zeit zu
  überlegen, welche Angebote wir machen können und wollen, um daran mit Open Source und
  OpenStreetMap anzuknüpfen.%
}%


%%%%%%%%%%%%%%%%%%%%%%%%%%%%%%%%%%%%%%%%%%%

% time: Thursday 14:15
% URL: https://pretalx.com/fossgis2023/talk/fossgis2025-57539-fahrzeugortung-db-regio-mehr-als-gps/

%

\noindent\abstractHSzwei{%
  Stefan Kowski%
}{%
  Fahrzeugortung DB Regio~-- mehr als GPS%
}{%
}{%
  Der Praxisvortrag gibt einen Einblick in die Fahrzeugortung von DB Regio und zeigt, welche Open
  Source-Software in verschiedenen Anwendungsbereichen zum Einsatz kommt.%
}%


%%%%%%%%%%%%%%%%%%%%%%%%%%%%%%%%%%%%%%%%%%%

% time: Thursday 14:15
% URL: https://pretalx.com/fossgis2023/talk/fossgis2025-58119-landesweite-datenerfassungen-organisieren-und-effizient-gestalten-mit-qgis/

%

\noindent\abstractHSdrei{%
  Gordon Schlolaut, Ingo Pfisterer%
}{%
  Landesweite Datenerfassungen organisieren und effizient gestalten mit QGIS%
}{%
}{%
  In diesem Vortrag wird die Entwicklung eines komplexen QGIS-Plugin vorgestellt, das den Prozess
  einer landesweiten Datenerhebung, mit einer Vielzahl an Einzelaufträgen, vereinfacht und
  standardisiert. In der Vergangenheit erforderten diese einzelnen Datenerfassungsaufträge eine
  umfangreiche manuelle Vorbereitung, Datenvalidierung und Integration in einen zentralen
  Datenbestand. Am Beispiel der Hessischen Lebensraum- und Biotopkartierung demonstrieren wir das
  Potential maßgeschneiderter Plugins.%
}%


%%%%%%%%%%%%%%%%%%%%%%%%%%%%%%%%%%%%%%%%%%%

% time: Thursday 14:15
% URL: https://pretalx.com/fossgis2023/talk/fossgis2025-58053-qfield-plugins-beispiele-und-mglichkeiten/

%

\noindent\abstractHSvier{%
  Michael Schmuki%
}{%
  QField Plugins~-- Beispiele und Möglichkeiten%
}{%
}{%
  In diesem Lightning Talk tauchen wir in die faszinierende Welt der QField-Plugins ein und
  entdecken ihre vielfältigen Anwendungsmöglichkeiten. Es wird eine Auswahl an praktischen
  Beispielen vorgestellt und aufgezeigt, wie diese Plugins installiert und genutzt werden können.
  Lassen Sie sich inspirieren und erfahren Sie, wie Sie QField noch effektiver in Ihren Projekten
  einsetzen können.%
}%


%%%%%%%%%%%%%%%%%%%%%%%%%%%%%%%%%%%%%%%%%%%

% time: Thursday 14:15
% URL: https://pretalx.com/fossgis2023/talk/fossgis2025-57450-opensensemap-anwendertreffen/

%

\noindent\abstractAnwBoFeins{%
  Frederick Bruch%
}{%
  openSenseMap Anwendertreffen%
}{%
}{%
  Die openSenseMap ist eine frei nutzbare interaktive Karte für Umweltdaten, die von Schulen,
  Forschungseinrichtungen und Bürger
  weltweit verwendet wird. Beim ersten Anwendertreffen wird die neueste Version der Plattform
  vorgestellt, inklusive neuer Features zur besseren Visualisierung und Analyse von Daten.
  Teilnehmer:innen
  können sich austauschen, Feedback geben und Ideen zur Weiterentwicklung der Plattform diskutieren.
  Alle Interessierten sind herzlich willkommen!%
}%


%%%%%%%%%%%%%%%%%%%%%%%%%%%%%%%%%%%%%%%%%%%

% time: Thursday 14:15
% URL: https://pretalx.com/fossgis2023/talk/fossgis2025-57535-qwc2-anwendertreffen/

%

\noindent\abstractAnwBoFzwei{%
  Daniel Cebulla%
}{%
  QWC2 Anwendertreffen%
}{%
}{%
  Der QWC2 (QGIS Web Client 2) ist die offizielle WebGIS-Anwendung des QGIS Projektes. Das Treffen
  soll QWC2-Anwendern und -Administratoren die Möglichkeit geben, eigene Erfahrungen mit anderen
  Anwendern zu teilen und neue Kontakte zu knüpfen. Teilnehmer können ihre eigenen, mit QWC2
  realisierten WebGIS-Projekte vorstellen und gemeinsam evtl. auftretende Probleme diskutieren oder
  anderen Tipps geben.
  Alle Interessenten sind herzlich eingeladen, beim Anwendertreffen vorbeizuschauen!%
}%


%%%%%%%%%%%%%%%%%%%%%%%%%%%%%%%%%%%%%%%%%%%

% time: Thursday 14:15
% URL: https://pretalx.com/fossgis2023/talk/fossgis2025-58236-karten-bausteine-eines-digital-zugnglichen-staates-/

%

\noindent\abstractAnwBoFdrei{%
  Mathias Großklaus, Robin Pfaff%
}{%
  Karten~-- Bausteine eines digital zugänglichen Staates?%
}{%
}{%
  Die öffentliche Verwaltung bieten ihren Bürger:innen eine Vielzahl kartenbasierter Angebote an.
  Diese sind sinnvoll~-- oft jedoch nicht intuitiv zu bedienen, auf Nischenthemen beschränkt,
  uneinheitlich designed und schwer auffindbar. Hier wird demokratisches Potenzial verschenkt:
  Nutzendenfreundlichere Karten könnten ein wichtiger Bestandteil eines digital zugänglichen Staates
  sein. Im Workshop wollen wir gemeinsam diskutieren, wie wir das zusammen mit der OSM-Community
  erreichen können.%
}%


%%%%%%%%%%%%%%%%%%%%%%%%%%%%%%%%%%%%%%%%%%%

% time: Thursday 14:20
% URL: https://pretalx.com/fossgis2023/talk/fossgis2025-58296-qgis-fernsteuern-kickstart-mit-pyqgis-zur-automatisierung/

%
\newTimeslot{14:20}
\noindent\abstractHSvier{%
  Dr.-Ing. Helge Staedtler%
}{%
  QGIS fernsteuern: Kickstart mit PyQGIS zur Automatisierung%
}{%
}{%
  QGIS ist ein mächtiges Werkzeug, aber auch ein komplexes Werkzeug. Man muss oft eine ganze Menge
  Einstellungen vornehmen, um zu einem gewünschten Ergebnis zu kommen. Will man das Ergebnis nicht
  nur einmal erzielen, darf man nicht in der Zwischenzeit vergessen, wie das nochmal genau ging. Es
  kann einfacher sein mit dem Python Modul **PyQGIS** wiederkehrende Operationen zu automatisieren.%
}%


%%%%%%%%%%%%%%%%%%%%%%%%%%%%%%%%%%%%%%%%%%%

% time: Thursday 14:25
% URL: https://pretalx.com/fossgis2023/talk/fossgis2025-57634-qgis-server-per-rest-api-konfigurieren/

%
\newTimeslot{14:25}
\noindent\abstractHSvier{%
  Jakob Miksch%
}{%
  QGIS Server per REST API konfigurieren%
}{%
}{%
  Das Projekt QSA (QGIS Server Administrator) ermöglicht es QGIS Projekte über eine
  REST-Schnittstelle zu konfigurieren.%
}%


%%%%%%%%%%%%%%%%%%%%%%%%%%%%%%%%%%%%%%%%%%%

% time: Thursday 14:50
% URL: https://pretalx.com/fossgis2023/talk/fossgis2025-58025-openstreetmap-ist-doch-vollstndig-/

%
\newTimeslot{14:50}
\noindent\abstractHSeins{%
  Michael Reichert%
}{%
  OpenStreetMap ist doch vollständig …%
}{%
}{%
  OpenStreetMap ist letztes Jahr 20 Jahre zwar alt geworden. Dennoch ist die Datensammlung weder
  vollständig noch fertig. Der Erfassungsgrad in den einzelnen Themenfeldern ist regional äußerst
  unterschiedlich. Der Vortrag zeigt auf, wie man Gegenden und Themen findet, in denen OpenStreetMap
  noch Aufholbedarf hat. Zielgruppe des Vortrags sind Mapper, die vor Ort Daten erheben möchten. Die
  gezeigten Methoden eignen sich auch, um spannende Reiseziele zu finden.%
}%


%%%%%%%%%%%%%%%%%%%%%%%%%%%%%%%%%%%%%%%%%%%

% time: Thursday 14:50
% URL: https://pretalx.com/fossgis2023/talk/fossgis2025-58239-dfs-deutsche-flugsicherung-open-source-und-sicherheit-bei-uas-ein-erfahrungsbericht/

%

\noindent\abstractHSzwei{%
  Vera Werner%
}{%
  DFS Deutsche Flugsicherung: Open Source und Sicherheit bei UAS? Ein Erfahrungsbericht%
}{%
}{%
  Unmanned Aircraft Systems (UAS) etablieren sich als neue Verkehrsteilnehmer im Luftraum. Die DFS
  unterstützt deren sichere Integration durch die Bereitstellung relevanter Geodaten und setzt dabei
  auf freie Open Source Software (FOSS) im Geodatenmanagement. Fünf Jahre nach Gründung des
  Geodatenmanagement-Teams bieten wir Einblicke in Herausforderungen bei der Datenbereitstellung,
  die Akzeptanz und den Mehrwert von FOSS sowie die Entwicklung der gesetzlichen Rahmenbedingungen.%
}%


%%%%%%%%%%%%%%%%%%%%%%%%%%%%%%%%%%%%%%%%%%%

% time: Thursday 14:50
% URL: https://pretalx.com/fossgis2023/talk/fossgis2025-58217-qgis-im-glasfaser-ausbau-der-deutschen-telekom/

%

\noindent\abstractHSdrei{%
  Felix von Studsinske, Torsten Drey%
}{%
  QGIS im Glasfaser-Ausbau der Deutschen Telekom%
}{%
}{%
  Der Anschluss von Haushalten mit Glasfaser durch die Deutsche Telekom ist zu einem hohen Grad
  automatisiert und wird massiv durch den Einsatz von Geodaten und freier Software gestützt. Ein
  wichtiger Schritt im Planungsprozess ist die sog. Detailplanung für ein Ausbaugebiet bei der
  mittels QGIS und dem entwickelten Plugin "`Plan[Goo]” der Trassenverlauf, die Kosten und der
  benötigte Materialeinsatz bestimmt werden. Der Vortrag gibt einen Überblick über den gesamten
  Prozess und Plan[Goo].%
}%


%%%%%%%%%%%%%%%%%%%%%%%%%%%%%%%%%%%%%%%%%%%

% time: Thursday 14:50
% URL: https://pretalx.com/fossgis2023/talk/fossgis2025-58290-zaubern-mit-dem-geopackage/

%

\noindent\abstractHSvier{%
  Claas Leiner%
}{%
  Zaubern mit dem Geopackage%
}{%
}{%
  Das Geopackage kann nicht nur Daten verpacken sondern ganz schnell im Hintergrund  Arbeit
  verrichten, die  anschließend im QGIS wie Zauberei sichtbar wird.  Die Verschneidung zweier Layer
  aktualisiert sich, wenn die Quell-Layer geändert werden, ohne das ein weiterer Prozess aufgerufen
  werden muss. Aggregierungen und Auswertungen sind stets auf dem Stand.
  Schließlich ist das Geopackage eine SQLite-Datenbank, in welcher der SQL-Zauberstab zum effektiven
  GIS-Helferlein wird.%
}%


%%%%%%%%%%%%%%%%%%%%%%%%%%%%%%%%%%%%%%%%%%%

% time: Thursday 15:25
% URL: https://pretalx.com/fossgis2023/talk/fossgis2025-58235-overpass-turbo-goes-postgis/

%
\newTimeslot{15:25}
\noindent\abstractHSeins{%
  Frederik Ramm%
}{%
  Overpass Turbo goes PostGIS%
}{%
}{%
  Der Vortrag stellt einen Dienst vor, der es erlaubt, das weit verbreitete "`Overpass
  Turbo"-Webfrontend mit einer PostGIS-Datenbank anstelle einer Overpass-Datenbank zu verwenden und
  stellt einige Feature- und Performance-Vergleiche an.%
}%


%%%%%%%%%%%%%%%%%%%%%%%%%%%%%%%%%%%%%%%%%%%

% time: Thursday 15:25
% URL: https://pretalx.com/fossgis2023/talk/fossgis2025-57031-geodatenmanagement-in-einer-baubehrde-wna-nord-ostsee-kanal-/

%

\noindent\abstractHSzwei{%
  Niny Zamora%
}{%
  Geodatenmanagement in einer Baubehörde (WNA Nord Ostsee Kanal)%
}{%
}{%
  Bauprojekte binden große Mengen an finanziellen, zeitlichen und personellen Ressourcen.
  Geodatenmanagement kann dabei helfen, diese effizient zu verwalten. Im fachlichen Kontext einer
  Baubehörde ist das Bewusstsein dafür nicht an allen Stellen gleich stark ausgeprägt, da es oft an
  GIS-Kenntnissen fehlt. Dieses Potenzial auszuschöpfen, ist eine Herausforderung. In diesem
  Vortrag, sollen die QGIS-Erweiterungen "`Baufeldverwaltung"' und "`Kampfmittelverwaltung"' vorgestellt
  werden.%
}%


%%%%%%%%%%%%%%%%%%%%%%%%%%%%%%%%%%%%%%%%%%%

% time: Thursday 15:25
% URL: https://pretalx.com/fossgis2023/talk/fossgis2025-58221-web-trifft-desktop/

%

\noindent\abstractHSdrei{%
  Matthias Kuhn%
}{%
  Web trifft Desktop%
}{%
}{%
  Mit realen Anwendungsfällen in QGIS und QField zeigen wir, wie das Django-Framework konsumierbare
  Geodatenebenen als OGC API~-- Features-Endpunkte erzeugen kann. Indem sowohl das Datenmodell als
  auch die Geschäftslogik in Python mit dem Django-ORM definiert werden, lassen sich
  Herausforderungen umgehen, die häufig bei herkömmlichen Datenbankansätzen auftreten. Auf diese
  Weise demonstrieren wir, wie die Nutzung von Django zu interessanten Perspektiven für solche
  Anwendungen führen kann.%
}%


%%%%%%%%%%%%%%%%%%%%%%%%%%%%%%%%%%%%%%%%%%%

% time: Thursday 16:45
% URL: https://pretalx.com/fossgis2023/talk/fossgis2025-57945-automatische-veredelung-von-offenen-nahe-echtzeit-daten/

%
\newTimeslot{16:45}
\noindent\abstractHSeins{%
  Stephan Holl, Christian Autermann, Benedikt Gräler%
}{%
  Automatische Veredelung von offenen Nahe-Echtzeit-Daten%
}{%
}{%
  Vorgestellt wird eine Plattform zur Echtzeit-Verarbeitung von Erdbeobachtungs- und Wetterdaten,
  die in der Bewältigung von Extremereignissen unterstützt. Open-Data-Quellen wie EFAS, GLOFAS und
  DWD-Radardaten werden automatisiert auf einer skalierbaren Kubernetes-Plattform verarbeitet und
  z.B. den Versicherungsunternehmen nahezu in Echtzeit zur Verfügung gestellt. Der Vortrag erklärt
  das Plattformkonzept und den Mehrwert für verschiedene Branchen.%
}%


%%%%%%%%%%%%%%%%%%%%%%%%%%%%%%%%%%%%%%%%%%%

% time: Thursday 16:45
% URL: https://pretalx.com/fossgis2023/talk/fossgis2025-57307-qfield-neue-strategie-und-anwendungspotentiale/

%

\noindent\abstractHSzwei{%
  Berit Mohr%
}{%
  QField: Neue Strategie und Anwendungspotentiale%
}{%
}{%
  Mit über 1 Million Downloads und 350.000 aktiven Benutzern wird QField als digitales öffentliches
  Gut anerkannt, dass die UN-Ziele für nachhaltige Entwicklung unterstützt.
  Anwendungsfälle aus der Praxis zeigen, wie FeldarbeiterInnen aus aller Welt, Ihre Datenlücken
  schließen, um qualifizierte und bewusste Entscheidungen zu treffen für das Wohl unserer
  Lebensgrundlagen und für eine nachhaltige Zukunft%
}%


%%%%%%%%%%%%%%%%%%%%%%%%%%%%%%%%%%%%%%%%%%%

% time: Thursday 16:45
% URL: https://pretalx.com/fossgis2023/talk/fossgis2025-56980-wie-knnen-openstreetmap-und-qgis-einen-wegewart-untersttzen-/

%

\noindent\abstractHSdrei{%
  Harald Hartmann%
}{%
  Wie können OpenStreetMap und QGIS einen Wegewart unterstützen?%
}{%
}{%
  OpenStreetMap und QGIS unterstützen Wegewarte bei der Pflege von Wegen durch präzise Kartierung,
  Visualisierung und Analyse. OSM ermöglicht das Erfassen und Aktualisieren von Wegen sowie das
  Hinzufügen von Wegedetails und touristischem Inventar entlang des Weges. Mit QGIS lassen sich
  diese Daten analysieren und mit weiteren GIS Daten verknüpfen. Weitere Tools erleichtern zudem den
  mobilen Einsatz und fördern die Zusammenarbeit durch Datenaustausch und Crowdsourcing.%
}%


%%%%%%%%%%%%%%%%%%%%%%%%%%%%%%%%%%%%%%%%%%%

% time: Thursday 16:45
% URL: https://pretalx.com/fossgis2023/talk/fossgis2025-58132-leerstandsmelder-a-thousand-channels-countermappings-aus-der-zivilgesellschaft/

%

\noindent\abstractHSvier{%
  Sebastian Fuchs, Ulf Treger%
}{%
  Leerstandsmelder \& A Thousand Channels~-- Countermappings aus der Zivilgesellschaft%
}{%
}{%
  Leerstandsmelder und A Thousand Channels sind zwei unkommerzielle, zivilgesellschaftliche
  Plattformen, die Openstreetmap und freie Software nutzen. Zwei Projekt-Beteiligte stellen die
  Entwicklung ihrer Plattformen vor, berichten von der Praxis mit OSM und FLOSS Software und von den
  jeweils spezifischen Ansätzen und Fragestellungen, die beim Visualisieren von Inhalten und
  Geoinformationen aufkommen.%
}%


%%%%%%%%%%%%%%%%%%%%%%%%%%%%%%%%%%%%%%%%%%%

% time: Thursday 16:45
% URL: https://pretalx.com/fossgis2023/talk/fossgis2025-58044-open-transport-meetup/

%

\noindent\abstractAnwBoFeins{%
  Volker Krause, Felix Gündling%
}{%
  Open Transport Meetup%
}{%
}{%
  Ein Treffen für alle die mit Mapping, Daten oder Softwareentwicklung rund um Mobilität und
  öffentlichen Personenverkehr zu tun haben.%
}%


%%%%%%%%%%%%%%%%%%%%%%%%%%%%%%%%%%%%%%%%%%%

% time: Thursday 16:45
% URL: https://pretalx.com/fossgis2023/talk/fossgis2025-58118-postnas-suite-anwendertreffen/

%

\noindent\abstractAnwBoFzwei{%
  Astrid Emde%
}{%
  PostNAS-Suite Anwendertreffen%
}{%
}{%
  Die PostNAS-Suite Anwender:innen kommunizieren über die Mailingliste und treffen sich zum
  Austausch. Das nächste Treffen soll auf der FOSSGIS 2025 stattfinden. Hier sollen aktuelle
  Entwicklungen im PostNAS-Suite Projekt vorgestellt und die Erfahrungen der Anwender:innen
  ausgetauscht werden.%
}%


%%%%%%%%%%%%%%%%%%%%%%%%%%%%%%%%%%%%%%%%%%%

% time: Thursday 16:45
% URL: https://pretalx.com/fossgis2023/talk/fossgis2025-57951-herausforderungen-bei-der-ausschreibung-von-fossgis/

%

\noindent\abstractAnwBoFdrei{%
  David Arndt, Torsten Wiebke, Claus Wickinghoff%
}{%
  Herausforderungen bei der Ausschreibung von FOSSGIS%
}{%
}{%
  Seit Juni 2021 beschäftigt sich die Arbeitsgruppe ["Öffentliche Ausschreibungen mit FOSS"' des
  FOSSGIS e.V.](https://www.fossgis.de/wiki/AG\_oeffentl\_Ausschreibungen\_FOSS) mit dem Thema der
  Beschaffung und Vergabe von IT-Lösungen auf Basis von FOSS.
  Wir wollen mit euch und Vertreter:innen aus Verwaltungen sowie verschiedenen Bereichen der
  digitalen Community Fragen zu Herausforderungen und Lösungsmöglichkeiten bei der Ausschreibung und
  dem Projektmanagement zur Entwicklung von FOSS diskutieren.%
}%


%%%%%%%%%%%%%%%%%%%%%%%%%%%%%%%%%%%%%%%%%%%

% time: Thursday 17:20
% URL: https://pretalx.com/fossgis2023/talk/fossgis2025-56913-amtliche-satellitenpositionierung-2-0/

%
\newTimeslot{17:20}
\noindent\abstractHSeins{%
  Andreas Gerschwitz%
}{%
  Amtliche Satellitenpositionierung 2.0%
}{%
}{%
  Der neue Satellitenpositionierungsdienst des Bundes und der Länder ermöglicht die Positionierung
  mit einer Genauigkeit weniger Zentimeter als niederschwelliges Infrastrukturangebot für jeden als
  open data in Echtzeit.%
}%


%%%%%%%%%%%%%%%%%%%%%%%%%%%%%%%%%%%%%%%%%%%

% time: Thursday 17:20
% URL: https://pretalx.com/fossgis2023/talk/fossgis2025-58026-smash-stand-der-technik-der-digitalen-feldkartierungs-app/

%

\noindent\abstractHSzwei{%
  Antonello Andrea%
}{%
  SMASH, Stand der Technik der digitalen Feldkartierungs-app%
}{%
}{%
  Diese Präsentation gibt einen Einblick in den Stand der Technik des SMASH-Ökosystems und seine
  aktuelle Roadmap. SMASH, die "`Smart Mobile App for Surveyor’s Happiness"', ist eine schnelle App
  für die digitale Feldkartierung.%
}%


%%%%%%%%%%%%%%%%%%%%%%%%%%%%%%%%%%%%%%%%%%%

% time: Thursday 17:20
% URL: https://pretalx.com/fossgis2023/talk/fossgis2025-58096-parkraumdaten-aus-osm-mit-der-verwaltung-pflegen-praxisbericht-/

%

\noindent\abstractHSdrei{%
  Tobias Jordans, Alex Seidel%
}{%
  Parkraumdaten aus OSM mit der Verwaltung pflegen (Praxisbericht)%
}{%
}{%
  Seit zwei Jahren kooperiert die Berliner OSM-Community mit dem Bezirksamt
  Friedrichshain-Kreuzberg, um hochwertige Daten zum Parkraum bereitzustellen. Dieser neuartige
  Ansatz zeigt Potenziale auf, wie Herausforderungen in der Nutzung von OSM-Daten für amtliche
  Zwecke. Nach der initialen Datenerhebung geht es darum, Prozesse zu etablieren, um die Daten
  aktuell zu halten. In diesem Praxisbericht berichten wir über unsere Erfahrungen in der
  Kooperation sowie bei der gemeinsamen Datenpflege.%
}%


%%%%%%%%%%%%%%%%%%%%%%%%%%%%%%%%%%%%%%%%%%%

% time: Thursday 17:20
% URL: https://pretalx.com/fossgis2023/talk/fossgis2025-57917-wie-mache-ich-eine-gute-karte-mit-qgis-/

%

\noindent\abstractHSvier{%
  Mathias Gröbe%
}{%
  Wie mache ich eine gute Karte mit QGIS?%
}{%
}{%
  Eine Karte mit QGIS zu erstellen ist einfach~-- aber wie erstellt man eine gute Karte? Die
  Demosession soll Grundlagen der Kartengestaltung vermitteln: wie man gute Farben wählt, wie groß
  oder klein Schriften sein darf und welche Liniendicke gut aussieht. Nach einem kurzen Teil mit
  Theorie folgt die praktische Umsetzung in QGIS von Daten, über die Signaturen bis zum Drucklayout.%
}%


%%%%%%%%%%%%%%%%%%%%%%%%%%%%%%%%%%%%%%%%%%%

% time: Thursday 17:55
% URL: https://pretalx.com/fossgis2023/talk/fossgis2025-58105-neue-methoden-zur-autarken-indoor-positionsbestimmung/

%
\newTimeslot{17:55}
\noindent\abstractHSeins{%
  Philipp Fiedler, Benjamin Würzler%
}{%
  Neue Methoden zur autarken Indoor-Positionsbestimmung%
}{%
}{%
  Wie kann die eigene Position in einem Gebäude bestimmt werden, wenn jede bekannte Infrastruktur
  fehlt? Im Projekt "`Denied Areas"' erforscht das Bundesamt für Kartographie und Geodäsie innovative
  Wege zur autarken Positionsbestimmung~-- unabhängig von WLAN und Bluetooth. Mithilfe tragbarer
  Sensoren werden potenzielle Lösungen untersucht und erprobt. Erleben Sie spannende Einblicke in
  aktuelle Ansätze und Technologien im Bereich der Indoor-Positionsbestimmung und -Navigation.%
}%


%%%%%%%%%%%%%%%%%%%%%%%%%%%%%%%%%%%%%%%%%%%

% time: Thursday 17:55
% URL: https://pretalx.com/fossgis2023/talk/fossgis2025-57817-merginmaps-mobile-geodatenerfassung-beim-regionalverband-ruhr/

%

\noindent\abstractHSzwei{%
  Stefan Overkamp%
}{%
  MerginMaps~-- Mobile Geodatenerfassung beim Regionalverband Ruhr%
}{%
}{%
  Seit 2024 setzt der Regionalverband Ruhr (RVR) die Software MerginMaps ein.
  Auslöser war die gute Integrationsmöglichkeit in die GDI des RVR.
  Wesentliche Funktionen
  - mobile Datenerfassung (Punkt, Linie, Fläche) im Gelände und Synchronisieren ins Backend
  - Weiterbearbeiten in QGIS mittels Plugin
  - Arbeiten auf einem gemeinsamen Datenbestand mit Erkennung von Konflikten
  - Versionierung aller Änderungen und dabei detaillierte Dokumentation%
}%


%%%%%%%%%%%%%%%%%%%%%%%%%%%%%%%%%%%%%%%%%%%

% time: Thursday 17:55
% URL: https://pretalx.com/fossgis2023/talk/fossgis2025-57806-brouter-und-brouter-web-fr-anfnger-und-fortgeschrittene/

%

\noindent\abstractHSdrei{%
  Arndt Brenschede, Serge Bellina%
}{%
  BRouter und BRouter-Web für Anfänger und Fortgeschrittene%
}{%
}{%
  BRouter ist das flexibelste Werkzeug für Streckenplanung auf Basis von OSM-Daten und seit 10
  Jahren besonders bei Radfahrern und Wanderern beliebt. Wir geben zunächst eine Einführung in die
  wichtigsten Verwendungsarten, sowohl mobil als auch am heimischen PC, und richten uns dabei auch
  an Neu-Einsteiger. Im zweiten Teil des Vortrags geht es dann um spannende Neu-Enwicklungen,
  insbesondere um die Berücksichtigung von Umgebungseigenschaften für noch bessere
  Routing-Ergebnisse.%
}%


%%%%%%%%%%%%%%%%%%%%%%%%%%%%%%%%%%%%%%%%%%%
