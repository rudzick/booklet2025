
% time: Thursday 11:10
% URL: https://pretalx.com/fossgis2023/talk/fossgis2025-57965-datenklassifikation-mehrwert-oder-manipulation-/

%
\newSmallTimeslot{}
\noindent\abstractOther{%
  Jochen Schiewe%
}{%
  Datenklassifikation: Mehrwert oder\linebreak Manipulation?%
}{%
}{%
  Standardmethoden zur Einteilung von Werten in Klassen, z. B. für die Herstellung von thematischen
  Karten, garantieren weder den Erhalt räumlicher Muster, noch können sie eine objektive
  Repräsentation bewirken. Der Beitrag macht auf diese Probleme der Datenklassifikation aufmerksam
  und behandelt mögliche Abhilfen.%
}%
{%
  Poster (Zelt)%
}%



%%%%%%%%%%%%%%%%%%%%%%%%%%%%%%%%%%%%%%%%%%%

% time: Thursday 11:20
% URL: https://pretalx.com/fossgis2023/talk/fossgis2025-58172-erfahrungsbericht-merginmaps-fr-die-inventarisierung-von-vogelnestern/

%
%\newSmallTimeslot{}
\noindent\abstractOther{%
  Geneviève Hannes%
}{%
  Erfahrungsbericht: MerginMaps für die Inventarisierung von Vogelnestern%
}{%
}{%
  In Zürich steht eine Gebäudebrüterinventarisierung an und diese soll in einem knappen Zeitrahmen
  mit einer mobilen App erfolgen. MerginMaps ermöglicht es uns, dank der Integration mit QGIS eine
  komplexe Feldaufnahmelogik schnell umsetzen. Der Bestand wird somit direkt vor Ort mit der Mobilen
  App erfasst, während die Daten zentral in unserer eigenen Cloud-Infrastruktur gespeichert werden.
  Eine innovative Lösung, die auf OpenSource Komponenten basiert.%
}%
{%
  Poster (Zelt)%
}%



%%%%%%%%%%%%%%%%%%%%%%%%%%%%%%%%%%%%%%%%%%%

% time: Thursday 11:30
% URL: https://pretalx.com/fossgis2023/talk/fossgis2025-58064-sheepsmeadow-agentenbasierte-modellierung-einfach-erlernen/

%
%\newSmallTimeslot{}
\noindent\abstractOther{%
  Michael Brüggemann%
}{%
  Sheepsmeadow: Agentenbasierte\linebreak Modellierung einfach erlernen%
}{%
}{%
  'Sheepsmeadow' ist ein in Java entwickeltes offenes Simulationsprogramm, das den Nutzer:innen das
  Themengebiet "`Agentenbasierte Modellierung"' (ABM) spielerrisch näher bringen soll. Agenten sind
  Wölfe und Schafe, die miteinander auf einer Wiese interagieren.
  'Sheepsmeadow' gibt zudem die Möglichkeit, eigene Entscheidungsregeln (Actions) direkt im
  Quellcode zu ergänzen und so die Simulation zu verändern. Es soll dabei helfen, die Entwicklung
  von ABMs mithilfe des MASON Frameworks zu erlernen.%
}%
{%
  Poster (Zelt)%
}%



%%%%%%%%%%%%%%%%%%%%%%%%%%%%%%%%%%%%%%%%%%%

% time: Thursday 11:40
% URL: https://pretalx.com/fossgis2023/talk/fossgis2025-65368-2-5d-indoor-karten-auf-basis-von-openstreetmap-daten/

%
%\newSmallTimeslot{11:40}
\noindent\abstractOther{%
  Richard Karl Fuchs, Jacques-Maurice Walther%
}{%
  2.5D Indoor-Karten auf Basis von OpenStreetMap-Daten%
}{%
}{%
  Das Poster stellt eine 2.5D-Darstellung von Indoor-Karten für einen digitalen Infopunkt vor. Durch
  die Kombination aus 2D- und perspektivischer Ansicht mehrerer Etagen wird die Orientierung
  erleichtert. Besondere Herausforderungen waren die Abgrenzung der Etagen und die realitätsnahe
  Darstellung von Treppenhäusern mit {\em Maptalks} und {\em THREE.js}. Zudem wurden
  Barrierefreiheitsaspekte wie ein Rollstuhlmodus integriert. Das Poster zeigt zentrale
  Designentscheidungen und technische Lösungsansätze.%
}%
{%
  Poster (Zelt)%
}%



%%%%%%%%%%%%%%%%%%%%%%%%%%%%%%%%%%%%%%%%%%%

% time: Thursday 11:50
% URL: https://pretalx.com/fossgis2023/talk/fossgis2025-57588-artificial-ground-truth-data-generation-for-map-matching-with-open-source-software/

%
%\newSmallTimeslot{11:50}
\noindent\abstractOther{%
  Adrian Wöltche%
}{%
  Artificial Ground Truth Data\linebreak Generation for Map Matching with\linebreak Open Source Software%
}{%
}{%
  We present a customizable open source pipeline for generating comparably realistic artificial
  ground truth data for map matching. We evaluate the generated data with our own open source map
  matching solution, and other existing open source solutions.%
}%
{%
  Poster (Zelt)%
}%



%%%%%%%%%%%%%%%%%%%%%%%%%%%%%%%%%%%%%%%%%%%

% time: Thursday 12:00
% URL: https://pretalx.com/fossgis2023/talk/fossgis2025-57932-efficient-processing-of-high-volume-spatial-data-with-spark/

%
\newTimeslot{}
\noindent\abstractOther{%
  Alexey Egorov, Jannis Jakobi%
}{%
  Efficient processing of high-volume\linebreak spatial data with Spark%
}{%
}{%
  Efficient retrieval of geospatial data is crucial but presents scaling challenges. During our
  transition from PostgreSQL to Apache Spark, we encountered limitations in spatial indexing. While
  PostgreSQL’s indexing supports efficient queries, this is not directly translatable to Spark. The
  transition required us to create new strategies for managing and querying spatial data
  effectively. In this talk, we’ll share the challenges we faced and the innovative solutions we
  implemented to address them.%
}%
{%
  Poster (Zelt)%
}%



%%%%%%%%%%%%%%%%%%%%%%%%%%%%%%%%%%%%%%%%%%%

% time: Thursday 12:10
% URL: https://pretalx.com/fossgis2023/talk/fossgis2025-57931-scalable-big-data-processing-postgres-drawbacks-eliminated-with-spark/

%
%\newSmallTimeslot{12:10}
\noindent\abstractOther{%
  Alexey Egorov, Jannis Jakobi%
}{%
  Scalable big data processing~-- Postgres’ drawbacks eliminated with Spark%
}{%
}{%
  PostgreSQL is often considered a standard solution for geospatial data processing. However,
  compute costs grow with the data volume and vertical scaling quickly becomes expensive. In
  contrast, distributed processing frameworks allow for horizontal scaling. In this talk, we will
  present our experience with Apache Spark, an open-source framework designed for high-volume data
  processing. We will show the benefits and highlight the challenges we faced during the
  implementation.%
}%
{%
  Poster (Zelt)%
}%



%%%%%%%%%%%%%%%%%%%%%%%%%%%%%%%%%%%%%%%%%%%

% time: Thursday 12:20
% URL: https://pretalx.com/fossgis2023/talk/fossgis2025-65656-ship-energy-demand-prediction-weather-forecasts-vs-onboard-data/

%
%\newSmallTimeslot{12:20}
\noindent\abstractOther{%
  Igor Alexander Galvão Quaresma%
}{%
  Ship Energy Demand Prediction:\linebreak Weather Forecasts vs. Onboard Data%
}{%
}{%
  Die Energieeffizienz in der maritimen Industrie ist entscheidend für die Verringerung der
  Treibhausgasemissionen (GHG). In dieser Studie wird untersucht, wie sich die Verwendung
  verschiedener Wetterdatenplattformen (GFS und ERA5) in einem digitalen Schiffszwilling auswirkt.
  Genauer wird beobachtet, wie sich dies auf den modellierten Energiebedarf des Schiffes für
  bestimmte Trajektorien bei ruhigen, mäßigen und rauen Seebedingungen im Vergleich zu einem in
  Betrieb befindlichen Schiff auswirkt.%
}%
{%
  Poster (Zelt)%
}%



%%%%%%%%%%%%%%%%%%%%%%%%%%%%%%%%%%%%%%%%%%%

% time: Thursday 12:30
% URL: https://pretalx.com/fossgis2023/talk/fossgis2025-62620-sidescantools-open-source-sidescan-processing-software/

%
%\newSmallTimeslot{12:30}
\noindent\abstractOther{%
  Mia Schumacher, Finn Spitz%
}{%
  SidescanTools~-- open source Sidescan Processing Software%
}{%
}{%
  Wir stellen eine neue Open-Source-Software SidescanTools vor. Sie kann Edgetech .jsf Dateien sowie
  das plattformübergreifend lesbare '.xtf'-Dateiformat verarbeiten. SidescanTools erlaubt es, eine
  xtf-Datei zu lesen, Bodenerkennung, Schrägentfernungskorrektur sowie empirischer
  Verstärkungsnormalisierung (Emprirical Gain Normalisation, EGN) anzuwenden. Die Daten können dann
  als Geotiff/netCDF in jedes GIS exportiert werden. Das Programm ist in Arbeit und offen für
  Beiträge aus der community.%
}%
{%
  Poster (Zelt)%
}%



%%%%%%%%%%%%%%%%%%%%%%%%%%%%%%%%%%%%%%%%%%%

% time: Thursday 12:40
% URL: https://pretalx.com/fossgis2023/talk/fossgis2025-58057-qfieldcloud-erweitern-ideen-und-praxisbeispiele/

%
%\newSmallTimeslot{12:40}
\noindent\abstractOther{%
  Michael Schmuki%
}{%
  QFieldCloud erweitern~-- Ideen und\linebreak Praxisbeispiele%
}{%
}{%
  In diesem Talk wird erläutert, wie QFieldCloud durch die Integration zusätzlicher Django-Apps
  erweitert werden kann. Dadurch lassen sich z.B. QField-Projekte generieren, auf Ereignisse aus der
  Feldarbeit reagieren, neue Webseiten und APIs hinzufügen und ganze QGIS-Modelle als
  QFieldCloud-Jobs ausführen.%
}%
{%
  Poster (Zelt)%
}%



%%%%%%%%%%%%%%%%%%%%%%%%%%%%%%%%%%%%%%%%%%%

% time: Thursday 12:50
% URL: https://pretalx.com/fossgis2023/talk/fossgis2025-58248-eine-automatisierte-foss-gdi-zur-exploration-von-erdsystem-forschungsdaten/

%
%\newSmallTimeslot{12:50}
\noindent\abstractOther{%
  Peter Konopatzky%
}{%
  Eine automatisierte FOSS-GDI zur Exploration von Erdsystem-Forschungsdaten%
}{%
}{%
  Ein Poster über den Aufbau und die Automatisierung einer containerisierten Geodateninfrastruktur
  in der Erdsystemforschung. Genutzt wird ein etablierter Stack aus PostGIS, GeoServer und
  GeoNetwork. Die gehosteten OGC-Dienste werden unter Nutzung einer Python-Bibliothek automatisiert
  aktuell gehalten.%
}%
{%
  Poster (Zelt)%
}%



%%%%%%%%%%%%%%%%%%%%%%%%%%%%%%%%%%%%%%%%%%%

% time: Thursday 13:00
% URL: https://pretalx.com/fossgis2023/talk/fossgis2025-58139-an-inventory-of-spatial-machine-learning-packages-in-r/

%
\newTimeslot{}
\noindent\abstractOther{%
  Jakub Nowosad, Hanna Meyer, Jan Linnenbrink%
}{%
  An Inventory of Spatial Machine Learning Packages in R%
}{%
}{%
  Machine learning for spatial problems faces unique challenges, notably spatial dependence.
  Effective modeling requires integrating spatial information and proper validation methods to
  preserve spatial structure. This poster will overview spatial machine learning packages in R,
  focusing on tools for feature engineering, validation, and interpretation. It will also serve as a
  guide for comparing these tools and critically assessing their strengths and limitations.%
}%
{%
  Poster (Zelt)%
}%



%%%%%%%%%%%%%%%%%%%%%%%%%%%%%%%%%%%%%%%%%%%

% time: Thursday 13:10
% URL: https://pretalx.com/fossgis2023/talk/fossgis2025-58087-openstreetmap-daten-zur-geolokalisierung-unbemannter-luftfahrzeuge/

%
%\newSmallTimeslot{13:10}
\noindent\abstractOther{%
  Rebecca Schmidt%
}{%
  OpenStreetMap-Daten zur Geolokalisierung unbemannter Luftfahrzeuge%
}{%
}{%
  Das Poster zeigt, wie OpenStreetMap-Daten mithilfe von Open-Source-Tools für die Geolokalisierung
  unbemannter Luftfahrzeuge aufbereitet werden können und beleuchtet die Chancen und
  Herausforderungen, die mit ihrer Verwendung einhergehen.%
}%
{%
  Poster (Zelt)%
}%



%%%%%%%%%%%%%%%%%%%%%%%%%%%%%%%%%%%%%%%%%%%

% time: Thursday 13:20
% URL: https://pretalx.com/fossgis2023/talk/fossgis2025-57876--k-nndm-kreuzvalidierung-im-geographischen-und-prdiktorraum/

%
\newTimeslot{}
\noindent\abstractOther{%
  Jan Linnenbrink%
}{%
  (k)NNDM~-- Kreuzvalidierung im\linebreak geographischen und Prädiktorraum%
}{%
}{%
  In der räumlichen Modellierung werden oft Kreuzvalidierungsmethoden genutzt, um Modelle zu
  evaluieren und ihre Hyperparameter zu tunen. Es existiert eine Vielzahl an räumlichen
  Kreuzvalidierungsmethoden, welche jedoch nicht universell einsetzbar sind.
  Mit diesem Poster werden wir verschiedene Ansätze systematisch vergleichen, sowie eine neue
  Kreuzvalidierungsmethode und ihre Implementierung in R vorstellen, die sowohl im Prädiktorraum,
  als auch dem geographischen Raum arbeitet.%
}%
{%
  Poster (Zelt)%
}%



%%%%%%%%%%%%%%%%%%%%%%%%%%%%%%%%%%%%%%%%%%%

% time: Thursday 13:30
% URL: https://pretalx.com/fossgis2023/talk/fossgis2025-58403-umweltnavi-eine-interaktive-und-kooperative-plattform-fr-umweltinformationen/

%
%\newSmallTimeslot{13:30}
\noindent\abstractOther{%
  Torben Fiedler, Martin Dittmann%
}{%
  UmweltNAVI: Eine interaktive und\linebreak kooperative Plattform für Umwelt\-informationen%
}{%
}{%
  UmweltNAVI ist eine innovative Open-Source-Smartphone-App, die amtliche und freie Umweltdaten
  standortbezogen visualisiert. Das von Niedersachsen initiierte und ab 2025 mit Rheinland-Pfalz und
  Schleswig-Holstein im Rahmen der VKoopUIS fortgeführte Projekt nutzt eine containerisierte
  Architektur. Das Datenbackend konsolidiert verschiedene Datenquellen und stellt diese über
  standardisierte Schnittstellen zur weiteren Nutzung bereit.%
}%
{%
  Poster (Zelt)%
}%



%%%%%%%%%%%%%%%%%%%%%%%%%%%%%%%%%%%%%%%%%%%

% time: Thursday 13:40
% URL: https://pretalx.com/fossgis2023/talk/fossgis2025-57512-offene-daten-fr-barrierefreie-mobilitt-und-inklusive-reiseketten-im-pv/

%
%\newSmallTimeslot{13:40}
\noindent\abstractOther{%
  René Apitzsch%
}{%
  Offene Daten für barrierefreie Mobilität und inklusive Reiseketten im ÖPV%
}{%
}{%
  Das Poster thematisiert die Rolle offener Daten bei der Förderung barrierefreier Mobilität im
  öffentlichen Personenverkehr (ÖPV). Es zeigt auf, wie durch die Erfassung und Nutzung
  standardisierter Daten Barrieren durch Infrastruktur und Fahrzeuge entlang von Reiseketten
  verringert werden können. Dazu werden Ansätze zur strukturierten Erfassung von Hindernissen im
  Verkehrswesen sowie Möglichkeiten zur Integration dieser Daten in bestehende Systeme dargestellt.%
}%
{%
  Poster (Zelt)%
}%



%%%%%%%%%%%%%%%%%%%%%%%%%%%%%%%%%%%%%%%%%%%

% time: Thursday 13:50
% URL: https://pretalx.com/fossgis2023/talk/fossgis2025-56800-qgis-im-einsatz-fr-die-energiewende/

%
%\newSmallTimeslot{13:50}
\noindent\abstractOther{%
  Jens Tielker, Johannes Quente%
}{%
  QGIS im Einsatz für die Energiewende%
}{%
}{%
  In diesem Vortrag geht es um den Aufbau und Betrieb einer Geo\-daten-Infrastruktur, die in diversen
  Projekten zur Planung der Energiewende eingesetzt wird. Dazu verwenden wir hauptsächlich
  Open-Source-Software wie QGIS, PostgreSQL, PostGIS, QField, R, R-Studio und Shiny Server.%
}%
{%
  Poster (Zelt)%
}%



%%%%%%%%%%%%%%%%%%%%%%%%%%%%%%%%%%%%%%%%%%%

% time: Thursday 14:00
% URL: https://pretalx.com/fossgis2023/talk/fossgis2025-58258-einbindung-von-ogc-apis-in-webtools-fr-wrmestromdaten/

%
\newTimeslot{}
\noindent\abstractOther{%
  Nikolas Ott%
}{%
  Einbindung von OGC APIs in Webtools für Wärmestromdaten%
}{%
}{%
  Für eine kompakte, clientseitige Webanwendung werden externe Daten im Weltwärmestrom Datenbank
  Projekt über OGC APIs bereitgestellt. Auf Nachfrage können so benötigte Werte an den Client
  übermittelt werden. Die entwickelte Architektur wird am Beispiel des "`Digitalen Bohrlochs"
  veranschaulicht. Dabei wird die Temperatur an einer beliebigen Position in Abhängigkeit der Tiefe
  berechnet. Globale Rasterdatensätze wie CRUST1.0 werden z.B. als OGC API~-- Coverages eingebunden.%
}%
{%
  Poster (Zelt)%
}%

\vspace{1.5cm}
\sponsorBoxA{415_GISCAD_Technologies.svg}{0.4\textwidth}{3}{%
\textbf{Bronzesponsor}\\
\noindent\small {\bfseries GISCAD Technologies GmbH \& Co. KG} Wir haben über 30 Jahre Erfahrung in Sachen Geoinformatik~\& GIS. Aktueller Entwicklungsschwerpunkt: Big GeoData~\& Open Data mit Open Source GIS, z.B. landes- und bald bundesweites ALKIS. Unser spezielles Know-How: Big GeoData in PostGIS~\& GeoServer, dazu komfortable WebGIS-Apps auf OpenLayers-Basis.
\normalsize
}%


%%%%%%%%%%%%%%%%%%%%%%%%%%%%%%%%%%%%%%%%%%%
